\chapter{Lab Installation Instructions} \label{ap:lab_installation}

\begin{verbatim}
Install Master Instructions/Notes

Ubuntu 12.04 LTS Server AMD64
drive that is not nor mounted by
the system when it is in standard use.

Installing.

Do a bare bones install:

Boot the install CDROM.
Partitioned the disk:
 90% ext2 (1st partition)
 10% swap (2nd partition)

Use dhcp (auto), in the lab this will map you to some computer named test
something or other.
Auto time
Enable root login and set a password. 
If this doesn't work, create a dummy account.
No auto update
select no extra packages
Go ahead and install grub (can still use lilo).
select "Continue" to reboot to the hard drive

Set root passwd and remove dummy user if needed:
 userdel -r dummy
   
NOTE if installing on new motherboards:
booted from CD, installed gcc with the packages from CD
    installed 10.04 
    put e1000e drivers (1.9.5) onto flashdrive
    used make install to install from src.
    bring eth0 up from there and do the install as usual.

Install additional items:

apt-get update           (list of packages and repositories)
apt-get -y dist-upgrade  (update/upgrade the packages)

Added packages (see packages file)
 (do this before Edits as the added packages create the auto.master file)
 (Warning: installing nis-common asks for the nis domain name)
 scp npickrel@heart:~conf/admin/linuxlabs/packages .
 apt-get -y install $(grep -v ^# packages)

Disable cups
 apt-get -y remove cups
 apt-get -y autoremove

mv /etc/init/gdm.conf /etc/init/gdm.conf.bak
  we do not want an X11 login screen 
Edit /etc/nsswitch.conf
  passwd, group, shadow changed to "files nis"
  hosts changed to files dns
  (activates on new logins only)
Revert to classic gnome:
  vi /etc/X11/Xsession.d/50x11-common_determine-startup
  replace /usr/bin/x-session-manager with /usr/bin/gnome-session-fallback

Modify the autofs files
 Installed the pre-built files.
  cd ~
  wget http://tiger/export/ubuntu/autofs.tgz
  cd /etc
  tar -xzf ~/autofs.tgz
 Description of the contents of autofs.tgz)
  auto.master: 
   /net/aardvark /etc/auto.aardvark
   /net/d1 /etc/auto.d1
   /net/d2 /etc/auto.d2
   /net/d3 /etc/auto.d3
   /net/a1 /etc/auto.a1
   /net/charlotte /etc/auto.charlotte
  auto.aardvark:
   u1 -ro,nolock,timeo=900,nfsvers=2 aardvrk.cecs.csulb.edu:/u1
  auto.a1:
   u1 -soft,intr,noacl,nolock,timeo=900 a1.cecs.csulb.edu:/u1
  auto.charlotte:
   mail -soft,intr,noacl,nolock,timeo=3600 charlotte.cecs.csulb.edu:/var/spool/mail
  auto.d1:
   u1 -soft,intr,suid,noacl,nolock,quota,timeo=9600 d1.cecs.csulb.edu:/u1
   u2 -soft,intr,suid,noacl,nolock,quota,timeo=9600 d1.cecs.csulb.edu:/u2
   u3 -soft,intr,suid,noacl,nolock,quota,timeo=9600 d1.cecs.csulb.edu:/u3
   u4 -soft,intr,suid,noacl,nolock,quota,timeo=9600 d1.cecs.csulb.edu:/u4
   u5 -soft,intr,suid,noacl,nolock,quota,timeo=9600 d1.cecs.csulb.edu:/u5
   u6 -soft,intr,suid,noacl,nolock,quota,timeo=9600 d1.cecs.csulb.edu:/u6
   u7 -soft,intr,suid,noacl,nolock,quota,timeo=9600 d1.cecs.csulb.edu:/u7
   u1BU -soft,intr,noacl,nolock d1.cecs.csulb.edu:/u1BU
   u2BU -soft,intr,noacl,nolock d1.cecs.csulb.edu:/u2BU
   u3BU -soft,intr,noacl,nolock d1.cecs.csulb.edu:/u3BU
   u4BU -soft,intr,noacl,nolock d1.cecs.csulb.edu:/u4BU
   u5BU -soft,intr,noacl,nolock d1.cecs.csulb.edu:/u5BU
   u6BU -soft,intr,noacl,nolock d1.cecs.csulb.edu:/u6BU
   u7BU -soft,intr,noacl,nolock d1.cecs.csulb.edu:/u7BU
  auto.d2,auto.d3: similar to auto.d1
  cd ~
  rm autofs.tgz
  reboot (to activate)

Edit /etc/group modify names for groups 30 and 40
 faculty:x:30
 student:x:40
 staff:x:50

Configure firefox to have the barefoot home page and no disk caching
 (because we can get it to limit it's cache amount based on the account
  quota)
 Edit /etc/xul-ext/ubufox.js
  pref("browser.startup.homepage", "file:/etc/xul-ext/homepage.properties");
  pref("browser.cache.disk.smart_size.enabled", false);
  pref("browser.cache.disk.enable", false);
  pref("browser.cache.offline.enable", false);
 Create file /etc/xul-ext/homepage.properties with following line
  browser.startup.homepage=http://barefoot.cecs.csulb.edu
 
Configure the mail system
 Change the mail link
  cd /var ; rmdir mail; ln -s /net/charlotte/mail
 Modify the mailer definitions
  cd /etc/mail 
  mv sendmail.mc sendmail.mc.bak
  wget http://tiger/export/ubuntu/sendmail.mc
  make
   This will modify the sendmail.cf, use "service sendmail reload" to activate
  sendmail.mc modifications:
   after the last define add:
    dnl # local setup
    define(`PROCMAIL_MAILER_PATH', `/usr/bin/procmail')dnl
    define(`SMART_HOST', `charlotte.cecs.csulb.edu')dnl
   after the last FEATURE add:
    dnl # local setup
    FEATURE(local_procmail, `', `procmail -t -Y -a $h -d $u')dnl
    MASQUERADE_AS(`cecs.csulb.edu')dnl
    FEATURE(`always_add_domain')dnl
    FEATURE(masquerade_envelope)dnl
    FEATURE(masquerade_entire_domain)dnl
    FEATURE(`allmasquerade')dnl
    MASQUERADE_DOMAIN(localhost)dnl
    MASQUERADE_DOMAIN(localhost.localdomain)dnl
   after the last MAILER add:
    dnl # local setup
    MAILER(`procmail')dnl
   delete the default masquerade list below mailer.

Symlink pine
  ln -s /usr/bin/alpine /usr/bin/pine

Edit /etc/init/control-alt-delete.conf
Comment out  the exec shutdown

Copied namemaster.user.notice (script file) into /usr/bin.
  This file contains a notice to the students to use "search account"...
 cd /usr/bin
 wget http://tiger/export/ubuntu/namemaster.user.notice
 chmod 755 /usr/bin/namemaster.user.notice
Replaced chfn,chsh,passwd,ypchfn,ypchsh,yppasswd with links to above notice
 cd
 wget http://tiger/export/ubuntu/set_namemaster.sh
 source set_namemaster.sh
  What this script does:
   1) Backup: /usr/bin/ ypchfn, ypchsh yppasswd passwd chsh chfn to *.org
   2) Replace all with  links to namemaster.usr.notice

Edit /etc/rc.local
  add powersave definition
 /usr/bin/setterm -blank 15 -powersave powerdown -powerdown 30
  add the call to set the screen to tty0 (alt-F1
 /bin/chvt 1

Edit /etc/logrotate.conf
 modify rotate 4 to rotate 13 (keep a semester's worth of backlogs)

Add foobar public key to /root/.ssh/authorized_keys
 cd
 mkdir .ssh
 chmod 700 .ssh
 cd .ssh
 wget http://tiger/export/authorized_keys2
 mv authorized_keys2 authorized_keys
Add foobar and tiger to known hosts (your are still in .ssh)
 wget http://tiger/export/known_hosts

Make all machines have the same ssh key so foobar can push
  cd  # root private directory
  wget http://tiger/export/ssh.tgz
  cd /etc/ssh
  tar -xzpf ~/ssh.tgz
  rm ~/ssh.tgz  # optional

Get the lab printcap (file has lp coded to 416)
 cd /etc; rm printcap
 wget http://tiger/export/ubuntu/printcap
(clones will change the printcap depending on the lab).

Edit /etc/fstab, remove the UUIDs and replace with /dev/sda1 and /dev/sda2
 /dev/sda1 /
 /dev/sda2 swap
Set options to defaults

Restrict access
 cd /etc/
 rm hosts.allow hosts.deny
 wget http://tiger.cecs.csulb.edu/export/
  hosts.allow
  hosts.deny
  cron.allow

Edit /etc/ntp.conf
  server 134.139.248.10
  server 134.139.248.8
service ntp restart
 (Alternatively: /etc/init.d/ntp restart)
 (if it does not restart, start it manually)

Build and install expired shells
 cd
 wget http://tiger/export/ubuntu/sus.c
 gcc sus.c -o /bin/sus
 gcc sus.c -o /bin/exp

 vi /etc/shells
  add /bin/exp and /bin/sus

/usr/local/bin
  cd
  Download source tarball wget http://tiger/export/ubuntu/local_bin_src.tgz
  tar -xzpf local_bin_src.tgz
  paper: gcc paper.c -o /usr/local/bin/paper
  (if it does not compile, use the -w option to do warnings instead of errors)
  retrieve: gcc retrieve-tar.c -o /usr/local/bin/retrieve
  shotgun (script): cp -p shotgun /usr/local/bin/ 
  diskspace (script): cp -p diskspace /usr/local/bin/ 
  spacecheck.sh (script): cp -p spacecheck.sh /usr/local/bin/
  --resetX (withheld until I figure out X)
  --show_users.sh (not done-withdrawn)
  --mpaper isn't necessary

Get the lilo.conf file:
  wget http://tiger/export/ubuntu/lilo.conf
Contents:
  LBA32
  boot=/dev/sda
  vga = 773
  image = /boot/vmlinuz
    root = /dev/sda1
    label=linux
    read-only

Link in the correct (latest) kernel. Make sure you've rebooted by now.
 cd /boot
 ln -s vmlinuz-<latest kernel>-generic vmlinuz

In /etc/X11/xinit/xinitrc edit, add following line above the Xsession line
 to enable the exit of X with ctrl-alt-bksp:
  setxkbmap -option terminate:ctrl_alt_bksp

In /etc/default/ edit:
  // console-setup: enable exit of X with ctrl-alt-bksp
  //  Set: XKBOPTIONS="lv3:ralt_switch,terminate:ctrl_alt_bksp"
  rcS:
   Set: DELAYLOGIN, VERBOSE and FSCKFIX to yes

/etc/gtk-2.0/
 gtkrc:
  To enable printing in X applications, build this file (mod 644) with the line:
   gtk-print-backends = "lpr,file"

Disable jetty: "rm /etc/rc*.d/S*jetty"

/etc/csh/login.d/sethostname
 create this file, contents (one line): setenv HOSTNAME `/bin/hostname`
/etc/csh/login.d/checkspace
 create this file, contents (one line): /usr/local/bin/spacecheck.sh

Edit: /etc/crontab
 Activate a once per night reboot to clear run away jobs
  0 3 * * * root /sbin/reboot

Allow users to run pmount for their usb drives
 chmod 4755 /usr/bin/pmount 
 chmod 4755 /usr/bin/pumount 

Disable the updating of motd
 unlink /etc/motd
 cp /var/run/motd /etc/motd
 vi /etc/motd (eliminate most lines)
 Add a note on startx and a note to log out when done. 

Disable the updating of resolv.conf
 rm /etc/resolv.conf  (get rid of the link)
 vi /etc/resolv.conf
  nameserver 134.139.249.20
  search cecs.csulb.edu


Add DrJava
 Download the latest drjava jar file (use startx and download via a browser)
  place it in /usr/local/lib
 Place the startup script drjava in /usr/local/bin. Contents of script:
  java -jar /usr/local/lib/drjava-<latest version from lib>.jar
  chown root:root drjava
  chmod 755 drjava

Add Apache Derby eclipse plugins
 Download latest derby plugins (should be 2 zips, may be one)
  unzip or untar the plugins. Should be org.* directories.
   org.apache.derby.core_10.8.2
   org.apache.derby.plugin.doc_1.1.3
   org.apache.derby.ui_1.1.3
 Place in /usr/lib/eclipse/plugins/
   
Remove admin capabilities from X window server:
 cd /etc/polkit-1/localauthority/50-local.d
 wget http://tiger/export/ubuntu/50-admin.pkla
  Contents of 50-admin.pkla:
      [disable suspend]
        Identity=unix-user:*
        Action=org.freedesktop.upower.suspend
        ResultAny=no
        ResultInactive=no
        ResultActive=no
      [disable hibernate]
        Identity=unix-user:*
        Action=org.freedesktop.upower.hibernate
        ResultAny=no
        ResultInactive=no
        ResultActive=no

Disable printing from eclipse
  in /etc/eclipse.ini add the line
   -Dorg.eclipse.swt.internal.gtk.disablePrinting

NEED

Turn off screen locks on gnome. Don't need for the lab
Fix it so it doesn't clear screen on logout (search and account issue).

Making the master image:
 netboot slackware 13
 If you have not done so during a previous install, fdisk /dev/sdb to have
  one Linux ext2 partition and mke2fs on that partition.
 mount /dev/sda1(2) /mnt
 mount /dev/sdb1 /cdrom
 rm /mnt/root/.ssh/known_hosts
 rm /mnt/etc/udev/rules.d/70*
  If you have done a previous install remove or move/backup the previous
  tarball:
 rm /cdrom/ubuntu.tgz
  Build the new tarball
 cd /mnt
 tar -czpf /cdrom/ubuntu.tgz .
  Copy the tarball to the image server
 reboot to hard drive
 login as root
 mount /dev/sdb1 /cdrom
 scp /cdrom/ubuntu.tgz root@tiger:/export/ubuntu/ubuntu.tgz

ssh Linux servers:

  These are 32-bit machines. Installed 32-bit bzImage and 32-bit initrd.img on
  the boot server (linux1).

  On a 32-bit box: booted 10.04 LTS 32-bit server CD
  Built/installed a 32-bit system (similar instructions as above).

  /usr/local/bin:
   built 32-bit versions of paper and retrieve, shotgun is a script so it had
    no problem

  lilo.conf: used "vga = normal" (because the KVM monitor won't support 773).
   Link in the correct kernel
     (the kernel name was different)


  master image:
   had to boot from CD (netboot in ECS 408 is 64-bit)
   called the master image ubuntu32.tgz

Cloning:

# Do the following steps once (unless you have a bug):
On the print server:
  copy in the kernel (bzImage), initrd (initrd.img) and boot message
   (message.txt) into /tftpboot/slackware.
   tgz of slackware is in ~vjd/ubuntu/bootstuff2.tgz, just cd to /tftpboot
   and undo the tar.
 cp slackware boot lines from tiger into /tftpboot/config/default
  (the three lines are in ~vjd/ubuntu/bootstuff1)
  (do not replace the file, add the 3 lines)

# Do the following steps once per loading a lab
On test163 (the Master Image building machine):
  Build the master image into /dev/sdb1 as ubuntu.tgz
   (see install.master)
  copy  the ubuntu.tgz file into /export on the printserver
   scp ubuntu.tgz root@psvr:/export/
On Tiger:
  build the initd.img with the updated ubuntu.install script on tiger,
   move it into /tftpboot/slackware/
   (see makingInitrd)
   Current initrd is in ~root/extractdir.
   Current initrd.img is in ~root/newInitrd
   Current script is ~root/extractdir/usr/lib/setup/ubuntu.install
On the print server (psvr416):
 Do once:
  copy /tftpboot/slackware/* from tiger into /tftpboot/slackware/*
   cd /tftpboot/slackware
   scp root@tiger:/tftpboot/slackware/initrd.img .
  create the exports directory for the image
   mkdir /export
  setup the nfs exports for the room the printserver is in
   vi /etc/exports
    add the line (psvr416)
     /export 134.139.247.192/255.255.255.192(ro,no_subtree_check)
      (psvr412 134.139.247.128, psvr414 134.139.247.64)
   start nfs
     chmod a+x /etc/rc.d/rc.nfsd
     /etc/rc.d/rc.nfsd start
   
 Do each time you clone:
  Copy the master image into the /export dirctory.
   go to the machine with the master image,
   use scp to copy the master image tarball to each of the printers
    (see install_master for detailed command)
  Modify dhcpd.conf to be dhcpd.conf.PXE and restart dhcpd
   cp -p /etc/dhcpd.conf.PXE /etc/dhcpd.conf
   killall -9 dhcpd
   /usr/sbin/dhcpd
  do the cloning
  After the installs disable the PXE boot and restart dhcpd
   cp -p /etc/dhcpd.conf.NOPXE /etc/dhcpd.conf
   killall -9 dhcpd
   /usr/sbin/dhcpd
-------
Clone (i.e. Install) do once per lab machine
boot: Slackware
fdisk /dev/sda
 about 80% sda1 linux(83), 20% sda2 swap(82)
install.YYY XXX  #where YYY is the room number and 
  XXX is the linux # of the machine
reboot

The following should be in the install.416 script:
The other scripts, install.414 install.412  will have different IP numbers
and names.

mkswap /dev/sda2
mke2fs /dev/sda1
mount /dev/sda1 /mnt
modprobe e100
ifconfig eth0 134.139.247.196 netmask 255.255.255.192
route add default gw 134.139.247.193 (may not be needed anymore)
mount -t nfs -o nolock 134.139.247.194:/export /cdrom
cd /mnt
tar -xzpf /cdrom/ubuntu.tgz
mount -o bind /dev /mnt/dev
# The next removes the warning about /proc/partitions not being found
mount -o bind /proc /mnt/proc
chroot /mnt lilo
echo  "ecs416lin$1.cecs.csulb.edu" > etc/hostname
----
ssh linux
 On the boot server (linux1, linux2):
  apt-get -y install atftpd ; edit ftpd out of inetd.conf
  apt-get -y install dhcp3-server ; add MAC addresses to dhcpd.conf
  /tftpboot
   Backed up 32-bit install directory as slackware32
   Created a new slackware directory with the initrd.img and kernel 
    from the ECS405 install of Slackware-13.37 because it handles the
    gigabit ethernet cards on the board.
   Start dhcpd and tftp by hand on the boot server (Ubuntu style):
    service dhcp3-server start
    atftpd --daemon /tftpboot
 On tiger:
  placed the install image in /exports as ubuntu.tgz
 On the ssh linux box we are installing
  BIOS
   SATA must be set to IDE, set boot to hard drive, network, F12 for net
  fdisk /sda
   sda1 linux (80-90% linux)
   sda2 swap (type 82)
  Warning: script is for intalling Slackware 13.37 in ECS 405, you must
    do this by hand (until we build a new initrd.img with a good script)
  mke2fs /dev/sda1
  mkswap /dev/sda2
  mount /dev/sda1 /mnt
  cd /mnt
  ifconfig eth0 134.139.249.xx netmask 255.255.255.192 (see front of box for xx)
  route add default gw 134.139.249.65
   mounted install image directory from tigere
  mount -t nfs -o nolock 134.139.249.167:/export /cdrom
  tar-1.13 -xzpf /cdrom/ubuntu.tgz
 edited /mnt/etc/hostname (name of the machine)
 edited /mnt/etc/lilo.conf
  change vga = 773 to vga = normal because that's what the monitor can do
 edit /mnt/etc/yp.conf
  ypserver 134.139.247.168  (psvr414)
 edit /mnt/etc/printcap  set psvr414 as lp
 edited /mnt/etc/network/interfaces (use static ip)
   iface eth0 inet static
    address 134.139.249.xx
    netmask 255.255.255.192
    gateway 134.139.249.65
  change /mnt/boot/vmlinuz to link to the 37 kernel (this time only)
  copy down e1000e.xx.tgz into /root, make, makeinstall
  installed boot sector
   mount -o bind /dev /mnt/dev
   # The next removes the warning about /proc/partitions not being found
   mount -o bind /proc /mnt/proc
   chroot /mnt lilo
   umount /mnt/dev
   umount /mnt/proc
   umount /mnt
   reboot
   cd e1001-1.9.5./src  (update the kernel network drivers)
   make ; make install
\end{verbatim} 
