\section{Student and Faculty Help}\label{sec:student_faculty_help}
\subsection{Introduction}
Regardless how well structured a lab is, user support is necessary for problems and potential confusion.  This support may come in the form of a tech support email, ticketing system, website with an FAQ, or some fix utilities.  CECS labs offers a website, a tech support email, some text files with instructions, and some Linux fix scripts.
\subsection{Support Website}
The first line of any support is a website.  This site should have at a minimum an FAQ with frequently asked questions.  It should be very easy to find on the site and be clearly visible.  This is the equivalent of a README file with a software distribution.  This FAQ should answer such things as getting started, creating an account, various problems with accounts, password reset instructions, or various issues users may commonly have.  It should also have in bold and/or red lettering certain things that students should know such as rules for not eating in the labs. 

The website is unfortunately rather pointless if no users know where it is (or that it even exists).  In academic environments, this can somewhat be remediated by having the faculty explain it to students or give a handout on basic lab usage.  Of course, many students will just ignore this.  A better solution (and one that works well in non-academic environments as well) is to just put a label on the front of every machine with the hostname, support website, and tech email.\footnote{Tech email can be replaced with ticket helpdesk url in environments that use one.}  This will hopefully lead users to the proper place rather than having them come and knock on the tech shop door every time they have a problem.  It is very important that the support website be featured very prominently and preferably have the text in bold.  

In CECS labs, we have a site similar to this that is the default homepage for every user. It has everything mentioned, but our problem is that it does not have everything updated due to changes being so frequent. We do the best we can but constant changes in our environment for each release make a completely accurate website infeasible. We are not the only ones with this problem though. We have yet to see the College of Engineering website completely updated with a list of all current faculty.  
\subsection{Support Email and Alternative}
Regardless how well put together a website is, it does not account for everything.  There has to be some method of contacting tech support to report problems and ask for troubleshooting help.  The simplest way to do this is setting up a contact email.  However, in environments with more than one technician, this contact email has to have a forwarding list.  This way, email sent will go to all techs involved.   Emails have the problem where techs may not always know what each other are doing and can potentially duplicate work.  This is not much of an issue with small tech shops, especially with each tech sticking to a specialty.  But in large organizations where IT support can exceed more than a few people, emails to support become very inefficient.  A small solution is to set up all email with a Reply-to field so that users always answer to the single tech email that forwards to all techs.  Not all email clients pay any attention to this though and some of the most common clients ignore it entirely.

Thus, a ticketing system may be a better solution.  Tickets are opened through some sort of custom interface, for example a support web application.  Users login to a ticket helpdesk and open tickets to support describing their problem.  They may select one or more categories of problems that their issue falls under.  From there, an IT support staff can login to the helpdesk as administrators and look for tickets to work on.  A good ticketing system also has in place mechanisms for leaving notes and attachments\footnote{Attachment capability is especially necessary for screen shots} on what was done and what the problem was.  More advanced features might include an interface for having managers login to the helpdesk and approve requests that require management approval.\footnote{Examples of actions needing management approval include production system updates, access management changes, and operating system configuration changes}  Lastly, the ticketing system is much easier to audit than long email trails and is practically a necessity in environments where strict compliance is needed.   

Ticket helpdesk systems usually do not work in small environments, especially academic.  The system is often too confusing for students to bother with.  They will usually just ignore it and continue on despite a potential problem.  Faculty will most likely never use the system and will simply email whatever technical support contact they can.  Ignoring faculty who do not follow the rules will not improve matters either.  When angry, faculty will simply go and complain to the department chair or to the dean and cause more trouble for technical support staff.  Thus, a ticket helpdesk is probably not feasible in our environment unless required as a campus wide regulation.  
\subsection{Fix Utilities}
Sometimes a lab load cannot handle certain configurations the students and faculty may have.  Various configuration files exist in their home directories and may override system defaults.  If the configurations are not correct, various issues may occur.  Sometimes these configurations may not even be intentional.  Home directories may be migrated from one distribution to another and configurations may not be compatible.  Thus, a set of fix scripts is necessary to reset configurations to a default.  These scripts may be located on all lab machines, but updating them would require ssh pushes to keep consistent even with some other repository software.  If network mounted home directories are used, it is wise to create one or more utility users (for example, "conf") whose files are left world readable and executable.  For example, students may have a faulty ".gnome2" configuration that could cause the gnome window manager to malfunction.  A script "resetX" could exist in conf's home directory "~conf/" and be world executable.  Students could then run ~conf/resetX from a machine that has conf's home directory mounted or automounted.  This script would delete a students copy of the gnome configuration and pull in a working configuration.  It could also have logic to determine if a student has a common problem in their configuration and correct accordingly without overwriting the entire configuration.   

There are a variety of other useful scripts that could be put in conf's home.  For example, you could create a script that would restore all configuration files from a tested skeleton account.  Sometimes, NFS files could become locked and students do not know how to unlock them.  A script to find and unlock the files would be very helpful.  Backup and restore scripts could be useful, especially if a nightly backup system is run.  Care should be taken as to where these scripts should reside, however.  If the script often changes it should be put in ~conf/ and shared over the network.  If the script is somewhat more permanent, it is often easier to just house the script locally.  The recommended location for these scripts is in "/usr/local/bin/" where they are made world executable.  Thus, with each master image revision they will be updated if needed.  
