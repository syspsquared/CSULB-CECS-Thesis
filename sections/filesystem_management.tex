\section{Filesystem Management} \label{sec:filesystem_management}
\subsection{Background}
Often in a lab environment, particularly an educational one, users will benefit from having persistent storage so that they can work day to day.   File storage has become less important in recent years than it was back before storage became commercially cheap.  Nowadays, flashdrives that can hold many gigabytes of data can be bought for less than 1 dollar per GB.  In addition, with the advent of cloud storage services, many companies are offering free storage online.  And as always, code repositories can help with revisioning.   That being said, it is still very convenient to have even a small amount of storage between computers in a set of labs.  

For network storage, central file servers are the most common way to go.  Some implementations use distributed storage across an entire lab's hard drives.  This method requires a number of security and redundancy considerations and should only be considered if the support staff is able to maintain it.  The central file servers in our environment run Suse Linux and export user home directory with Network File System (NFS).  
\subsection{NFS Exports}
There are 3 fileservers and they each have 7 hard drives apiece.  Each drive contains home directories for student and faculty to keep files persistently.  The drives do not have any special configuration and are simply formatted with a standard Linux filesystem.  Each drive is exported via NFS.  The exports are only available within CECS lab subnets.  Because of this, we can set the exports to run in insecure mode for compatibility reasons.  We also export with root\_squash enabled for all but one subnet.  This prevents an attacker who may have gained root privileges from damaging or stealing all of the files held.  All of this is configured in the "/etc/exports" configuration file on each fileserver.  The simplicity of the configuration has made this system run flawlessly for roughly a decade.  The only issues we have had were occasional hard drive replacements after several years of use.  All other issues concerning file shares have been client issues and the most common one involves clients locking files inappropriately.  
\subsection{Home Directory Changes}
Although root\_squash is useful for preventing possible disasters, no\_root\_squash is highly useful for fixing things affecting all students.  For example, when updating between the Linux distributions Fedora to Ubuntu, students still kept the same home directory with the same set of configuration files.  Files that functioned in Fedora could potentially break various features in Ubuntu (most often with window managers).  These files are often consistently the same with all students and consistently cause the same issues.  The hard way to deal with this is to give students instructions (perhaps on the support FAQ) on how to fix it.  This will inevitably lead to many emails to tech support anyway.

Instead, it is useful to have a server or small subnet that has permission to mount student home directories on fileservers with no\_root\_squash.  That way, root can then make edits all student homes at once via a script.  This script would need to have a lot of checks in place to ensure that it is fixing the problem and only fixing the problem.  It would also need highly verbose logging to figure out if anything went wrong.  In its simplest form, this script would just loop through a list of affected student home directories, cd to each of them, and run whatever command necessary to fix the issue.  

\subsection{Disk Quotas}
Total space for network storage is highly limited.  Although hard drives have come down in price considerably, with almost 1500 accounts at any given time, a disk quota is a necessity.  We set a quota conservatively at 512 MB for each student and 1 GB for each faculty member.  This may seem small considering the price per GB of hard drives, but the students are mostly writing code in text files.  They are not doing heavy picture or video editing so that amount of space has been sufficient so far.  Recently, we have noticed that web pages are are becoming larger and browsers are caching more as a result.  Mozilla Firefox in particular now sets its cache limit at 1 GB.  Thus, students are filling up their quotas quickly just by visiting websites.  We have modified the cache to return to its old 50 MB limit, but we realize that our fileservers will need to be upgraded in the next couple of years to keep up with increasingly larger files.  Typically, when the fileservers collectively become more than half full, we begin to worry about space.  

\subsection{Automount}
We use the automount tool on our workstations and servers to mount file shares from our core fileservers.  The automounter has several advantages over using the classic method of relying on the fstab (filesystem table).  The fstab mounts all the time starting from when the /etc/fstab is read on system startup.  This generates unnecessary network traffic when a mount is not being used.  In addition, if a mount fails, it will not remount until either a sysadmin remounts it manually or the system restarts.  The automounter addresses both these issues.  It runs as a daemon in the background and waits for NFS requests by other processes.  It handles the given request and only mounts the file share that it needs to.  In addition, if the mount fails for some reason, it actively attempts to mount again in case there was a random network issue.  The /etc/auto.master configuration file governs the automounter's behavior.

\subsection{Backups}
We have two stages of backups.  One backup is done nightly to a set of fileserver clones.  These clones have the same hard drive and export configurations as the main fileservers.  Thus, if a fileserver goes down, we need only change the hostname of the backup to get it up and running again (this has yet to happen).  The other backup is done every Saturday to a set of hard drive arrays attached via a scsi cable.  Both backups are done with the tar command to keep things simple.  Because this system is so uncomplicated, it is simple to maintain and troubleshoot.  Our fileservers are easily the most stable point of our environment.  They do require periodic checks by a staff member though.  Unfortunately, while they have been fine for years, the space they provide is becoming insufficient.  We will likely not be able to upgrade them any further, nor will we be given the funds to replace them.  Likely, once they are decommissioned, we will have no more file storage unless someone donates to us.  

\subsection{Student Media}
Students can also use their own USB drives to store their files if the space provided by us is insufficient.  The problem with student drives is a potential security risk when they are plugged in.  Windows has the tendency to look onto the drives for drivers and may run malicious code that could infect the workstations.  This has become less an issue now that Microsoft has finally addressed it.  On Linux, we simply do not allow any files on the drives to execute.  We also prevent the drives from being mounted with administrative privileges.  Students can use the "pmount" and "pumount" commands to mount the drives as normal users without privileges.  This also forces them to mount manually and prevents exploits of an automatic USB mounting system.  The only issue we have had so far is lack of knowledge of the system by users.

\subsection{Malware}
We occasionally receive reports of malware on our lab workstations.  We have powerful anti-virus software but this does not prevent everything a student may run.  Many things are just programs in memory that disappear as soon as a student logs off.  Others may just reside in temporary files that are easily cleared.  Regardless, we run an environment that teaches programming.  We cannot prevent students from running malicious code intentionally or unintentionally.  We do, however, have the ability to track which students are repeat offenders.  We have the authority to temporarily reduce the ability to do further damage by locking accounts.  Further action can be taken by involving the faculty, the department administrators, or even upper campus administrators in the event of extreme wrongdoing. 
