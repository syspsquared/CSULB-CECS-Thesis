\chapter{Conclusion} \label{ch:conclusion}
\section{Summary} \label{sec:summary}
To conclude, the efficient management of the CECS labs is made possible by automation and careful crafting of our infrastructure, which allows us to problem solve a wide range of problems such as tedious deployments and constant updates.  

%For documentation, we use a shared user account and store instructions there.  We maintain a support website and shared technical support email.  Our infrastructure consists of fileservers and printservers.  Our install process involves creating a master lab image and cloning it to many workstations over the network.  Maintenance during the semester is done via Secure Shell pushes and periodic automated cleaning.  Lastly, we maintain a mixed environment with many different platforms across regular labs and specialty labs.  Maintaining all of this can be challenging, but we make it easier through automation. 
The uniqueness of a university setting has allowed us to be creative in our infrastructure implementation.  Our permissions limit students and faculty sufficiently enough to provice minimal security risks while still allowing them to do work in a technical field.  We handled the paper accounting problem by creating the Paperd program and connecting printers to print servers.  All of our main labs are uniformly based on clones of a master image which saves considerable time and effort customizing each one.  Although we have many special labs, the same master image and clone techniques are used in each one to facilitate efficient management with faculty assistance.  Student data is protected and available with two different backups.  Down time is very rare because our file servers are well maintained and reliable.  And although cross-platform interoperability has always had a significant time cost, we, the technicians, have so far managed to find solutions by combining our experience.

It is important to note that our solutions are not the only solutions.  Other universities may be run very differently.  Some universities build their infrastructure with a Microsoft Windows based core instead of Linux.  Others may have chosen to standardize completely on one platform to avoid the various issues we have faced.  Though the technical details vary by university IT environments, the fundamental concept of automation where possible, in addition to building a solid core, is of great importance to all IT environments. 
