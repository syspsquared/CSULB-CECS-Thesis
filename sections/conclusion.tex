\chapter{Conclusion} \label{ch:conclusion}
\section{Summary} \label{sec:summary}
In conclusion, managing the CECS labs is a difficult task made easier by using efficient management.  We have problems to resolve on all fronts from tedious deployments to constant updates.  This is all made easier through automation and careful crafting of our infrastructure.  

%For documentation, we use a shared user account and store instructions there.  We maintain a support website and shared technical support email.  Our infrastructure consists of fileservers and printservers.  Our install process involves creating a master lab image and cloning it to many workstations over the network.  Maintenance during the semester is done via Secure Shell pushes and periodic automated cleaning.  Lastly, we maintain a mixed environment with many different platforms across regular labs and specialty labs.  Maintaining all of this can be challenging, but we make it easier through automation. 

As far as common university problems, we have handled most of them.  Our permissions limit students and faculty sufficiently while still allowing them to do work in a technical field.  We handled the paper accounting problem by creating the paperd program and connecting printers to print servers.  All of our main labs are uniformly based on clones of a master image which saves considerable time and effort customizing each one.  Although we have many special labs, the same master image and clone techniques are used in each one to facilitate efficient management and we have had faculty to support to assist us.  Student data is highly available as we have 2 different backup servers back up every night.  Down time is very rare because our file servers are well maintained and very reliable.  And although crossplatform interoperability has always had a significant time cost, we have so far managed to find solutions by combining our experience.

It is important to note that our solutions are not the only solutions.  Other universities may be run very differently.  Some universities build their infrastructure with a Microsoft Windows based core instead of Linux.  Others may have chosen to standardize completely on one platform to avoid the various issues we have faced.  The technical details given in this paper's appendix section have probably been written very differently at other universities.  However, the fundamental concept of automation where possible in addition to building a solid core is of great importance not just in universities but in other IT environments. 
