% READ ME, PLEASE !!!!
% This document is based on the LaTeX format of Mark T. McLaughlin's thesis
% which was made with WinEdt 5.4, October 2005.  This top section is his
% comments.
%
% 1) IF YOU CAN'T FORMAT SOME PART OF YOUR THESIS ACCORDING TO THE CSULB
%    THESIS GUIDELINES:   DON'T WORRY!!
%    Just rewrite the crap in MSWord. The command, \usepackage{times}
%    puts everything in the same font as the times roman used by MSWord
%
% 2) There are a few SMALL PROBLEMS with the
%    table of contents, list of figures, and list of tables. The problems are
%    a) appendix subhead titles are not supposed to be included in the table of
%    contents, b) figures in the appendices are not supposed to be included
%    in the list of figures, and c) tables in the appendices are not supposed to be
%    included in the list of tables. -EASY SOLUTION: Make up a phony appendix, say
%    phonyAppendixA, which is blank, but the same number of pages as your real
%    appendix A. (Similarly, do the same with appendix B, appendix C, etc...)
%    Then at the end of this file, main.tex, comment out your real
%    appendix A, and uncomment this phony appendix when you LaTex your document
%    for the purposes of generating a table of contents, list of figures, and
%    list of tables. Then when you want to LaTex the rest of your real document,
%    comment out your phony appendices, and uncomment your real appendices.
%
% Everything else passed the thesis office inspection!!! -Bibliography, Signature page,
% Equation numbering, figures, tables, margins, all that junk......
% Thanks Yamato Matsuoka for creating such a nice thesis template!
%
% MORE HELPFUL STUFF:
% TIPS FOR MAKING FIGURES AND TABLES
%   AUTOCAD:
%       In Windows: autocad -> HP color laserjet postscript -> print to filename.ps
%       In Linux: pstoedit -f xfig filename.ps filename.fig
%                 xfig filename.fig
%        in xfig: export -> filename.eps
%        put filename.eps in LaTex document
%
%   EXCEL: excel -> print to postscript (in printer settings click on eps postscript)
%       In Linux: pstoedit -f xfig filename.ps filename.fig
%        in xfig: export -> filename.eps
%        put filename.eps in LaTex document
%

%% =====================================================================
%% Thesis --- Main LaTeX File  ver 0.1
%%                 Date: 9/1/2012 started
%% =====================================================================

\documentclass[12pt]{csulb_thesis}
%\documentclass[12pt]{report}  %% regularly use this
\usepackage{amsmath}   %% mainly to use align environment
\usepackage{amssymb}   %% to use \boldsymbol
\usepackage{times}
\usepackage{alltt}
%\usepackage{txfonts}
%%% txfonts converts default CM fonts to the followings
%%  Computer Modern Roman      -->  Times
%%  Computer Modern San Serif  -->  Helvetica
%%  Computer Modern Typewriter -->  Corresponding Typewriter Font


%\usepackage[dvipdfm]{color}  %% if you like

%\usepackage[dvips]{graphicx}
% \usepackage[dvipdfm]{graphicx}
\usepackage{graphicx}
\usepackage{hyperref}

%% For printing use   [Option `hyperref.sty' is needed]
\hypersetup{colorlinks=false}

% -------------
% For pdfLaTeX
%\usepackage[pdftex]{graphicx,color}
%\usepackage[pdftex,colorlinks]{hyperref}
% -------------

%%
%\allowdisplaybreaks[1]
%%% \allowdisplaybreaks[n] ... allow pagebreak within eqnarray/align environment n = 1~4

\nonstopmode  % Keep compiling even when errors occur


%% ----------------------------------
%%            Newcommands          %%
% \newcommand{\vect}[1]{\boldsymbol{#1}}  % vector (bold symbol)
% \newcommand{\uvec}[1]{\hat{\vect{#1}}}  % unit vector (bold & hat)
% \newcommand{\mytxt}[1]{\texttt{#1}}
% \newcommand{\mybold}[1]{\textbf{#1}}
% \newcommand{\prof}[1]{Dr$.$ {#1}}
% \newcommand{\myA}[1]{A$({#1})$} %A(t), the acoustic signal from the microphone
% \newcommand{\myB}[1]{B$({#1})$} %B(t), a signal N times the freq of A(t)
\newcommand{\tab}{\hspace*{1.5em}}

%% ----------------------------------


%% ----------------------------------
%%          Author's Info          %%
\title{Efficiently Managing the Computer Engineering and Computer Science Labs}
\author{Nathan Pickrell}
\gradmonth{December }  % month & year of the graduation
\gradyear{2012}  % month & year of the graduation
\department{Department of Computer Engineering and Computer Science} % department
\degree{Master of Science} % CSULB degree
\bachelor{B.S.}  % bachelor degree (abbreviated)
\bacheloryear{2010}  % The year bachelor degree was given
\undergraduniv{California State University, Long Beach} % university where the degree was earned
%% ----------------------------------



%% ---------------------------------%%
%%       Thesis Committee           %%
\firstmember{Dennis Volper, Ph.D. (Chair)} %%% Don't forget to put (Chair) at last
\deptfirst{Computer Science}

\secondmember{Kenneth James, Ph.D.} 
\deptsecond{Computer Science}

\thirdmember{Tracy Maples, Ph.D.} 
\deptthird{Computer Science}

\deanname{Kenneth James, Ph.D.} 
\deandept{Department Chair, Computer Science}
%% ----------------------------------


%%------------- Begin Document -------------------%%
%%                                                %%
\begin{document}
\abstract{\thispagestyle{empty}

University lab environments are handled quite differently from other IT environments.  While they have the common issues of scalability and crossplatform interoperability, they must also handle student permissions, student files, special education labs, and handling of documentation.  The emphasis is on uniformity across lab machines for a uniform course curriculum. 

This thesis describes a case study on how a specific set of Computer Science labs is maintained.  It describes how the documentation is maintained, how the lab infrastructure is set up, how the technicians managing the lab build master lab images, how all of the workstations in the lab are cloned, and how some maintenance is handled.  Additionally, this paper also describes some of the specialty labs provided for courses with specialized topics. 
}
\maketitle

%\pagenumbering{arabic} %%%%%%% Comment Out if the abstract is one page long %%%%%%%%%%
%\pagestyle{empty}     %%%%%%% Uncomment   if the abstract is one page long %%%%%%%%%%


%%%---- Insert them if \signaturepage is not used ----
%\pagenumbering{roman}
%\setcounter{page}{2}   %%% Adjustment of Page numbering
%%%----

\signaturepage
% \include{acknowledgement}
\tableofcontents 
%\listoffigures
%\listoftables

%%%%%%%%%%%%%%%%%%%%%%%%%%%%%%%%%%%%%%%%%%%%%%%%%%%%
%%    This is manually inserted TOC entry!        %%
\addtocontents{toc}{\protect\contentsline{chapterhead}{\chaptername}{}{}}
%% the last {} may be removed if `hyperref.sty' is NOT used.
%%%%%%%%%%%%%%%%%%%%%%%%%%%%%%%%%%%%%%%%%%%%%%%%%%%%

%%%% Similarly, the following should be inserted somewhere in the document
%%%%%%%%%%%%%% TOC entry ! %%%%%%%%%%%%%%%%%%%%%
%%\addtocontents{toc}{\protect\contentsline{chapterhead}{\chaptername}{Page}{}}
%%%%%%%%%%%%%%%%%%%%%%%%%%%%%%%%%%%%%%%%%%%%%%%%

%%%% Or, insert the following if page breaking occurs in appendix entries
%%%%%%%%%%%%%% TOC entry ! %%%%%%%%%%%%%%%%%%%%%
%%\addtocontents{toc}{\protect\contentsline{chapterhead}{\null}{Page}{}}
%%%%%%%%%%%%%%%%%%%%%%%%%%%%%%%%%%%%%%%%%%%%%%%%


\clearpage \pagenumbering{arabic}
%\chapter{Introduction}\label{ch:introduction}
\section{Optimization}

Improvement means finding better solutions to problems.  These problems may be to win a game \cite{re:mri}\cite{re:lightbot}.  Any system for which the quality of a  function.\footnote{Some objective functions make the goal minimization rather than maximization, but the process is effectively the same.}  Some situations require  method is the Genetic Algorithm, the subject of this work.  

The simplest optimization method is to examine every solution in the search space, to compute, time is often given in iterations rather than seconds.

\subsection{Mastermind}
Consider a simple code-breaking game to be referred to henceforth as Mastermind \cite{re:mastermind}.  The objective of the player is to guess a multi-digit sort of difficult problem for which approximate optimization is often used \cite{re:knuth}\cite{re:stuckman}.

The objective function is defined in equation 1.  The motivation behind it will be explained in section~\ref{sec:evolutionaryheuristics}.
	\begin{equation}
		\label{eq:mastermindfitness}
		f(code) = correct + log(partially correct)
	\end{equation}

\subsection{Random Search}
Another obvious optimization method would be to repeatedly guess while keeping probability of having found the optimum in $k$ trials given a search space of size $n$ given by the following equation and shown for various values of $n$ in Figure~\ref{fig:randomsearchprobability}.  
 
	\begin{equation}
		p(k|n) = 1 - \left(1 - \frac{1}{n}\right)^k
	\end{equation}
	
Given the probability of guessing the solution in $n$ trials is approximately 63\%the objective function for random search is shown in Figure~\ref{fig:randomsearchquality}, normalized to 1 for the maximum possible value.  This shows that, while 

	
	\begin{figure}
		\begin{center}
			\scalebox{0.7}{\includegraphics{RandomSearchProbability}} 
		\end{center}
	\caption{Probability of having guessed optimal solution in $k$ trials given search size $n$, for $n$=100, 1000, and 10000.}
	\label{fig:randomsearchprobability}
	\end{figure}

	\begin{figure}
		\begin{center}
			\scalebox{0.80}{\includegraphics{RandomSearchQuality}} 
		\end{center}
	\caption{Normalized objective function quality for random search of Mastermind for $k$ iterations given search size $n$=100, 1000, 10000.  Averaged over 1000 trials.}
	\label{fig:randomsearchquality}
	\end{figure}

Most evolutionary algorithms include notions of a population, a fitness function, a selection method, and a breeding method \cite{re:evolutionaryalgorithms}.  The population contains potential solutions to an optimization problem, the quality of which is judged by the fitness function.  A generic evolutionary algorithm is shown in Table~\ref{tab:genericevultionaryalgorithm}.

	\begin{table}
		\caption{Generic Evolutionary Algorithm} \hline
		\label{tab:genericevultionaryalgorithm}
		% \begin{tabular}{0.75\textwidth}{l}
		\begin{tabular}{l} \\
		population \(\gets\) random initial values \\ \\
		WHILE fitness of best population member \(\leq\) required fitness \\ \\
			\tab new\_population \(\gets\) empty list\\ \\
			\tab REPEAT \\ \\
				\tab \tab Select parents from the population proportionally to fitness \\ \\
				\tab \tab Breed parents to create children. \\ \\
				\tab \tab Add children to new\_population. \\ \\
			\tab UNTIL size(new\_population) == size(population) \\ \\
			\tab population \(\gets\) new\_population \\
		\end{tabular}
		\hline
	\end{table}
	

A more specific version of evolution is genetic evolution, which treats the gene as the main element of evolution rather than the organism \cite{re:selfishgene}.  of genes, and the breeding methods should be akin to genetic crossover and mutation.  

\subsection{Heuristics} \label{sec:evolutionaryheuristics}
A major way in which algorithms, including optimization methods, can be improved is by designing it to take advantage of knowledge and assumptions made by its 


\chapter{Introduction} \label{ch:introduction}
\section{The Problem} \label{sec:the_problem}
%# Describe the problem that your case study solved. Not all this faculty garbage.  
University educational lab environments have a very different set of problems to manage than other Information Technology (IT) environments such as those in commercial, government, or corporate settings.  Most work IT environments care about such things as a stable and up to date email system, a defense system to thwart attacks on the outside, theft of trade secrets or misuse of customer information from employees, and creating a permission hierarchy to limit which resources employees can access.  Universities do not usually consider these as important, but they do have other concerns not present elsewhere.  Whereas most IT infrastructures handle groups of employees in various settings, university users are mostly students.  Faculty are also users but have different needs than other employees due to their need to teach courses using a computer lab environment.  

%-Limiting what students can do while still enabling them to do coursework. 
%-Limitations to faculty. 
Whereas employees may have varying levels of permissions, universities care primarily about two groups:  students and faculty.  Limitations on faculty are similar to other employees but faculty must be given enough permissions in order to teach a class and optionally administrate a group of students through a test or an interactive activity.  Full administrative rights on lab workstations are often requested but can be problematic to give to faculty who share a lab.  Thus a limited amount of rights should be given to a faculty group as needed.  For students, very heavy restrictions must be placed to ensure both the health of the workstations and that the students are limited to activities allowed for coursework.  It is important to realize that university environments do not necessarily have to be as restrictive as a lower educational institute.  For example, elementary schools typically have heavy content filters for web access that a university does not usually need.

%-printing and how students break things more because they aren't really liable.
Printing is a common feature managed in IT environments.  Universities manage printing by contending with printer compatibility and heavy printer use by students.  Students will often use much more paper than employees in their courses.  The paper and ink cost can become exceedingly expensive for a university budget, so a system is often put into place to either limit the pages a student can print or allow them to buy paper as needed.  This paper accounting system may require special printers or separate print servers.  

%-special labs that don't fit with main labs.
A university typically prefers uniform labs as they are easier to maintain.  However, many labs are for special purposes.  For example, engineering and science students may have special test equipment connected to the workstations that needs separate skill sets from technicians.  Some digital design labs may have special licensed software that can only be installed into one dedicated lab for a subset of students.  Teaching IT courses will itself require special labs on isolated networks where students are granted elevated privileges.  Each of these labs must be handled separately and with the proper skill.  However, it can become very costly to manage too many of them. 

%-Handling of student files and media. 
Student data rarely contains anything requiring a high level of confidentiality, such as the case of bank customer data in bank IT environments.  Most of the time, students simply complete assignments in courses.  However, while outside entities may care little for this information, other students can potentially cheat by stealing it.  Care must be taken to ensure internal security of student documents.  The availability of the documents is also very important as downtime during courses can cause delays and potentially shorten a required curriculum.  Nightmare scenarios also appear where students lose access to assignments when they are due.  Care must be taken to ensure availability of any system that students rely on for coursework. 

%-automation of deployment. how to make all machines clones without going to every machine or relying on contract. 
Despite differences, university lab environments and other IT environments share the two largest overall problems, scalability and cross-platform interoperability..  IT staff is budget limited, therefore solutions to manage the lab environments must scale not only to an arbitrary environment size but also stay within the limits of a strictly set budget.  A single technician may be able to build and configure a handful of machines in a short amount of time, but university labs rarely consist of only a handful of machines.  Often, classrooms hold dozens of students that must be using each machine separately.  In large education environments, especially those in technical fields or requiring technical tools, the number of machines to manage can be in the hundreds.  To build and maintain all of them one by one quickly becomes impossible without either increasing the staff to a very expensive size or by managing them with a process that scales to large sizes.  In most cases, this is handled by a system of automated deployment of a lab master image to a large group of clones. 

%-cross-platform interoperability. 
The other main issue facing lab environments is cross-platform interoperability.  Different companies build their products very differently.  For example, account and login information is not uniform across Unix and Microsoft Windows platforms.  This means that users must either have two different sets of accounts and passwords, or some solution must be implemented to make these platforms share that information.   Solutions to interoperability vary.  In many cases, third-party tools can be purchased to handle interoperability at a cost.  Many platforms contain built-in tools that can be used although configuration can take a significant amount of time on the part of the technicians.  In some cases, it is impossible for certain features of a platform to correctly interact with other platforms, so workarounds must be found or the features cannot be used. 


%Doesn't help. Mention very briefly in case study but note that this is a management issue not a technical one.
%\subsection{Limited Faculty Testing}
%One of the main difficulties we face is a lack of faculty feedback during the architectural phase of lab builds.  Requests for changes pre-build come sparingly from a small handful of faculty.  Most faculty will wait until a week before the semester starts before making change or update requests.  Some of them will even wait until the semester starts.

%We understand the various reasons why this may occur.  Most faculty are not paid during the summer months and are, therefore, under no obligation to work.  However, those faculty wishing to avoid this conflict at the beginning of the semester, need only submit requests at the end of the previous semester.  Part-time faculty members face the dilemma of receiving their schedules a few days before or the week of start of term.  Tenured faculty, however, have no set penalties for postponing their lesson planning.  

%Furthermore, this is a problem for us technicians because it limits the amount of testing we can perform on our lab images.  Oftentimes, we lack the knowledge base to test software needed for classes.  We research as much as we can and hope that our constricted knowledge is enough.  While we have a relatively fast lab cloning procedure to recreate the labs in the event of an emergency when the semester starts, in many cases, we must ignore requests from faculty members until the next iteration of the lab images are created for the following semester. 

%doesn't help
%\subsection{Staff and Faculty Turnover}
%In addition to limited faculty testing, we also face a high turnover of faculty.  Part-time lecturers do not stay if their student generated reviews are poor.  Student reviews not only reflect student approval, but also faculty endorsement.  The fast  turnaround, therefore often leaves us technicians with an incomplete faculty list.  Incomplete faculty lists often poses a problem.  We often run into problems where part time staff will not have user accounts to log into our workstations.  Without official verification details, we cannot grant them access to our accounts.  This will cause delays in courses that require lab components. 

%Move to intro to case study
%\subsection{Mixed Environment}
%The fourth floor of the Engineering and Computer Science (ECS) building is dedicated to our labs.  We have 3 Linux labs of 30 workstations each, 4 Microsoft Windows labs of 30 workstations each, an Apple iMac lab of 25 workstations, and several specialty labs.  Operating systems on these workstations are very different from each another.  This requires experience in many areas and maintenance is virtually impossible for a single technician to accomplish alone.  The servers that support the labs are also running different operating systems and require separate levels of experience as well. 

%handled above
%\subsection{Frequent Software Updates}
%There are numerous pieces of software to install and maintain on our workstations.  We create master lab images for each platform and deploy them through a cloning procedure.  Maintenance of these master images is time consuming as there are frequent software updates as well as operating system updates.  Updates while courses are being taught are risky because they have the potential to break some software feature that instructors need to teach. 

%reword as a common issue in IT industry but is often unaddressed.
%\subsection{Limited Staff}
%We have a limited amount of technicians available to us.  This is not uncommon in academia or in industry.  Information Technology (IT) departments are often kept lean because they are seen as an expense.  This results in limits to how much work can be done and how many technicians are available to support a number of workstations.  We improve this situation through automation, but there is only so much we can automate.  Often, we have to postpone needed updates and feature requests until we have enough resources.

%introduce case study
%\vfill\eject

This thesis describes a case study in which I the author helped to manage several university lab environments as a technician.  This environment is specifically for purposes of teaching Computer Engineering and Computer Science.  It has most of the IT issues described in addition to the problem of more periodic updates for an ever changing course curriculum.  The case study describes how I, as part of the technical staff responsible for lab building and maintenance, handled the various issues.

\chapter{The Documentation Solution} \label{ch:documentation_solution}
\section{Lab Documentation}\label{sec:documentation_solution}
\subsection{Overview}
With any lab management, documentation and procedures need to exist.  For example, if master lab images exist, full sets of instructions need to exist to describe how to rebuild the master from scratch at any point during the process.  That said, a secure place needs to exist to hold this documentation.  At a minimum, it needs to be in a directory somewhere that is periodically backed up.  More advanced documentation solutions may use online wikis or even code repositories for revisions along the way.  In the CECS lab environment, we have a good backup system in place for user accounts, so we created a user called ``conf" to store documentation in.  

\subsection{Special Documentation User}
The conf user was created in the same way as a standard user account.  The disk quota is unlimited, however, that is not a problem because the account only stores text documentation.  In our case, conf is used to hold a student’s skeleton directory, therefore, a quota is not applied.  Since conf is a network user account, it is available from all machines that have Network Information Service (NIS) and Network File System (NFS) mounts set up.  

Accessing the account can be done in several ways.  If an administrator can become root on a machine where conf is accessible from, they can ”su - conf” on the machine to become conf.  The rlogin command can be used assuming a user is in conf’s .rhosts file.  The modern equivalent of rlogin, Secure Shell (ssh), can be set up by adding an authorized administrator’s public key to conf’s ”authorized keys” file.  

In conf's home directory, we created a folder called ``admin" to hold administrative documentation.  Subfolders for the various lab builds exist.  We have folders for Linux labs, Mac labs, a few notes on the network, patches, the print servers, and the web servers.  Other directories are created as needed.  These directories should contain such things as system build notes, system patching notes,\footnote{Notes will include both distribution patches and manual fixes} supporting documentation,\footnote{These include things like user manuals and web links} or just general system notes.  Storage of the actual system patches can become quite large for a network filesystem.  It is usually much more efficient to have a dedicated file server to store large patches and distribution images.  In our implementation, all of these folders are set to world readable.  We do not store any sort of password or key information and do not have a problem with students being able to read how our labs are set up and managed.  Security through obscurity is not really security at all.  

\subsection{Alternatives}

Alternatively, the information can be stored on a wiki.  Setting up a wiki requires a webserver and usually needs php and mysql support.  The setup of the wiki can become somewhat convoluted depending on the configuration.  The advantage of the wiki is that it can be accessed by any web browser.  It may even be stored on the same server used for the support website.  It is probably not a good idea to open this webserver to the Internet since it has many more security vulnerabilities than flat files.  In addition, because so many components are involved, backing up the wiki can become very cumbersome.  To date, we have not been able to correctly back one up in CECS labs without backing up the entire webserver.  

Another alternative is to just use a code repository such as Subversion or Git.  These services require complex configurations to set up properly.  However, the advantage of using these services is the revisioning and editing ease.  These services can even be integrated into a web server to get the advantages of both.  For our purposes though, we found that simple text files work adequately so long as they are periodically backed up.  

\section{Student and Faculty Help}\label{sec:student_faculty_help}
\subsection{Introduction}
Regardless how well structured a lab is, user support is necessary for problems and potential confusion.  This support may come in the form of a tech support email, ticketing system, website with an FAQ, or some fix utilities.  CECS labs offer a website, a tech support email, some text files with instructions, and some Linux fix scripts.
\subsection{Support Website}
The first line of support is a website.  The site should have at a minimum a frequently asked questions (FAQ) section.  FAQ pages should be very easy to find and be clearly visible.  This is the equivalent of a README file with a software distribution.  This FAQ section should answer such things as getting started, creating an account, troubleshooting accounts, resetting passwords, or answering various issues users may commonly have.  It should also have bold and/or red lettering to highlight important procedures and rules students should follow, e.g. no food or drink in the labs. 

The website is unfortunately rather pointless if users do not know where it is (or that it even exists).  In academic environments, this can somewhat be remedied by having faculty members explain to students where to go in case of a problem, or provide a handout on basic lab usage.  Of course, many students will just ignore this.  A better solution\footnote{This solution works well in non-academic environments as well because it is easier in IT department to have systems labeled with hostnames.} is to just put a label on the front of every machine with the hostname, support website, and tech email.\footnote{Tech email can be replaced with ticket helpdesk url in environments that use one.}  This will hopefully lead users to the proper place rather than coming to the technicians.  It is very important that the support website be featured prominently with clearly legible text. 

In the CECS labs, we have a support site that is the default homepage for every user.  It has everything mentioned above, however, our problem is that it does not have the latest up to the minute updates.  We do the best we can but constant changes in our environment for each release make a completely accurate website unfeasible. We strive to be accurate within a few days of the changes.

\subsection{Support Email and Alternative}

Regardless how well put together a website is, it does not account for everything.  There has to be some method of contacting tech support to report problems and ask for troubleshooting help.  The simplest way to do this is by setting up a contact email.  However, in environments with more than one technician, this contact email has to have a forwarding list.  This way, email sent will go to all techs involved.  Emails lack the ability to inform all recipients if an email has been read, or in this case, if the problem addressed in the email has been resolved.  This may cause technicians to duplicate work.  This is not much of an issue with small tech shops, especially with each tech sticking to a specialty.  But in large organizations where IT support can exceed more than a few people, support emails can be inefficient.  A solution is to set up an email with a Reply-to field so that users always answer to the single tech email that forwards to all techs.  Not all email clients properly implement this feature and some of the most common clients ignore it entirely.

Thus, a ticketing system can be a better solution.  Tickets are opened through a custom interface, e.g. a support web application.  Users login to a ticket helpdesk and open tickets describing their problem.  They may select one or more categories of problems that their issue falls under.  From there, an IT support staff can login to the helpdesk as administrators and look for tickets to work on.  A good ticketing system also has in place mechanisms for leaving notes and attachments\footnote{Attachment capability is especially necessary for screen shots.} on how the problem was resolved and what the problem was.  More advanced features might include an interface for managers to login to the helpdesk and approve requests that require management approval.\footnote{Examples of actions needing management approval include production system updates, access management changes, and operating system configuration changes.}  Lastly, the ticketing system is easier to audit compared to long email trails and is practically a necessity in environments where strict compliance is needed.   

Ticket helpdesk systems usually do not work in small environments, especially academic settings because the system is often too confusing for students to bother with.  They will usually just ignore it and continue on despite a potential problem.  Faculty will most likely never use the system and will bypass the system by emailing the first technical support contact they can.  Thus, a ticket helpdesk is probably not feasible in our environment unless required as a campus wide regulation.  

\subsection{Fix Utilities}

Sometimes a lab load cannot handle certain configurations the students and faculty may impose locally.  Various configuration files may exist in their home directories and may override system defaults.  If the configurations are not set up correctly, various issues may occur.  Sometimes these configurations may not even be intentional.  Home directories may be migrated from one distribution to another and configurations may not be compatible.  Thus, a set of fix scripts is necessary to reset configurations to a default structure.  These scripts may be located on all lab machines, but updating them would require Secure Shell (SSH) pushes to keep machines consistent even with some other repository software.  If network mounted home directories are used, it is wise to create one or more utility users (for example, ``conf") whose files are left world readable and executable.  For example, students may have a faulty ``.gnome2" configuration that could cause the gnome window manager to malfunction.  A script called ``resetX" exists in conf's home directory ``~conf/" and is world executable.  Students can run ~conf/resetX from a machine that has conf's home directory mounted or automounted.  This script will delete a student's copy of the gnome configuration and pull in a working configuration.  The script could also have logic to determine if a student has a common problem in their configuration and correct the problem without overwriting the entire configuration.  A copy of this script can be found in Appendix~\ref{ap:lab_maintenance}. 

There are a variety of other useful scripts that could be put in conf's home.  For example, you could create a script that would restore all configuration files from a tested skeleton account.  Sometimes, NFS files become locked and students do not know how to unlock them.  A script to find and unlock the files would be very helpful.  Backup and restore scripts could be useful, especially if a nightly backup system is run.  Care should be taken as to where these scripts should reside, however.  If the script often changes it should be put in ~conf/ and shared over the network.  If the script is somewhat more permanent, it is often easier to just house the script locally.  The recommended location for these scripts is in ``/usr/local/bin/" where they are made world executable.  Thus, with each master image revision they will be updated if needed.  


\chapter{Lab Infrastructure} \label{ch:lab_infrastructure}
\section{Print Servers} \label{sec:printservers}
\subsection{Overview}

Since the user accounts are available on all machines in all labs, servers are essential to allow an almost seamless transition of student accounts between labs.  We have discovered that having a server in each lab has distinct advantages over having only central servers.  This server should handle a variety of functions to serve the client machines in the lab.  In CECS labs, the oldest and most obvious function for these servers is printing, so they are called print servers. They are named with the preface psvr followed by the room number, e.g. psvr416.\footnote{Printing is described in Chapter~\ref{ch:lab_infrastructure}, Section~\ref{sec:lab_printing}.}  The rest of the Print Server functionality consists of: Dynamic Host Configuration Protocol (DHCP), Trivial File Transfer Protocol (TFTP), Preboot Execution Environment (PXE) boot, NFS, Samba (SMB), Norton Ghost (for windows), and to function as NIS slaves.  

\subsection{DHCPD Server}

The DHCP daemon (DHCPD) server functions to hand out Internet Protocol (IP) addresses to the client machines in the lab.  The DHCPD has to be on the same subnet as the clients due to how DHCP addresses work.  Clients send a broadcast on the local subnet to request an IP, and DHCPD responds.  It is possible for this broadcast to escape the subnet, but that requires the router or a layer 3 switch to use an IP helper protocol.  This protocol maps broadcast requests to their origin subnets so that answers can be returned.  In the CECS labs, we prefer to keep DHCPD as well as many other services on the same subnet to relieve bandwidth congestion going into the main servers and that in the event a connection outside the lab is lost, clients can still function locally.  

In addition, the DHCPD server should hand out addresses based on Media Access Control (MAC) addresses.  This is because IP addresses are mapped to hostnames by Domain Name System (DNS).  Hostnames should match the labels listed on the client machines so that if a problem with a client machine is reported, Tech Support knows both where the machine is physically and how to reach it over a network.  IP addresses should not be handed out randomly for normal operations (though it could be done for cloning if static addresses are assigned later).  

Alternatively, static addresses can be assigned on a per machine basis during the cloning process.  The advantage to this is that clients will be on the network regardless of what happens to the DCHPD server.  The disadvantage to this is that the cloning process will require someone to manually enter an IP for each client at some point, and this will break the ability to do fully automatic lab installs.   Regardless how IP addresses are handled during normal operations, DHCPD must be turned on during cloning to support TFTP and PXE for network booting.  A sample ``dhcpd.conf" file can be found in Appendix~\ref{ap:lab_infrastructure}.

\subsection{Network Booting}
Cloning has become one of the most important uses of the print servers in the labs.  Full details for that are listed in the cloning section, but the print server side is described here.  The core part of the cloning requires the ability for clients to network boot (netboot) from the print server.  

Netbooting requires 3 things:  DHCP (listed above), TFTP, and PXE.You will also need NFS for the cloning process.  DHCP is required because the machines do not know who is who and require an IP address to get started.  DHCP is needed to support the bootp protocol which is used to boot from network.  After they have successfully netbooted, the IP address they have can be changed to the one they actually need or be mapped automatically with the Internet Systems Consortium DHCP Client (dhclient) if the print server knows their MAC address.  

TFTP is required for loading the initial boot image.  TFTP has very little security and must be highly restricted on the print server.  It has no authentication and only allows downloading of specific files.  This is necessary because during netboot, clients do not have any sort of identification or credentials that they can use.  TFTP is only used for the loading of the boot image initial ramdisk (initrd) and associated configuration files used by PXE.  The lab image to clone with is transferred with NFS after the machine has fully booted.  

DHCP and TFTP are used to support PXE.  This is the protocol used to support netbooting without the help of any storage device on the client.  Clients will go out and grab an IP from DCHP, then download a file listed in the Print Server's dhcpd.conf file.  That file (usually called pxelinux.0 or something similar) is then downloaded and executed.  It gets its configuration files, loads the ram disk containing the network boot filesystem, and loads the kernel.  From there, it is then usable as a stripped down Linux Operating System (OS) run fully from memory.  It can then be used to both create the master lab image and deploy the clones.  It can also be used to troubleshoot where a full installation is not available (though tools are limited to the size of the ram disk).

\subsection{Network File System}
The Print Servers also mount NFS shares from the core fileservers.  It uses automount for this because automount will periodically check if the mount fails and attempt a remount.  Linux clients can mount the Print Server shares, thus making them a relay.  However, we have found it easier to just authorize all lab subnets the privilege to mount the core server shares directly instead of using a Print Server as a middleman.  

\subsection{Samba Support}
Since the file servers are all Linux based with clients mounting via NFS, some compatibility is needed for Windows Clients to mount those shares.  Windows uses its own network file system called SMB (Samba).  This is not compatible with Unix NFS natively.  Thus, a translation of sorts is needed.  The Print Server handles this by mounting the NFS shares from the core servers and running a Samba server that acts as a middleman for the Windows servers to mount.  The Samba server requires a fair amount of CPU, memory resources, and network bandwidth to do on-the-fly translation of NFS to Samba. As a result, we keep the SMB traffic local to the labs and have equipped the Print Servers in the Windows labs with quad core processors that can handle a large classroom full of Windows SMB file requests. These would not be necessary if Windows could natively mount NFS. Microsoft advertises that they can natively mount NFS in certain versions of Windows 7, but we have been unable to make it work.

\subsection{Norton Ghost:  Windows Alternative}
For installing Microsoft Windows, a Linux network boot image is insufficient.  Instead, a tool called Norton Ghost is netbooted by all the clients and used to do a byte by byte copy of the Windows image to all the Windows clones.  This tool is proprietary and its workings are held secret by the Symantec Corporation, but it is free to use.  It is useful for Windows administrators who may not necessarily have experience with Linux.  One advantage it has is the use of multicast to send the image to all clients at once instead of one at a time.  The disadvantage to this is that if a client misses a packet, it must be cloned again.  An alternative to using Norton Ghost is just netbooting to Linux and using the command ``dd" to do a binary copy of the Windows image.  Either method is problematic on hard drives of different sizes as the image could only be the size of the smallest hard drive used in order to install correctly.  

\subsection{NIS Slave}
Print servers are important for authentication because they provide NIS information to the rest of the lab.  The core server holding the NIS namemaster does not grant login information to all machines.  Instead, it designates each Print Server in a lab as an NIS slave.  This way, the Print Servers themselves can authorize the clients under them.  This hierarchy grants obvious advantages to administrators, especially if the administrators themselves have a hierarchy. This hierarchy also reduces the amount of network traffic and requests handled by the core servers.  

\section{Filesystem Management} \label{sec:filesystem_management}
\subsection{Background}
Often in a lab environment, particularly an educational one, users will benefit from having persistent storage so that they may work day to day.   File storage has become less important in recent years because storage is now commercially cheap.  Nowadays, flashdrives that can hold gigabytes of data can be bought for less than one dollar per GB.  In addition, with the advent of cloud storage services, many companies offer free storage online.  Additionally, code repositories not only facilitate filesystem management, but can help with revisioning.   That being said, it is still very convenient to have a small amount of storage between computers in a set of labs.  

For network storage, central file servers are the most common.  Some implementations use distributed storage across an entire lab's hard drives.  This method requires a number of security and redundancy considerations and should only be considered if the support staff is able to maintain it.  The central file servers in our environment run Suse Linux and export user home directories with NFS.  

\subsection{NFS Exports}
In the CECS labs, there are 3 file servers each holding 7 hard drives apiece.  Each drive contains home directories for students and faculty members to use daily.  The drives do not have any special configuration and are simply formatted with a standard Linux filesystem.  Each drive is exported via NFS and its exports are only available within CECS lab subnets.  Because of this, we can set the exports to run in insecure mode for compatibility reasons.  We also export with "root\_squash" enabled for all but one subnet.  This prevents an attacker who may have gained root privileges from damaging or stealing all of the files held.  All of this is configured in the "/etc/exports" configuration file on each fileserver.  The simplicity of the configuration has made this system run well for roughly a decade.  Most issues hard drive replacement and client configurations.

\subsection{Home Directory Changes}
Although "root\_squash" is useful for preventing possible disasters, "no\_root\_squash" is highly useful for fixing things affecting all students.  For example, when updating between the Linux distributions Fedora to Ubuntu, students still kept the same home directory with the same set of configuration files.  Files that functioned in Fedora break various features of Ubuntu (most often with window managers).  These file issues are consistently the same with most students. Students can be given instructions, perhaps on the support FAQ, on how to fix issues on their own, but that inevitably leads to many emails to tech support.  Instead, it is useful to have a server or small subnet that has permission to mount student home directories on file servers with "no\_root\_squash."  That way, root can then make edits on all student homes at once via a script.  This script needs to have a lot of checks in place to ensure that it fixes the problem without interfering with other processes.  It also needs highly verbose logging to figure out if anything went wrong.  In its simplest form, this script would just loop through a list of affected student home directories, change directory (cd) to each of them, and run the necessary commands to fix the issue.  A copy of a simple script can be found in Appendix~\ref{ap:lab_maintenance}.

\subsection{Disk Quotas}
Total space for network storage is highly limited in the CECS labs.  Although hard drives have come down in price considerably, with almost 1500 accounts at any given time, a disk quota is necessary.  We set conservative disk quota at 512 MB for each student and 1 GB for each faculty member.  This may seem small considering the price per GB, but students write code in text files.  They are not doing heavy picture or video editing, so that amount of space has been sufficient so far.  However, we recently noticed that web pages are are becoming larger and browsers are caching more as a result.  Mozilla Firefox in particular now sets its cache limit at 1 GB.  Thus, students are filling up their quotas quickly just by visiting websites.  We have modified the cache in browsers to an older 50 MB limit, but we realize our file servers will need to be upgraded in the next couple of years to keep up with increasingly larger files.  Typically, when the fileservers collectively become more than half full, we begin to worry about space.  

\subsection{Automount}
We use the automount tool on our workstations and servers to mount file shares from our core fileservers.  The automounter has several advantages over the classic method of relying on the file system table (fstab).  The fstab mounts immediately from when the /etc/fstab is read on system startup.  This uses system resources when a mount is not being utilized.  In addition, if a mount fails, it will not remount until either a sysadmin remounts it manually or the system restarts.  The automounter addresses both these issues.  It runs as a daemon in the background and waits for NFS requests by other processes.  It handles the given request and only mounts the file share that it needs to.  In addition, if the mount fails for some reason, it actively attempts to mount again in case there was a random network issue.  The /etc/auto.master configuration file governs the automounter's behavior.

\subsection{Backups}
We have two stages of backups.  One backup is done nightly to a set of file server clones.  These clones have the same hard drive and export configurations as the main file servers.  Though it has never been necessary, theoretically, if a file server goes down, we need only change the hostname of the backup to get it up and running again.  The other backup is done every Saturday to a set of hard drive arrays attached via a Small Computer System Interface (SCSI) cable.  Both backups are done with the tar command to keep things simple.  Because this system is so uncomplicated, it is simple to maintain and troubleshoot.  Students are able restore file from backup by running the "retrieve" command and restore their entire home directory with the "shotgun" script\footnote{A copy of the shotgun script can be found in Appendix~\ref{ap:lab_maintenance}.}.

Though they do require periodic checks by our technicians, their simplicity makes our file servers the most stable point of our environment.  Unfortunately, due to higher space demands, their current capacity is quickly becoming insufficient.  Likely, once decommissioned, we will have no more file storage unless we receive a grant.   

\subsection{Student Media}
Students can also use their own USB drives to store their files if the space provided by us is insufficient.  Student media drives, however, pose a security risk. Upon insertion of a memory drive into a Universal Serial Bus (USB) port, Windows has the tendency to look onto the drives for drivers and may run malicious code that could infect the workstations.  This has become less an issue now that Microsoft has finally addressed it.  On Linux workstations, we simply do not allow any files on the drives to execute.  We also prevent the drives from being mounted with administrative privileges.  Students can use the "pmount" and "pumount" commands to mount/unmount the drives as normal users without privileges.  This also forces them to mount manually and prevents exploits of an automatic USB mounting system.  

\subsection{Malware}
We occasionally receive reports of malware on our lab workstations.  We have powerful anti-virus software but this does not prevent all pieces of code from running.  Many instances are just programs in memory that disappear as soon as a student logs off.  Others may just reside in temporary files that are easily cleared.  Regardless, we run an environment that teaches programming.  We cannot prevent students from running malicious code intentionally or unintentionally.  We do, however, have the ability to track which students are repeat offenders.  We have the authority to temporarily reduce the ability to do further damage by locking accounts.  Further action can be taken only by involving the faculty, the department administrators, or even upper campus administrators in the event of extreme wrongdoing. 

\section{Lab Printing} \label{sec:lab_printing}
\subsection{Background}
Although digital copies of work are becoming more common, physical prints are still a practical necessity in labs, especially for education.  Instructors still collect paper copies and write feedback on them.  With the advent of tablet computing this is becoming less and less necessary but is still a requirement as of this writing.  

\subsection{Printing Setup}
Printing in labs is relatively straightforward to set up.  The simplest way to set up printing is to put the printer on a network and configure it as such.  With a network printer, this setup is made even easier by plugging the printer directly into the network and giving it a static IP.  This works in situations where paper quotas are not needed such as with faculty offices.  This assumes that faculty will not abuse the print privilege.  

In a student environment, however, it is usually not a good idea to assume students will not abuse the system.  In the case of CECS labs, it was noted that implementing a paper quota where student have to pay a small amount per paper ream caused the amount of printing to go down to manageable levels.  As far as printing costs though, ink will almost always be the most expensive portion of print costs.  

That being said, a limited paper system is somewhat more complex to setup.  Most printers, even with network capability, will not have the features necessary.  Instead, the printer must be plugged into a print server that can manage print queues and a paper crediting system.  The server may have other functionality,\footnote{The other functionality is described in Chapter~\ref{ch:lab_infrastructure}, Section~\ref{sec:printservers}.} but one of its main functions is to supply the labs with printing.  

The first step is choosing which printing system to use.  On Linux, the two choices are LPRNG (Line Printer Daemon Protocol) or CUPS (Common Unix Printing System).  CUPS seems to be the more common choice in most Linux distributions as of this writing.  However, while arguably having more features, it generates much more network traffic.  This could become a problem especially in mixed Windows and Linux environments.  CECS chose to use LPRNG.  Our main reason was because it is much less complicated to implement printing filters, which are necessary for the paper accounting system.

The LPRNG handles network printing without much configuration.  Clients simply have to point to it and send print jobs.  The accounting system requires more configuration.  The CECS accounting system was built in house using the standard TCP/IP stack in Unix.  At its core is the ``paperd" program hosted on one of the core servers.  It holds the master accounting files to keep track of how much paper a user has and how much they have used.  It is run by a dedicated user account, ``prmaster," which also has its own group.  

To credit a user's paper account, an administrator is added to the prmaster group which has the ability to run the modpaper command.  This must be done on the same server that paperd is housed on.  Thus, the administrator must be allowed to login to the core server and be in the prmaster group to edit user paper accounts.  
\vfill\eject
The clients in the paperd system are the printservers in each lab.  These run LPRNG and service print requests.  As part of LPRNG, the printers have a filter set up to catch who is printing what and forward that information to the master paperd system.  Assuming the user has paper in their paper account, the print job is sent to the printer. This only works if the workstations send the job to a printserver and by default we have them point to the one in the same room. They can point to printservers in other rooms in the event paper or ink runs out. 

\subsection{Printing Issues and Recent Changes}
While this system is fairly efficient, it has a few disadvantages.  The paperd is not using any form of encryption.  Technically, if someone on the network acts as a printserver they can send paper requests.  This is not feasible on networks where steps are taken to prevent spoofing.  This system also does not prevent people from unplugging the printers and plugging them into their own laptops.  For that, physical security of the labs needs to be maintained.  More importantly though, this system only tracks paper.  Is does not properly account for ink cartridges which are the true cost of printing.  These take a lot of the maintenance budget, but careful planning allowed us to manage for many years.  

Recently, our printing system has begun to be phased out.  Academic Technology Services has installed their own printing system in our labs.  They have paid for the printers, the paper, and most importantly the ink.  This gives us a significant portion of our maintenance budget back and cuts down our workload slightly by not having to manage a paper system.  However, this cost is transferred back to the students. Students add monetary credit to their California State University, Long Beach (CSULB) identification card (ID) through an online service or the value transfer stations (VTS) located around campus.  Alternatively, users can also buy a courtesy card at these VTS.  

The cost per black and white page is 10 cents a page and the cost of color pages is 1 dollar and 50 cents per page.  The cost for printing under our system was approximately .008 cents for black and white.\footnote{We did not offer a color printing system.}  Though the cost of ink was expensive for us, if we were allowed to collect a small amount of money from the students to offset the ink cost, we calculate that a reasonable and sustainable printing system could be maintained in house by having students pay 3 cents a page for black and white.\footnote{The actual cost was calculated to be slightly less than 3 cents but we rounded up.}  

This new system has decreased student printing and has caused a significant amount of confusion because the students are not used to the new system.  Faculty who realize the cost have also started to implement digital assignment submission.  This saves students money, but students do not have the benefit of feedback on their work paper margins.  The final result of this new system and whether it will stay will ultimately be decided in the future.  


\chapter{Lab Installation} \label{ch:lab_install}
\section{Lab Images} \label{sec:lab_images}
\subsection{Overview}
The most time consuming part of managing a lab is building and maintaining the master image to the lab.  All workstations in the lab are clones of either the Windows master image or the Linux master image.  Linux master images are rebuilt every semester while Windows master images are upgraded.  Details on cloning can be found in Chapter~\ref{ch:lab_installation}, Section~\ref{sec:cloning}.  Descriptions on the various steps to build the master image are here.  What is most important about the image is the instruction set used to build it.  The actual image itself is pointless if there are no instructions on how it was built.  If instructions exist, they can further be tailored if distribution upgrades are made, or possibly if the choice of distribution changes.  

We determine what goes into the image by taking input from faculty and students from past semesters as well as from our own experience.  Our cycle involves building during the summer or winter months with changes noted from the previous semester.  We test as much as we can once the image is built.  Once we can consider the image relatively stable, we begin the cloning process.  We cannot test for all cases, so unfortunately, most of the testing occurs by the students and faculty during the next semester.  Changes during the semester are very difficult because we have to close down all the labs.  We can do minor changes with ssh pushes.  In extreme cases where functionality is broken to the point where a course can not be taught, we will rebuild the master image, close the labs on a Friday because it has fewest classes, and clone again with the changes.  This is always a last resort because it involves downtime and lab courses canceled for the day.  In addition, a change to fix something always has the risk of potentially breaking something else.  

This would be less of an issue if we got more early feedback from faculty members, but we have not found a solution to that problem.  As a workaround, we will occasionally poll the students taking the courses.  This might not work as some students polled may request something that is either infeasible or was just not made known to them by their instructor.  For example, students periodically ask if we can synchronize their files across Windows workstations without having to use a shared drive.  This is infeasible because it puts heavy strain on our lab switches.  It might be possible if we upgraded them, but this would require far more funding than we are reasonably allocated.  Some students, particularly seniors or graduate students, have been around our labs long enough to offer helpful feedback.  Generally, student requests are considered and will be used subject to feasibility and security concerns. 

Exactly what goes into the image is difficult to determine because of the changes.  Some things rarely change while others are temporary fixes that change every semester.  In the appendix section, I have provided a copy of our current install\_master instructions in \ref{appendix install}.  They are only valid now and will be changed at our next release cycle.  All of the following instructions will have a corresponding entry in the instructions. 

\subsection{Booting from Install Media}
At some point, a CD or DVD must be used to run the installer for the image.  This may not be a physical disk if the image is netbooted, but for CECS labs, we have not found a way to netboot from our chosen distribution of Ubuntu.  Fortunately, the CD only needs to be used once on the master image.  The rest of the labs can be cloned via network.  Once the CD starts, some type of setup program or script must be run to do a standard install of the distribution.  How this setup is run varies depending on the distribution.  In CECS, we run the server edition of Ubuntu and do a basic install with nothing but essentials running.  We added the various packages we need later.  

\subsection{Setting Up Administrative User}
Sometimes during the install, it will prompt to create a non-root administrator account.  This is because Unix and Linux should usually not be run directly with the root user under normal circumstances.  In our lab environment, the root user is only used to do ssh pushes and troubleshoot issues.  However, some distributions will force creation of a non-root system administration account.  To workaround this, we create an account called dummy.  After the install is finished, we use the administrative privileges this account grants to set a root passwd, then we use root to delete the account.  

\subsection{Setting Up Networking}
Network settings vary depending on how dhcp is set up.  The most important aspect of network settings is making sure a workstations hostname matches its physical label.  This is the only way to ensure each machine can be identified for troubleshooting purposes.  As mentioned in the printserver section, the dhcpd server can be given a list of MAC address mappings to IP addresses so that all the client machines can be set to take their network settings from dhcp.  The alternative is using static IP addresses which will require going to each machine during the cloning process and assigning the IP which matches the physical label.  How the settings are set depends on the distribution.  They are usually found in the ``/etc" directory somewhere.  

\subsection{Master Package List and Installation}
The reason we only install essentials during the CD install stage is because we only install packages we need.  We build a master packages list of things to install and run that list into a package manager.  The manager then goes out to its distribution's download site and fetches the packages requested.  It then handles the installation process and dependency resolving for all packages it is able to retrieve.  This process requires minimal interaction from a system administrator.  As a result, the number packages that can be installed is only limited by the minimum size of the disks the image will be cloned onto.  Since most Linux distributions rarely take up more than 10-15 gigabytes and we have disks of at least 80 gigabytes large.  

After the installation of packages, we remove unneeded packages that were brought in during the install process.  In CECS labs, the most important package to remove is CUPS since we use LPRNG for printing and the two can conflict. Ubuntu provides a nice ``autoremove" command that removes unneeded packages. We run this at the very end of the installation. 

\subsection{Manual Package Installation}
The package list only works for packages that are available from the Linux distribution's package repositories.  If a faculty member requests a package that is not available in the standard repository, it must be added manually to the image.  This will require extra work on the part of the system administrator building the image, so there should be a limit placed on the number of manual packages to install.  This can be lessened if an automated script can be used.  Actual instructions for installing packages manually varies depending on the package.  Sometimes, files only need to be copied to an executable path to be run.  Other times, they must be compiled from source code.  And for some complicated packages, there may be an arbitrary number of steps to set up and install the package.  

\subsection{Text-Only Login}
Part of our environment involves keeping text-only login screens and command lines.  This allows us to run courses which students will primarily use the command line for.  This will build their experience with it which is important for many industry jobs.  It also allows us to use special shell accounts that will run scripts on login.  For example, we use a user account called ``search" with the password ``account" that runs an account creation script at login that students use to register new accounts.  

Actually enabling text-only mode seems to be becoming harder and harder with each new Linux distribution release.  Most want to boot straight into a graphical user interface (GUI) to handle logins.  This makes sense for non-technical users, but makes things difficult for technical students.  Until recently, the easiest way was to set the default runlevel to 3.  This will allow networking and multiple users while remaining in text mode.  However, this is only possible on systems that run the classic Unix init process.  Our current distribution, Ubuntu, does not use init.  We have managed to still get text mode by removing the configuration script for the GUI login interface.  This presents several other problems, but after carefully working through them, we have managed to get our current system stable.  With each new distribution release though, new workarounds are needed for problems that appear.  

\subsection{Automount}
Information on what automount does can be found in Chapter~\ref{ch:lab_maintenance}, Section~\ref{sec:filesystem_management}.  We have automated its setup by just copying the configuration files to the appropriate directory.  However, upgrading in the past has invalidated our configuration file.  With each new image, we confirm that the configuration still works and find out how to fix it if it does not. 

\subsection{Groups}
Some of the groups in the ``/etc/groups" configuration file must be renamed or added.  We have a faculty group, a student group, and a staff group.  They have numbers 30, 40, and 50 respectively.  Back when they were created, there were not many Unix system groups, so the original administrators were not worried about conflicts.  While there are conflicts today, renaming the groups has not caused any known issues except on the Apple Macintosh platform.  

\subsection{Limiting Student Control}
There are certain tools that we wish to prevent students from using.  Any tool related to disabling the machine in some way has to be prevented.  All methods of putting the machine on standby or hibernation have to be removed from student control.  This is difficult to do because they usually have the ability to do this from the GUI interface.  Disabling controls in this interface is not always well documented.  In addition, students should be prevented from doing a total shutdown and even rebooting the machines.  This is because while the machines are powered off, they cannot be accessed remotely and must be physically powered on again.  In addition, the combination ctrl-alt-del has historically been used to reboot the machines, so this must also be disabled.  

Students must also be prevented from editing their own account information, particularly the name information because this is how we identify them.  Since our environment uses Unix NIS and YP  maps, some YP tools must be disabled from student access.  We do this by replacing the ``ypchfn", ``ypchsh", and ``yppasswd" binaries with dummy versions that only print warnings.  Local versions of these tools are similarly replaced.  We do this because in order to maintain our crossplatform accounts, we must force students to only create and change account information and passwords through the custom interface we have created.  Further information on this can be found in the crossplatform considerations section.

\subsection{Power Saving Configuration}
To conserve power, we set the monitor to go black after 15 minutes, and turn off after 30 minutes.  This is set in the ``/etc/rc.local" configuration file which runs at startup.  

\subsection{System Log Configuration}
Our system logs are usually retained for only 30 days.  However, we set them to stay for a semester long (15 weeks) to match our semester environment.  This way, we can see a trail of logs if a problem has been occurring all semester long.  We can also optionally set up log forwarding to a central log server for auditing purposes.  However, we have not been required to do this.  Furthermore, a log server would need to have many gigabytes of storage space to hold logs for all of our workstations for a semester.  

\subsection{Secure Shell Key Setup}
Descriptions for setting up and configuring ssh can be found in Section~\ref{sec:ssh_push}.  We have further automated this by creating a tarball of a properly configured ssh directory and then untarring the files onto the new master image.  

\subsection{Lab Printing}
A description of our printing system can be found in Section~\ref{sec:lab_printing}.  A default printer must be chosen for the master to test printing, but it will be overridden during the cloning process.  In addition, the ``paper" command must be copied down so that users can check their quotas.  

\subsection{Local Support Commands}
Our support commands are described in Section~\ref{sec:student_faculty_help}.  All we do on the master image is copy them to the appropriate place, set their permissions, and test them to make sure upgrades have not broken them.  We attempt to fix them if they are broken. 

\subsection{Restricting Access}
Access to the machines should be restricted only to those who need to use them.  We restrict access to our labs by only allowing students who are Computer Engineering or Computer Science majors or who are taking a CECS course.  We do this by carefully managing the authorized user accounts on our nameserver.  When students no longer have a need to access the workstations, their shell is set to an expired shell that prints an error message before logging them out.  They can still use the ftp protocol to reach their files though.  After a semester, we remove their accounts and all associated files.  

Steps have to be taken to restrict outside access to workstations as well.  In our environment, every workstation has a public facing IP address.  Access to it should therefore be restricted.  We do this by only allowing connections from IP addresses in our environment and denying everything else.  This is done through the ``/etc/hosts.allow" and ``/etc/hosts.deny" configuration files.  While this works fine for us, it is not ideal because every server and workstation must be configured this way.  For clones, it is not much of an issue, but for servers these steps could potentially be forgotten.  That would expose the server to attackers from across the Internet.  Ideally, our environment should be behind a restricted firewall module that would have a strict access control list (ACL).  This module is exceedingly expensive however.  We have a layer 3 switch that can do simple ACLs, but we must still secure each workstation ourselves and through the cloning process.  An even more secure route would be to use a Virtual Private Network (VPN) module that only allows authenticated users to connect to our network.  This has been implemented campus wide, but we cannot justify the added cost of VPN modules to implement it for our network.

\subsection{NTP Server}
The network time protocol (NTP) server must be pointed to the same server that every other machine in the lab does.  This is necessary because all machines should have the same time for troubleshooting purposes.  

\subsection{Bootloader Setup}
The bootloader settings must define what hard drive partition to point to, what kernel to link to, what label the system is called, and what type of video output should be used.  These are defined in the ``/etc/lilo.conf" file.  These are necessary for the clones because they will be reading this configuration file to create a boot sector.

\subsection{Rebooting to Cleanup}
Descriptions of unclogging methods can be found in Section~\ref{sec:unclogging}.  It is currently implemented by adding the reboot command to root's crontab at 3 AM every night.  

\subsection{Login Scripts}
Login scripts are run every time a user logs in.  That can be made as quick fixes to issues that users may be having.  Currently, we only have two.  All of the other fixes were temporary and permanent workarounds were found.  One script sets the environment variable ``hostname" to be the system hostname for backward compatibility reasons.  The other script checks a users diskspace and displays a screen warning if it is too low.  

\subsection{Message of the Day}
The default message of the day (MOTD) gives a lot of useless information for students.  Instead, we change it so that it only shows a welcome message and explains how to use the GUI interface.  

\subsection{Miscellaneous Tweaks}
There are other miscellaneous tweaks we make to the image.  These tweaks are mostly on a per release basis and change each semester.  What may have been a problem with one distribution release may be fixed in another.  

It would be impossible to document all of these tweaks so I will provide some common examples.  For instance, the Mozilla Firefox web browser makes all sorts of changes between releases that could bring up problems.  With the latest release, the local cache has been raised to 1GB.  Our quotas are only capped at 512MB, so Firefox alone could fill up a student's quota.  Some simple examples involve a program name changing.  The email client pine changed to alpine.  This confused users who were used to running pine, so we created a symbolic link from the pine command to the alpine command.  The ctrl-alt-backspace key combination was the default method to shut down the GUI interface.  It changed in one release to the confusion of users, so we had to manually rebind it to the way it was.  Lastly, there are all sorts of issues with each window manager release.  In our latest lab image, the newest window manager was broken, so we had to force the old one to be used.  

\subsection{Considerations for Different Hardware Architectures} 
We have had issues with going from a 32 bit architecture to 64 bit.  The first issue involved the fonts for the system changing to a different package.  Adding that package to the master package list fixed it.  We also to recompiled our custom programs to 64 bit because in the latest release of Ubuntu, libraries that support 32 bit code were not installed by default.  We also added the 32 bit libraries to our package list in case we still need the compatibility.  We have also run into cases where the default 64 bit drivers would not support our hardware, so we have had to recompile the drivers and add them as kernel modules.  

\subsection{Summary}
Overall, each semester, the master image requires considerable effort to ensure it is adequate for use in our labs.  However, once it is built, our cloning process allows us to clone all ninety of our Linux workstations in just over two hours.  This assumes no problems occur along the way and we always have a few each time. 

\section{Cloning} \label{sec:cloning}
\subsection{Overview}
The master image for lab loads is very important, but just as important is the method for deployment.  With the number of machines we have present in the labs, installing each machine individually would require far more time than we have available.  In addition, because of human error, many inconsistencies between lab machines would appear.  Instead, we define a method to copy the master image to the rest of the labs.  We do this by copying the lab image to a file and moving that file to a server that the lab workstations can access.  The specific instructions we follow to do this can be found in Appendix~\ref{ap:lab_installation}.

\subsection{Setup and Preparation}
In our labs, the master image file is moved to the Print Servers.  The Print Servers have to be set up with DHCP, TFTP, PXE, and NFS.  Specific instructions for this process can be found in the Print Server section.  The image must be copied to the NFS export directory on each Print Server in each lab.  We have tried using a central server that all workstations access for the image, however, this caused all workstations to compete for the bandwidth given by one switch.  This slowed the process down significantly.  By deploying the image to each Print Server, it allows the bandwidth used to be isolated within each lab subnet.  All labs can then be imaged in parallel.  

Deployment of the image requires configuring the workstations to boot from network.  The clients must have their BIOS setup so that netbooting is enabled and is first priority.  It may also be a good idea to disable other booting options since they are unnecessary for clones and could pose a security risk.  The BIOS should also be passworded to prevent changes.  Technically, the BIOS password can be reset by opening the machine and setting a jumper, but in our environment, the workstation cases are locked so that the jumper is inaccessible.  

\subsection{Deploying the Clone Image}
The workstations can be booted one by one to load a temporary network image.  It will load a bootloader first to determine what operating system to finish booting with.  The top option will likely just be to boot from the local hard disk.  Another option we show is to boot from network (and there may be multiple network images).  The image to clone with is just a variation of the distribution installer modified to be used for cloning.  It contains the clone script and any other tools for use in cloning.  Before the clone script can be run, the hard disk will need to be partitioned manually once.  This is per hard disk not per clone operation, so once it is finished, it does not need to be done again until hardware changes.  Once the formatting is done, the script can be run while booted from network.

This script does several things to prepare, deploy, and configure the clone image.  It takes a single argument of the last octet of the workstation's IP address.  It first enables networking if it is not already enabled.  Then it sets the machines address temporarily to the one it was given in an argument.  It then sets the default gateway for the lab.  This could require either a second argument to the script or separate scripts for each lab subnet (we use the latter for convenience).  It then runs filesystem tools to make a new file system and a swap area to the partitions.  It then mounts the hard drive partitions to a local mount point.  It will also need to mount the NFS export directory of the Print Server so that it can access the master image.  The master image is in compressed tarball format.  Once it can reach this file via a network mount, it can untar the image onto the local hard disk.  This is the longest step.  After it is finished, two more mount commands are needed to bind the process and system directories of the new image.  After that, the bootloader, LILO, runs to install the bootloader into the boot sector of the hard drive. 

At this point, the system is bootable.  It is not yet configured to be unique though.  The last thing the script does is set the hostname and the permanent IP address by editing the configuration files on the hard disk partitions.  On some systems, the monitor may not work with the defaults the master image had.  If so, the appropriate configuration files are changed to update that.  Some other configurations may need to be changed, but these vary for each distribution and release so we have no way of documenting them.  The script edits the ``/etc/printcap" file and sets the default printer to be the one in the room in which the workstation is located.  All workstations in a room are clones except for the printserver.  Figure~\ref{fig:RoomNetworkLayout} shows our lab layout. 

\begin{figure}
  \begin{center}
    \scalebox{0.7}{\includegraphics{../figures/RoomNetworkLayout.png}}
  \end{center}
  \caption{Lab Layout}
  \label{fig:RoomNetworkLayout}
\end{figure}

\
\vskip1pt
\ 

\subsection{Cloning Efficiency and Potential Improvements}
This system is fairly efficient and only requires about 2 hours to clone all three of our labs if cloned in parallel.  The only interaction between the sysadmin and each clone is to run the clone script with the IP address of the machines.  This process could be improved though.  To do so, every workstation needs to have its MAC address copied to the Print Server's DHCP table.  In addition, a copy of this table would need to be available during the netboot since the small netboot image does not usually have a DHCP client.  This would then prevent the need for the sysadmin to enter the address since every machine would be uniquely identified by MAC address.  This can be further automated by broadcasting wakeonlan packets from the Print Servers to the machines to start them.  This would prevent the need for the sysadmin to even enter the lab at all.  

All of these improvements only work to the benefit of the system administrator though.  They do not lessen the time it takes for the labs to deploy from two hours.  To this, the biggest bottleneck, network bandwidth, would have to be addressed.  The expensive solutions is to do switch upgrades.  Currently, we use 100MB managed HP switches.  We could update them to 1GB (1000MB) switches and that may cut the time down significantly.  However, there are two inexpensive solutions.  The first is to enable QOS (Quality of Service) on our managed switches.  This would force the bandwidth to be evenly distributed across machines whereas currently one machine can take most of the bandwidth and slow all other machines down.  QOS requires managed switches to support it.   The scripts could further be timed such that the machines stagger their installs.  We could send wakeonlan packets to each workstation on a timer, allowing each machine five minutes to get high bandwidth before others start competing for it.  As more machines enter, machines that started earlier finish.  Five minutes may not be the ideal time.  That can be optimized with each successive cloning session.  These improvements may or may not work as expected because we have not tested them yet.  Fatal drawbacks may be discovered during testing.  



\chapter{Lab Maintenance} \label{ch:lab_maintenance}
\section{Pushing Commands and Updates} \label{sec:ssh_push}
\subsection{Remote Commands Background}
In a large environment with hundreds of systems, a method to push commands and updates to them needs to be in place.  This is a separate process from installing and cloning a lab.  For example, all systems in a lab need to be shut down for a planned power outage to the building or some other maintenance.  Without a method to remotely shut them down, a team of techs is needed to go to every system and shut them down cleanly.  This method quickly becomes impractical in labs with hundreds if not thousands of machines.  

Historically, Remote Shell (RSH), part of the rlogin tools, has been used for this purpose.  RSH uses the TCP port 514 to send remote commands to hosts.  It does this by using the username and the source host sending the commands as the authentication mechanism.  So long as this combination is in the authorized hosts ``.rhosts" file, the command will be accepted.  This is done without any form of encryption over a TCP connection.  In an area where the network is heavily secured, this is not necessarily a bad idea and in the past has been used in our labs.  However, it is fairly trivial to spoof an RSH command by impersonating a user and host.  

The telnet protocol can also be used for this purpose.  This protocol requires a username and password to authenticate.  It uses TCP port 23.  Like RSH, it is not encrypted.  While it would be harder to spoof, it is possible to sniff network traffic and gather the authentication credentials.  It is also possible to do a man in the middle attack.  This is less of an issue on switched networks then it was on hubs historically, but the threat of this occurring still exists, especially if the credentials used are the same for all machines.  In addition, scripting a telnet command can be tricky without using a language that can expect a password prompt and act accordingly.

\subsection{Secure Shell Authorized Keys}
As of this writing, the most common protocol used is Secure Shell (SSH).  SSH functions similarly to telnet as far as functionality goes.  However, SSH uses public key exchange to create an encrypted channel.  In addition, it uses host key identification to make sure that the host connected to is who it says it is.  Thus, unless the user disables strict host key checking,\footnote{Strict host key checking is enabled by default on OpenSSH servers.} an SSH session is practically immune to man in the middle attacks and much more difficult to spoof than RSH or telnet methods.   

An SSH session begins by establishing a connection and then exchanging host keys.  SSH authentication comes in either via username/password or via authorized keys.  Username/password prompts function the same as telnet and are thus not very useful for pushing out commands to a lab.

For the purposes of pushing out remote commands, authorized keys are used to prevent the slow process of typing in passwords for every machine pushed to.  Keys come in pairs of public keys and private keys.  Each user can generate a key pair using the SSH-keygen command.  To authenticate with keys instead of passwords, your generated private key should be kept in a ``.SSH/" directory of your user account.  Your matching public key is sent to a remote host which uses it to encrypt traffic to send back to you.  For example, if you push your public key out to root's home directory on a system, you will be able to login as root on that system without a password.  Furthermore, due to the key length, it is much more difficult to brute force a private key than it is a password, and dictionary attacks are practically worthless in this context.  In the CECS labs, we build the master image with the keys and configuration files already in place.  A diagram of how the pushes work can be found in Figure~\ref{fig:sshpush}.

\begin{figure}
  \begin{center}
    \scalebox{0.7}{\includegraphics{../figures/sshpush.png}}
  \end{center}
  \caption{SSH remote command push.}
  \label{fig:sshpush}
\
\vskip1pt
\
\vskip1pt
\
\end{figure}


Because a password prompt never occurs, SSH can be used to run a command immediately after authentication.  With the use of semicolons and other special shell characters, multiple commands can be run.  Hypothetically, a command to download a script from a webserver can be followed by the command to run that script.  However, each system touched must have their public key accepted by the push server at least once to prevent prompts from occurring during automation.  For clones, it is useful to have all the clones use the same key.  And for subsequent clones, the previous key may be used to prevent SSH from failing due to a key identification mismatch.  This should not be done in highly secure environments though, as it circumvents protection against man in the middle attacks.  

\subsection{Push Script}
Thus, a relatively short script can be written to send commands to all participating servers.  In its simplest form, all this script does is loop through a list of hosts and execute the SSH command on each host.  In some shells, the loop may malfunction if SSH is not run in batch mode to prevent stdin, so care must be taken to remember that argument if using such a Unix shell.  

A more advanced script would fork each SSH command to the background to run commands on hosts in parallel.  This has the potential to overload the push server,\footnote{The push server is most likely to be a fairly old machine.} so adding a delay after so many forked commands would help if the CPU is being taxed too much.  A fair amount of CPU cycles are used to negotiate the key exchange and encrypt the traffic.  The other issues that may occur are hosts that hang the SSH session.  This will cause the SSH command process to hang until killed.  Sometimes it can only be killed with signal 9.  If hosts do not seem to be doing this often, it may be fine to overlook and fix them manually after the script is run.\footnote{We log the malfunctioning systems and investigate them later.}  To remedy failed SSH sessions, at the end of the script, we force the script to sleep for a reasonable amount of time then kill all SSH sessions with the ``killall" command.  


\section{Unclogging}\label{sec:unclogging}
\subsection{Background}
Sometimes restricting privileges on lab machines is not enough to prevent students from clogging the machines.  If not locked down sufficiently, students can potentially cause headaches to system administrators.  Often, the student's intentions are not malicious.  They could simply be doing assignments that have the potential, if done incorrectly, to overload the lab machines.

For example, in an operating systems course, students must learn how to properly create, fork, kill, and end processes.  As with any programming, practice is needed to properly learn a new programming technique and many mistakes will be made along the way with each program run.  With operating systems concepts, many leftovers will remain after each program run.  Forking of processes will leave behind orphaned children if the program does not exit cleanly.  In addition, if a student's loop runs away, the potential for overloading the system with forked processes exists.  
\subsection{Limiting}
To prevent this, steps need to be taken to limit what a student can do.  For example, a student can run through a finite set of process identifications (PIDs) very quickly with a loop.  If the system does not have a limit for a single user to have, the user can make the system run out of PIDs.  By default, some Linux distributions set the kernel limit for a single user to 1024 PIDs.  However, on other distributions and other types of Unix, this limit is not necessarily in place or is too high.  Thus, it is important to check this and set it accordingly.  It may be a good idea to drop the process count to below a hundred if the student does not require a window manager (which could also be useful for forcing students to learn good command line habits).  

In addition to PIDs, students can still clog the machine by using too much Central Processing Unit (CPU) resources, memory resources, or disk space.  CPU and memory usage can both be limited by editing kernel parameters by user.\footnote{On most Unix systems, CPU and memory usage are unlimited by default.}  Disk space can be limited by setting a quota on home directories.  However, users may still be able to write to temporary directories on the local machine and fill those up.  It might, therefore, be useful to mount those directories from separate partitions of limited size.  Temporary directories may still be important for operating system performance.  If it becomes necessary, a cronjob\footnote{A cronjob is a scheduled task that the system runs at a user defined time} could be created to delete excess temporary files.  However, this problem is rare, so we have not implemented such capability.  Besides that, each machine can be cloned again within 10-15 minutes.  We have found that it is simpler from an administration perspective to just let the students use the full resources of lab machines (minus administrative privileges) for programming.  

\subsection{Simplest Solution}
For many of these issues, a very simple solution is to schedule a reboot every night.  This clears all orphaned processes, interrupts any processes that could be using resources, and resets the environment.  We have done this by making an entry in root's crontab.  The reboots take place in the middle of the night while the labs are closed. 

This may create a few issues on some Linux distributions though.  On Ubuntu, the file system can get somewhat dirty after a few weeks and on reboot will mount the file system read-only if it detects errors.  To prevent this, we set it to do a File System Check (FSCK) if errors exist.  Delayed login mode is required for this though, but since students are prevented from manually rebooting the machines, this will not affect much.  It also helps that the reboots take place at night.  Lastly, the setup should not require an administrator to interact if problems are found.  We have set FSCK to fix errors automatically without asking for confirmation.  Instructions on how to do this can be found in Appendix~\ref{ap:lab_installation}.

We have also found that booting into verbose mode is very helpful for troubleshooting machines quickly especially while classes are in session.  A splash screen is helpful when hiding confusing output from non-technical users, but in a room full of computer science students this really is not necessary.  

\section{Local Repository}\label{sec:local_repository}
\subsection{Background}
One of the future changes we could make is implementing a local software repository for our given Linux distribution.  A repository contains software packages and updates supported for a given distribution.  Whenever we build a master image or update it, it must go out to a distribution repository for what we need.  This can generate a huge amount of bandwidth for large packages and update request lists.  In bandwidth limited environments, it would be a good idea to create a local repository on a server that workstations and servers can pull updates from instead of going out to the Internet.  This server would service these requests.  The reason we could potentially need it is because our workstations are not being updated during the semester.  For that, all 90 workstations would have to download the same set of updates.  This would cause unneeded and excessive bandwidth requests against the campus routers.  

\subsection{Possible Setups}
There are two types of repository servers that we can use.  A repository mirror downloads all packages and updates from a remote repository and stores them locally.  It updates itself periodically to be current.  It initially requires significant bandwidth resources.  The currently mirror for Ubuntu requires about 30 GB.  This means we need that much hard disk space plus some extra in case the image gets larger.  Disk space is les of a problem than bandwidth though.  If the appropriate approvals from the university technology administrators can be obtained for the initial bandwidth use, it will prevent the need for more bandwidth later.  There will still be a need for some significantly sized bandwidth during updates, but it will be minimal.  

\vfill\eject

The other method is a repository cache.  Rather than mirroring an entire repository, this server acts as a middleman for repository requests.  Clients make a request to it for packages and updates.  It goes out to the remote repository and downloads it.  It then allows the client to download the package from it.  It does not remove the package though.  Instead, it caches it locally in case other clients need the same package.  Thus, if all 90 workstations need the same package, it is only downloaded once to the repository cache server who makes it available on the local network.  

We do not have a set of instructions for building such a server.  However, we know it requires that an apache webserver be set up because client machines download via the standard web protocol.  This may seem insecure, but the packages are one-way hashed and the matching hashes are downloaded initially via a secure connection and compared.  This also helps in the event of data corruption.  The repository mirror or cache is then shared via the apache web directory.  All clients can then be pointed to the webserver as their target repository.  In the case of mirroring, the last step is to set up periodic synchronization so that the local repository remains current.  We assume this is done automatically but a crontab entry can be created if necessary.

\subsection{Why Local Repositories Not Needed}
We do not use a local repository as of this writing.  For security reasons, our labs are relatively isolated from the rest of the world.  We therefore do not need immediate automatic updates.  This could become an issue if we choose a more targeted distribution, but we have not seen a security incident in our Linux labs as of yet (we are far more worried about our Windows environment).  The other reason we hesitate to implement it is because some software updates can break a course curriculum software.  For example, if our major development IDE Eclipse were to get an update mid-semester that changes its interface or breaks it, we would have confusion from the students and anger from the faculty.  Even security updates could potentially cause issues.  The reason we would consider it is in case a critical package is needed mid-semester that no one told us about.  Rather than clone the entire lab again, we would just need to run the package installation tool on the required workstations via an ssh push.  However, so far the need for it is mitigated by the fact that we build to a single master image and clone. This has been sufficient so far, but it would be efficient to be able to update mid-semester if needed without taking down all of the workstations to perform the cloning. 


\chapter{Special Considerations} \label{ch:special_considerations}
\section{Specialty Labs} \label{sec:specialty_labs}
\subsection{Overview}
General labs are not always sufficient for all applications.  Sometimes, specialty labs are necessary to do functions that general labs cannot do.  These labs will often have special requirements, such as unique equipment, a different network configuration, a different accounting system, and often a different room environment.  The labs we have covered consist of:  a system administration lab, a network administration lab, a computing cluster lab, a dedicated security lab, a terminal lab, a virtualized lab, and an individual projects lab. While these labs enhance our environment considerably, maintaining so many different configurations with such a limited staff becomes difficult. Thus, these labs are difficult to keep maintained permanently without faculty assistance.  
\subsystem{System Administration Lab}
The system administration lab is for teaching student system administrators (sysadmins).  Because they are given administrative privileges on their machines, they have a huge risk of causing damage or installing something malicious.  For that reason, this lab is isolated from outside networks and the Internet.  Students still need to do work remotely, and certain things must be brought into the lab, so a gateway server is set up with two network interfaces.  This server is open to the outside via one interface, but does not route traffic between the two.  This way, a student can remotely access the gateway, then from the gateway access the rest of the lab.  

Because of this isolation, the lab needs a Linux distribution that can be installed with all tools locally.  Certain distributions, such as Ubuntu, only offer a CD with the core installation and make administrators download what they need on top of it.  Since that is not an option in an isolated lab, a distribution that is fully offered on a CD or DVD set is needed.  In the CECS sysadmin lab, we use Slackware.  This is also arguably the best Linux distribution to teach system administration with because it makes administrators do everything manually.  From an efficiency standpoint, this is not ideal.  However, by making students do everything from scratch, it allows them to troubleshoot problems and truly understand what is going on.  

The sysadmin lab also needs more than 1 subnet in the lab.  For this, we use a second server to act as a 3 way router.  This division is necessary to separate the servers in the lab from the lab workstations, as well as to split the workstations up.  Since students have administrative privileges, they can potentially configure network settings wrong.  For example, if a student doing a system install assignment mistakenly gives their machine the address of the default gateway, they will cause the rest of the lab to send traffic to both them and the gateway.  By dividing the subnets up, they will only take down their subnet rather than the entire lab.  And they will not affect the servers.  It is still annoying to trace who actually did it but it is manageable.  It will also give the students practice with subnet calculations since they will have to work with small subnets under 8 bits.

As with the main labs, the sysadmin lab will need its own set of accounts, its own core servers, its own master image, and be cloned separately.  It needs accounts because the lab can not talk to the core servers for the rest of the labs.  In addition it will need a server to PXE boot from for clones (ideally, it would be the server that functions as a router in the lab).  Lastly, it would need its own master image.  The image would be simple, however, because one assignment in the labs would be rebuilding the image from scratch on individual machines.  It cannot be too complex, just enough to show some basic optimization and software installation choices that a good sysadmin should know.  
\subsection{Network Lab}
The network administration lab is very similar to the sysadmin lab except for a handful of other things.  This lab's purpose is to teach network fundamentals to students as well as troubleshooting and administration skills.  For research purposes, this lab would also be used to demonstrate applications of network theory.  

Since it will be dealing with network devices, the machines would benefit from having wireless cards to connect to wireless nodes.  It would be ideal if this lab were in a location where it would not interfere too heavily with other wireless infrastructures.  Many years ago, it was also useful for these workstations to support other network protocols.  However, since TCP/IP has become the standard across the board, this is no longer an issue.  

If this lab is teaching students home networking, they would benefit from the lab using the most common subnets:  192.168, 172.16, or 10.  In addition, this lab could not remain completely isolated since home networks expect to be able to reach the Internet.  Thus, a 2 way firewall can be used to restrict all inbound traffic behind a NAT (Network Address Translation) border.  It will also restrict outbound traffic to only the most common ports (web, ftp, ssh, etc.).  It should also be blocked from accessing other on-site resources and thus only have an avenue to the public Internet.  This will prevent the majority of issues with student network administrators.   
\subsection{Computing Cluster Lab}
A computing cluster is a set of machines working together to perform a complex computation.  This is a fundamental function for distributed computing.  They usually consist of a head node, which acts as the controller, and child nodes, which work together to perform calculations and communicate the results to the head node.  A lab of these machines requires special considerations.  

Often it is not a lab per se, but a server room full of racks.  This is not absolution necessary but it helps in situations where things can be automated.  In some clusters, there is no automation for the cluster software so each machine must be touched directly to add them to the cluster.  This is not very desirable, but it is common with software made by Adobe.  In this case, all the nodes would be workstations with monitors, keyboards, and mice.  This also makes it simpler to add hardware since you only need to set the hardware on a table and connect it rather than rack mount it.  

It is important that this lab has adequate cooling to support the heat these nodes will generate.  This is probably the most important aspect of the lab.  The room will hugely benefit from having a backup power supply (a generator or UPS) and a backup air conditioning system.  If the cluster is not well funded though, these things may not be necessary.  This would mean the cluster would be periodically shut down and hence would not be suitable for production work.  

The lab machines should have a very barebones operating system installation and only have the software used to run the cluster installed.  Anti-virus and other utility software that could use precious CPU cycles should not be installed.  As such, this lab would also benefit from being isolated from the outside world and have a gateway device people remotely login to.  It is a relatively insecure lab in that respect.  Access to this lab should be restricted only to the people using it for its designed purpose and it should not for any reason be used for general purpose computing (such as web browsing, email, etc.).  If it needs to go outside to download and upload finished computations, only the head node should be allowed to do it.  

Lastly, this lab needs a powerful network infrastructure.  Since these machines are working heavily with each other, the network becomes very chatty.  It should be using a switch (not a hub) with a high throughput.  A nice gigabit switch without QOS (quality of service) would work nicely.  Since this lab is relatively isolated, there is no point in giving them public IP addresses either, so they should be fine with a private 192.168 subnet.  

Given all the above, a dedicated cluster lab would be very costly both in funds available and in room/rack space.  Although CECS has a dedicated cluster occasionally, when we do not we compromise by allowing cluster software to run on our main Linux lab workstations.  Thus, one of our labs of 30 machines can be turned into a makeshift cluster so long as the cluster software does not require administrative rights.  
\subsection{Terminal Lab}
A terminal lab does not consist of workstations but of terminals.  Terminals (sometimes called dumb terminals) do not perform any work on their own.   Rather, they exist just to allow users to communicate with servers they are tied to.  Users do not do any work in terminal labs but instead connect to servers to do work there.  Terminals have not seen much use in recent years.  They were much more common when computers were much larger and could only fit in dedicated machine rooms.\footnote{Mainframes were a perfect example of this}  However, users can still use lab workstations to do work on servers with network clients.  For example, a lab could consist of very weak Linux workstations that only provide students with an ssh client.  They were ssh to a server in another room and use that server's compiler to develop code.  Alternative, they could use a remote desktop client to communicate with Windows Terminal Servers to use Windows-only applications.  These labs are relatively easy to manage because they have very simple purposes.  Actually setting up the servers they connect to represents the bulk of the workload for administrators.  
\subsection{Dedicated Security Lab}
A dedicated security lab is for teaching students about system and network security as well as potentially prepare them to become penetration testers (pen testers).  This lab could also be used to test and analyze malware.  As such, there is no reason this lab should not be completely isolated.  Even a gateway server to access it remotely is a risk because the gateway server could become compromised.  If the lab absolutely has to be accessed remotely, it should be through a remote KVM console that only allow a user to see screen output and pass keyboard strokes (no network traffic allowed).  Other aspects of this lab are beyond the scope of this thesis.  
\subsection{Virtual Machine Lab}
This lab does not take up space in room.  Rather, very spacious servers host virtual machines (VMs) that users access only remotely.  These servers must have a lot of CPU, Memory, and Hard Disk space to support operating systems running on top of them.  The advantage to this environment is very easy deployment of VMs from clone images.  All work can (and should) be done remotely and with automation.  In addition, because they are virtualized, they can be torn down and replaced quickly if a problem emerges.  The disadvantage is these VMs can not quite reach the performance of separate physical hosts.  In addition, if the VM host is not set up to properly schedule resources, one VM can potentially take up most of the resources available and slow down all other VMs on the host.  
\subsection{Individual Projects Lab}
Lastly, it is useful to have a lab solely dedicated to individual projects.  In CECS, we have a dedicated senior projects and graduate student lab that students can use to work in.  This lab may be the most difficult to manage because there are so many different project students do in this lab.  They may or may not even need workstations or network access but instead just need a space and tools to work on hardware.  Determination on how to manage this lab should be on a project by project basis.  Each student should have some sort of project plan or charter and hold discussions with lab managers on what exactly they plan to do.  Communication with the users is more important than anything else in this sort of lab environment as each project team may have special needs.  

\section{Crossplatform Considerations} \label{sec:crossplatform}
\subsection{Background}
Up until now, only Linux based labs have been described.  However, to be competitive in today's market, many platforms need to be considered.  In CECS, we have a combination of Windows, Mac OSX, and various distributions of Linux.   Getting all of them to play nicely with each can often be an aggravating (sometimes maddening) process.  In addition, the choice of platforms is often dictated by people other than the System Administrators. 
\subsection{Centralized Accounting}
The most important thing to consider in a mixed platform environment is the user accounting system.  This is the system that manages login user IDs, user attributes, and passwords.  User attributes may consist of a user's full name, student/employee name, contact email, etc.  Some other useful attributes could be a list of courses that student takes or what his major is.  

At the very least, the system needs to manage user IDs and passwords centrally.  If this is not done, each platform must have its own central authentication system set up and each user must remember separate sets of user IDs and passwords (though the users may set them all the same).  This will also make shared resources among the different platforms difficult to use securely.  File servers in particular need a central authentication mechanism to know which files are owned by whom.  

There are two choices for central authentication that we have looked into.  The first is Network Information Services (NIS) with its set of yp commands.  This was created by Sun Microsystems as a method to exchange network and directory information between Unix systems.  The advantage to it is that every flavor of Unix supports it and will share passwd file information easily.  The disadvantage to it is that is shares information in clear text form.  This becomes a problem because password hash values can be sniffed if the network is breeched.  The other disadvantage is it will not natively support Microsoft Windows systems.  

The second choice is Microsoft Active Directory.  This is Microsoft's method for sharing information across Windows systems.  It can be difficult to configure but it is the ideal (and only) choice in a pure Windows environment.  The disadvantage is that Unix systems can not usually use it.  There have been many attempts at achieving compatibility and we are currently looking into another one, but no smooth tool has yet been made to fully support all of Microsoft's features.  

What we have done in CECS labs is have NIS manage authentication and created our own system to transfer credentials to a Windows Active Directory domain controller (sometimes called a PDC).  Unix clients can bind natively to an NIS namemaster while Windows clients must bind through Active Directory (AD).  Thus, we need two types of authentication: NIS and AD.  Our solution to this is called the search program.  We have a special user account called ``search" with the password ``account" that is the only method students can use to create their user accounts.  This program stores the usernames and passwords that students create in a clear text file temporarily.  This file is allowed to be read by another account called ntamaker (NT Account Maker).  This account is used from the PDC to access this clear text file.  Every 15 minutes, the PDC remotely copies the accounts in this file and adds them to its Active Directory accounts.  This is possible because it has access to clear text passwords for a brief period.  After it creates the accounts, it overwrites the file and erases the password information.

With the common accounts in place, we can then implement centralized file systems.  Since user IDs between different platforms are not usually the same, we are forced to go strictly by username.  This requires that all usernames are unique across all of our platforms.  It has not been an issue for CECS but it could potentially be for other environments.  With the usernames in place, the file servers can use them to distinguish who owns which files.  Thus, when users authenticate to either the NIS namemaster or the PDC, they are allowed to access files that they own.  The other problem that arises is incompatibility between Windows and Unix NFS.  Windows can only mount Samba shares.  This is fixed by running a Samba server on the printservers which is further described in detail in Chapter~\ref{ch:lab_infrastructure}, Section~\ref{sec:printservers}.  

\subsection{Software}
Even though a platform may not always work nicely with another platform, a lot of software has been ported to work across different platforms.  This is nice because certain labs can then be scheduled for use with multiple courses in a curriculum.  For example, Adobe's Web Development Suite has been ported to both Windows and Macintosh platforms.  So either lab can be scheduled for Web Development courses.  Eclipse and other Java based applications work on all platforms that a Java interpreter has been ported to.  This allows some flexibility in lab schedules so that courses that need platform specific software can get it more easily.  In addition, a lab may be able to remotely access servers of another platform.  For example, the Linux labs can be scheduled to use a Windows specific piece of software because students can use the ``rdesktop" command to remotely use a Windows Terminal Server.  

\subsection{Issues}
Issues with multiple platforms are many and varied.  It would be impossible to describe all of the issues we have faced, so I will give some of the more common examples.  

The biggest issue by far occurs when one platform updates and breaks compatibility with another platform.  For example, often during Apple OS X updates, either Samba, NFS, or NIS stops working as it did in previous versions.  Apple sometimes releases fixes after complaints, but we have had to do our own workarounds usually.  For example, when Samba broke, we could no longer bind the Macs to a Windows PDC.  We instead had to bind them to our Unix NIS namemaster.  However, the Macs could not support user home directories being NFS mounted.  After only a few minutes, the workstations would hang with no warning and have to be hard rebooted.  In addition, the error codes had no documentation available to help troubleshoot.  As a workaround, we mounted the directories to an alternate location and created a symbolic link to them.  However, in Apple's next release, even this workaround did not help because NIS was broken.  We ended up having to downgrade the workstations until Apple could release a patch six months later.

Another issue is the occasional failure to do Samba mounts.  This happens because Samba has an operation called an oplock for which NFS does not have an equivalent capability.  We have been able to mostly fix this issue by having the Samba server on the printservers fake oplocks.  We still occasionally have issues but they are minor.  

We also had trouble finding a client on Windows that could communicate via ssh or telnet easily.  When we upgraded to Windows 7, the Hyperterm program was removed.  We eventually found the Putty program that could use ssh and give shell access to Linux servers, but it did not have some of the features that we were used to in Hyperterm.  This invalidated the instruction sets we gave to faculty and caused confusion while we updated.  

\chapter{Conclusion} \label{ch:conclusion}
\section{Summary} \label{sec:summary}
To conclude, the efficient management of the CECS labs is made possible by automation and careful crafting of our infrastructure, which allows us to problem solve a wide range of problems such as tedious deployments and constant updates.  

%For documentation, we use a shared user account and store instructions there.  We maintain a support website and shared technical support email.  Our infrastructure consists of fileservers and printservers.  Our install process involves creating a master lab image and cloning it to many workstations over the network.  Maintenance during the semester is done via Secure Shell pushes and periodic automated cleaning.  Lastly, we maintain a mixed environment with many different platforms across regular labs and specialty labs.  Maintaining all of this can be challenging, but we make it easier through automation. 
The uniqueness of a university setting has allowed us to be creative in our infrastructure implementation.  Our permissions limit students and faculty sufficiently enough to provice minimal security risks while still allowing them to do work in a technical field.  We handled the paper accounting problem by creating the Paperd program and connecting printers to print servers.  All of our main labs are uniformly based on clones of a master image which saves considerable time and effort customizing each one.  Although we have many special labs, the same master image and clone techniques are used in each one to facilitate efficient management with faculty assistance.  Student data is protected and available with two different backups.  Down time is very rare because our file servers are well maintained and reliable.  And although cross-platform interoperability has always had a significant time cost, we, the technicians, have so far managed to find solutions by combining our experience.

It is important to note that our solutions are not the only solutions.  Other universities may be run very differently.  Some universities build their infrastructure with a Microsoft Windows based core instead of Linux.  Others may have chosen to standardize completely on one platform to avoid the various issues we have faced.  Though the technical details vary by university IT environments, the fundamental concept of automation where possible, in addition to building a solid core, is of great importance to all IT environments. 



\appendix
\chapter{Lab Infrastructure Instructions} \label{lab_infrastructure_instructions}

\begin{verbatim}

\end{verbatim}

\chapter{Lab Installation Instructions} \label{ap:lab_installation}

\begin{alltt}
\normalfont
\singlespacing
#############################
Install Master Instructions/Notes

Ubuntu 12.04 LTS Server AMD64
drive that is not nor mounted by
the system when it is in standard use.

Installing.

Do a bare bones install:

Boot the install CDROM.
Partitioned the disk:
 90% ext2 (1st partition)
 10% swap (2nd partition)

Use dhcp (auto), in the lab this will map you to some computer named test
something or other.
Auto time
Enable root login and set a password. 
If this doesn't work, create a dummy account.
No auto update
select no extra packages
Go ahead and install grub (can still use lilo).
select "Continue" to reboot to the hard drive

Set root passwd and remove dummy user if needed:
 userdel -r dummy
   
NOTE if installing on new motherboards:
booted from CD, installed gcc with the packages from CD
    installed 10.04 
    put e1000e drivers (1.9.5) onto flashdrive
    used make install to install from src.
    bring eth0 up from there and do the install as usual.

Install additional items:

apt-get update           (list of packages and repositories)
apt-get -y dist-upgrade  (update/upgrade the packages)

Added packages (see packages file)
 (do this before Edits as the added packages create the auto.master file)
 (Warning: installing nis-common asks for the nis domain name)
 scp npickrel@heart:~conf/admin/linuxlabs/packages .
 apt-get -y install $(grep -v ^# packages)

Disable cups
 apt-get -y remove cups
 apt-get -y autoremove

mv /etc/init/gdm.conf /etc/init/gdm.conf.bak
  we do not want an X11 login screen 
Edit /etc/nsswitch.conf
  passwd, group, shadow changed to "files nis"
  hosts changed to files dns
  (activates on new logins only)
Revert to classic gnome:
  vi /etc/X11/Xsession.d/50x11-common_determine-startup
  replace /usr/bin/x-session-manager with /usr/bin/gnome-session-fallback

Modify the autofs files
 Installed the pre-built files.
  cd ~
  wget http://tiger/export/ubuntu/autofs.tgz
  cd /etc
  tar -xzf ~/autofs.tgz
 Description of the contents of autofs.tgz)
  auto.master: 
   /net/aardvark /etc/auto.aardvark
   /net/d1 /etc/auto.d1
   /net/d2 /etc/auto.d2
   /net/d3 /etc/auto.d3
   /net/a1 /etc/auto.a1
   /net/charlotte /etc/auto.charlotte
  auto.aardvark:
   u1 -ro,nolock,timeo=900,nfsvers=2 aardvrk.cecs.csulb.edu:/u1
  auto.a1:
   u1 -soft,intr,noacl,nolock,timeo=900 a1.cecs.csulb.edu:/u1
  auto.charlotte:
   mail -soft,intr,noacl,nolock,timeo=3600 charlotte.cecs.csulb.edu:/var/spool/mail
  auto.d1:
   u1 -soft,intr,suid,noacl,nolock,quota,timeo=9600 d1.cecs.csulb.edu:/u1
   u2 -soft,intr,suid,noacl,nolock,quota,timeo=9600 d1.cecs.csulb.edu:/u2
   u3 -soft,intr,suid,noacl,nolock,quota,timeo=9600 d1.cecs.csulb.edu:/u3
   u4 -soft,intr,suid,noacl,nolock,quota,timeo=9600 d1.cecs.csulb.edu:/u4
   u5 -soft,intr,suid,noacl,nolock,quota,timeo=9600 d1.cecs.csulb.edu:/u5
   u6 -soft,intr,suid,noacl,nolock,quota,timeo=9600 d1.cecs.csulb.edu:/u6
   u7 -soft,intr,suid,noacl,nolock,quota,timeo=9600 d1.cecs.csulb.edu:/u7
   u1BU -soft,intr,noacl,nolock d1.cecs.csulb.edu:/u1BU
   u2BU -soft,intr,noacl,nolock d1.cecs.csulb.edu:/u2BU
   u3BU -soft,intr,noacl,nolock d1.cecs.csulb.edu:/u3BU
   u4BU -soft,intr,noacl,nolock d1.cecs.csulb.edu:/u4BU
   u5BU -soft,intr,noacl,nolock d1.cecs.csulb.edu:/u5BU
   u6BU -soft,intr,noacl,nolock d1.cecs.csulb.edu:/u6BU
   u7BU -soft,intr,noacl,nolock d1.cecs.csulb.edu:/u7BU
  auto.d2,auto.d3: similar to auto.d1
  cd ~
  rm autofs.tgz
  reboot (to activate)

Edit /etc/group modify names for groups 30 and 40
 faculty:x:30
 student:x:40
 staff:x:50

Configure firefox to have the barefoot home page and no disk caching
 (because we can get it to limit it's cache amount based on the account
  quota)
 Edit /etc/xul-ext/ubufox.js
  pref("browser.startup.homepage", "file:/etc/xul-ext/homepage.properties");
  pref("browser.cache.disk.smart_size.enabled", false);
  pref("browser.cache.disk.enable", false);
  pref("browser.cache.offline.enable", false);
 Create file /etc/xul-ext/homepage.properties with following line
  browser.startup.homepage=http://barefoot.cecs.csulb.edu
 
Configure the mail system
 Change the mail link
  cd /var ; rmdir mail; ln -s /net/charlotte/mail
 Modify the mailer definitions
  cd /etc/mail 
  mv sendmail.mc sendmail.mc.bak
  wget http://tiger/export/ubuntu/sendmail.mc
  make
   This will modify the sendmail.cf, use "service sendmail reload" to activate
  sendmail.mc modifications:
   after the last define add:
    dnl # local setup
    define(`PROCMAIL_MAILER_PATH', `/usr/bin/procmail')dnl
    define(`SMART_HOST', `charlotte.cecs.csulb.edu')dnl
   after the last FEATURE add:
    dnl # local setup
    FEATURE(local_procmail, `', `procmail -t -Y -a $h -d $u')dnl
    MASQUERADE_AS(`cecs.csulb.edu')dnl
    FEATURE(`always_add_domain')dnl
    FEATURE(masquerade_envelope)dnl
    FEATURE(masquerade_entire_domain)dnl
    FEATURE(`allmasquerade')dnl
    MASQUERADE_DOMAIN(localhost)dnl
    MASQUERADE_DOMAIN(localhost.localdomain)dnl
   after the last MAILER add:
    dnl # local setup
    MAILER(`procmail')dnl
   delete the default masquerade list below mailer.

Symlink pine
  ln -s /usr/bin/alpine /usr/bin/pine

Edit /etc/init/control-alt-delete.conf
Comment out  the exec shutdown

Copied namemaster.user.notice (script file) into /usr/bin.
  This file contains a notice to the students to use "search account"...
 cd /usr/bin
 wget http://tiger/export/ubuntu/namemaster.user.notice
 chmod 755 /usr/bin/namemaster.user.notice
Replaced chfn,chsh,passwd,ypchfn,ypchsh,yppasswd with links to above notice
 cd
 wget http://tiger/export/ubuntu/set_namemaster.sh
 source set_namemaster.sh
  What this script does:
   1) Backup: /usr/bin/ ypchfn, ypchsh yppasswd passwd chsh chfn to *.org
   2) Replace all with  links to namemaster.usr.notice

Edit /etc/rc.local
  add powersave definition
 /usr/bin/setterm -blank 15 -powersave powerdown -powerdown 30
  add the call to set the screen to tty0 (alt-F1
 /bin/chvt 1

Edit /etc/logrotate.conf
 modify rotate 4 to rotate 13 (keep a semester's worth of backlogs)

Add foobar public key to /root/.ssh/authorized_keys
 cd
 mkdir .ssh
 chmod 700 .ssh
 cd .ssh
 wget http://tiger/export/authorized_keys2
 mv authorized_keys2 authorized_keys
Add foobar and tiger to known hosts (your are still in .ssh)
 wget http://tiger/export/known_hosts

Make all machines have the same ssh key so foobar can push
  cd  # root private directory
  wget http://tiger/export/ssh.tgz
  cd /etc/ssh
  tar -xzpf ~/ssh.tgz
  rm ~/ssh.tgz  # optional

Get the lab printcap (file has lp coded to 416)
 cd /etc; rm printcap
 wget http://tiger/export/ubuntu/printcap
(clones will change the printcap depending on the lab).

Edit /etc/fstab, remove the UUIDs and replace with /dev/sda1 and /dev/sda2
 /dev/sda1 /
 /dev/sda2 swap
Set options to defaults

Restrict access
 cd /etc/
 rm hosts.allow hosts.deny
 wget http://tiger.cecs.csulb.edu/export/
  hosts.allow
  hosts.deny
  cron.allow

Edit /etc/ntp.conf
  server 134.139.248.10
  server 134.139.248.8
service ntp restart
 (Alternatively: /etc/init.d/ntp restart)
 (if it does not restart, start it manually)

Build and install expired shells
 cd
 wget http://tiger/export/ubuntu/sus.c
 gcc sus.c -o /bin/sus
 gcc sus.c -o /bin/exp

 vi /etc/shells
  add /bin/exp and /bin/sus

/usr/local/bin
  cd
  Download source tarball wget http://tiger/export/ubuntu/local_bin_src.tgz
  tar -xzpf local_bin_src.tgz
  paper: gcc paper.c -o /usr/local/bin/paper
  (if it does not compile, use the -w option to do warnings instead of errors)
  retrieve: gcc retrieve-tar.c -o /usr/local/bin/retrieve
  shotgun (script): cp -p shotgun /usr/local/bin/ 
  diskspace (script): cp -p diskspace /usr/local/bin/ 
  spacecheck.sh (script): cp -p spacecheck.sh /usr/local/bin/
  --resetX (withheld until I figure out X)
  --show_users.sh (not done-withdrawn)
  --mpaper isn't necessary

Get the lilo.conf file:
  wget http://tiger/export/ubuntu/lilo.conf
Contents:
  LBA32
  boot=/dev/sda
  vga = 773
  image = /boot/vmlinuz
    root = /dev/sda1
    label=linux
    read-only

Link in the correct (latest) kernel. Make sure you've rebooted by now.
 cd /boot
 ln -s vmlinuz-<latest kernel>-generic vmlinuz

In /etc/X11/xinit/xinitrc edit, add following line above the Xsession line
 to enable the exit of X with ctrl-alt-bksp:
  setxkbmap -option terminate:ctrl_alt_bksp

In /etc/default/ edit:
  // console-setup: enable exit of X with ctrl-alt-bksp
  //  Set: XKBOPTIONS="lv3:ralt_switch,terminate:ctrl_alt_bksp"
  rcS:
   Set: DELAYLOGIN, VERBOSE and FSCKFIX to yes

/etc/gtk-2.0/
 gtkrc:
  To enable printing in X applications, build this file (mod 644) with the line:
   gtk-print-backends = "lpr,file"

Disable jetty: "rm /etc/rc*.d/S*jetty"

/etc/csh/login.d/sethostname
 create this file, contents (one line): setenv HOSTNAME `/bin/hostname`
/etc/csh/login.d/checkspace
 create this file, contents (one line): /usr/local/bin/spacecheck.sh

Edit: /etc/crontab
 Activate a once per night reboot to clear run away jobs
  0 3 * * * root /sbin/reboot

Allow users to run pmount for their usb drives
 chmod 4755 /usr/bin/pmount 
 chmod 4755 /usr/bin/pumount 

Disable the updating of motd
 unlink /etc/motd
 cp /var/run/motd /etc/motd
 vi /etc/motd (eliminate most lines)
 Add a note on startx and a note to log out when done. 

Disable the updating of resolv.conf
 rm /etc/resolv.conf  (get rid of the link)
 vi /etc/resolv.conf
  nameserver 134.139.249.20
  search cecs.csulb.edu


Add DrJava
 Download the latest drjava jar file (use startx and download via a browser)
  place it in /usr/local/lib
 Place the startup script drjava in /usr/local/bin. Contents of script:
  java -jar /usr/local/lib/drjava-<latest version from lib>.jar
  chown root:root drjava
  chmod 755 drjava

Add Apache Derby eclipse plugins
 Download latest derby plugins (should be 2 zips, may be one)
  unzip or untar the plugins. Should be org.* directories.
   org.apache.derby.core_10.8.2
   org.apache.derby.plugin.doc_1.1.3
   org.apache.derby.ui_1.1.3
 Place in /usr/lib/eclipse/plugins/
   
Remove admin capabilities from X window server:
 cd /etc/polkit-1/localauthority/50-local.d
 wget http://tiger/export/ubuntu/50-admin.pkla
  Contents of 50-admin.pkla:
      [disable suspend]
        Identity=unix-user:*
        Action=org.freedesktop.upower.suspend
        ResultAny=no
        ResultInactive=no
        ResultActive=no
      [disable hibernate]
        Identity=unix-user:*
        Action=org.freedesktop.upower.hibernate
        ResultAny=no
        ResultInactive=no
        ResultActive=no

Disable printing from eclipse
  in /etc/eclipse.ini add the line
   -Dorg.eclipse.swt.internal.gtk.disablePrinting

NEED

Turn off screen locks on gnome. Don't need for the lab
Fix it so it doesn't clear screen on logout (search and account issue).

Making the master image:
 netboot slackware 13
 If you have not done so during a previous install, fdisk /dev/sdb to have
  one Linux ext2 partition and mke2fs on that partition.
 mount /dev/sda1(2) /mnt
 mount /dev/sdb1 /cdrom
 rm /mnt/root/.ssh/known_hosts
 rm /mnt/etc/udev/rules.d/70*
  If you have done a previous install remove or move/backup the previous
  tarball:
 rm /cdrom/ubuntu.tgz
  Build the new tarball
 cd /mnt
 tar -czpf /cdrom/ubuntu.tgz .
  Copy the tarball to the image server
 reboot to hard drive
 login as root
 mount /dev/sdb1 /cdrom
 scp /cdrom/ubuntu.tgz root@tiger:/export/ubuntu/ubuntu.tgz

ssh Linux servers:

  These are 32-bit machines. Installed 32-bit bzImage and 32-bit initrd.img on
  the boot server (linux1).

  On a 32-bit box: booted 10.04 LTS 32-bit server CD
  Built/installed a 32-bit system (similar instructions as above).

  /usr/local/bin:
   built 32-bit versions of paper and retrieve, shotgun is a script so it had
    no problem

  lilo.conf: used "vga = normal" (because the KVM monitor won't support 773).
   Link in the correct kernel
     (the kernel name was different)


  master image:
   had to boot from CD (netboot in ECS 408 is 64-bit)
   called the master image ubuntu32.tgz

Cloning:

# Do the following steps once (unless you have a bug):
On the print server:
  copy in the kernel (bzImage), initrd (initrd.img) and boot message
   (message.txt) into /tftpboot/slackware.
   tgz of slackware is in ~vjd/ubuntu/bootstuff2.tgz, just cd to /tftpboot
   and undo the tar.
 cp slackware boot lines from tiger into /tftpboot/config/default
  (the three lines are in ~vjd/ubuntu/bootstuff1)
  (do not replace the file, add the 3 lines)

# Do the following steps once per loading a lab
On test163 (the Master Image building machine):
  Build the master image into /dev/sdb1 as ubuntu.tgz
   (see install.master)
  copy  the ubuntu.tgz file into /export on the printserver
   scp ubuntu.tgz root@psvr:/export/
On Tiger:
  build the initd.img with the updated ubuntu.install script on tiger,
   move it into /tftpboot/slackware/
   (see makingInitrd)
   Current initrd is in ~root/extractdir.
   Current initrd.img is in ~root/newInitrd
   Current script is ~root/extractdir/usr/lib/setup/ubuntu.install
On the print server (psvr416):
 Do once:
  copy /tftpboot/slackware/* from tiger into /tftpboot/slackware/*
   cd /tftpboot/slackware
   scp root@tiger:/tftpboot/slackware/initrd.img .
  create the exports directory for the image
   mkdir /export
  setup the nfs exports for the room the printserver is in
   vi /etc/exports
    add the line (psvr416)
     /export 134.139.247.192/255.255.255.192(ro,no_subtree_check)
      (psvr412 134.139.247.128, psvr414 134.139.247.64)
   start nfs
     chmod a+x /etc/rc.d/rc.nfsd
     /etc/rc.d/rc.nfsd start
   
 Do each time you clone:
  Copy the master image into the /export dirctory.
   go to the machine with the master image,
   use scp to copy the master image tarball to each of the printers
    (see install_master for detailed command)
  Modify dhcpd.conf to be dhcpd.conf.PXE and restart dhcpd
   cp -p /etc/dhcpd.conf.PXE /etc/dhcpd.conf
   killall -9 dhcpd
   /usr/sbin/dhcpd
  do the cloning
  After the installs disable the PXE boot and restart dhcpd
   cp -p /etc/dhcpd.conf.NOPXE /etc/dhcpd.conf
   killall -9 dhcpd
   /usr/sbin/dhcpd
-------
Clone (i.e. Install) do once per lab machine
boot: Slackware
fdisk /dev/sda
 about 80% sda1 linux(83), 20% sda2 swap(82)
install.YYY XXX  #where YYY is the room number and 
  XXX is the linux # of the machine
reboot

The following should be in the install.416 script:
The other scripts, install.414 install.412  will have different IP numbers
and names.

mkswap /dev/sda2
mke2fs /dev/sda1
mount /dev/sda1 /mnt
modprobe e100
ifconfig eth0 134.139.247.196 netmask 255.255.255.192
route add default gw 134.139.247.193 (may not be needed anymore)
mount -t nfs -o nolock 134.139.247.194:/export /cdrom
cd /mnt
tar -xzpf /cdrom/ubuntu.tgz
mount -o bind /dev /mnt/dev
# The next removes the warning about /proc/partitions not being found
mount -o bind /proc /mnt/proc
chroot /mnt lilo
echo  "ecs416lin$1.cecs.csulb.edu" > etc/hostname
----
ssh linux
 On the boot server (linux1, linux2):
  apt-get -y install atftpd ; edit ftpd out of inetd.conf
  apt-get -y install dhcp3-server ; add MAC addresses to dhcpd.conf
  /tftpboot
   Backed up 32-bit install directory as slackware32
   Created a new slackware directory with the initrd.img and kernel 
    from the ECS405 install of Slackware-13.37 because it handles the
    gigabit ethernet cards on the board.
   Start dhcpd and tftp by hand on the boot server (Ubuntu style):
    service dhcp3-server start
    atftpd --daemon /tftpboot
 On tiger:
  placed the install image in /exports as ubuntu.tgz
 On the ssh linux box we are installing
  BIOS
   SATA must be set to IDE, set boot to hard drive, network, F12 for net
  fdisk /sda
   sda1 linux (80-90% linux)
   sda2 swap (type 82)
  Warning: script is for intalling Slackware 13.37 in ECS 405, you must
    do this by hand (until we build a new initrd.img with a good script)
  mke2fs /dev/sda1
  mkswap /dev/sda2
  mount /dev/sda1 /mnt
  cd /mnt
  ifconfig eth0 134.139.249.xx netmask 255.255.255.192 (see front of box for xx)
  route add default gw 134.139.249.65
   mounted install image directory from tigere
  mount -t nfs -o nolock 134.139.249.167:/export /cdrom
  tar-1.13 -xzpf /cdrom/ubuntu.tgz
 edited /mnt/etc/hostname (name of the machine)
 edited /mnt/etc/lilo.conf
  change vga = 773 to vga = normal because that's what the monitor can do
 edit /mnt/etc/yp.conf
  ypserver 134.139.247.168  (psvr414)
 edit /mnt/etc/printcap  set psvr414 as lp
 edited /mnt/etc/network/interfaces (use static ip)
   iface eth0 inet static
    address 134.139.249.xx
    netmask 255.255.255.192
    gateway 134.139.249.65
  change /mnt/boot/vmlinuz to link to the 37 kernel (this time only)
  copy down e1000e.xx.tgz into /root, make, makeinstall
  installed boot sector
   mount -o bind /dev /mnt/dev
   # The next removes the warning about /proc/partitions not being found
   mount -o bind /proc /mnt/proc
   chroot /mnt lilo
   umount /mnt/dev
   umount /mnt/proc
   umount /mnt
   reboot
   cd e1001-1.9.5./src  (update the kernel network drivers)
   make ; make install

#############################

# NOTE: Packages may be listed even if they were already installed by another package. 
#Installing a package installs the dependencies (prereq's)
#The -y answers "yes" to all question (required in a script)
# Ubuntu 12.04 LTS Server AMD64

#>apt-get -y install 

# apt-get -s simulates
#apt-get install $(<file)

# To do a bulk install:
# cat packages | grep -v ^# > packages_do
# apt-get install $(<packages_do)

#X Server utils fonts (these are not in default fonts for 64-bit install)
xinit 
x11-xserver-utils 
x11-common 
x11-utils 
x11-apps
xfonts-75dpi 
xfonts-100dpi 

#Windows managers
twm 
fvwm 
gnome-app-install 
gnome-desktop-environment
kde-window-manager 
fluxbox
blackbox
openbox
xfce4

#Display managers
#kdm
#(not needed after all, but can be set in /etc/X11/default-display-manager)

# office tools
calligra
openoffice.org
dia
gimp
inkscape

#web browsers and plugins
firefox 
chromium-browser
lynx 
elinks
epiphany-browser

#Email client and utils, chat clients
thunderbird 
mailutils
alpine
sendmail 
sendmail-bin
sendmail-doc
sasl2-bin
biff
pidgin
empathy

# viewers
gv 
poppler-utils
#xpdf (broken in 12.04. But evince works now)
djview

#Compilers and debuggers
gcc 
gcc-doc 
g++ 
#(probably includes the 4.6 jdk files)
gcj-jdk 
ant
gdb 
gdb-doc 
gcl 
gcl-doc
gobjc 
gobjc++ 
gnat 
gnat-doc 
gprolog
gprolog-doc
xxgdb
kcachegrind 
valgrind
doc-base
ruby
scala
scala-doc
clojure
perl
php5-cli
php-doc

#Game dev libraries
libsdl1.2-dev 
libsdl-image1.2-dev 
libsdl-mixer1.2-dev 
libsdl-ttf2.0-dev 
libdevil-dev
glew-utils 
freeglut3
freeglut3-dbg
freeglut3-dev

#Extra Libraries
libboost-all-dev
qt-sdk

#python tools.
python 
python-doc
python-dbg 
python-numpy 
python-numpy-doc 
python-numpy-dbg 
python-scipy

# MPI tools
openmpi-bin 
openmpi-dev 
openmpi-doc 
openmpi-dbg 
python-mpi
python-matplotlib

#IDEs
bison 
bison-doc 
flex
flex-doc
clisp
clisp-dev 
clisp-doc
codeblocks
eclipse 
eclipse-jdt
eclipse-cdt
eclipse-cdt-jni
netbeans
expect-dev

#Code Repository clients
git
git-doc
git-man
subversion
subversion-tools
git-svn

#Flash
nasm  
flasm 
yasm
flashplugin-installer

#Shells
bash
bash-doc
ksh 
tcsh 
zsh 
zsh-doc

#Editors
nvi 
emacs 
elvis
bluefish
gedit
gedit-plugins
vim
vim-doc
vim-gnome
vim-scripts
vim-latexsuite
vim-gui-common
vim-gnome
exuberant-ctags

#tex
ispell 
enscript 
texlive
#(gnu pic)
gputils 

#remote utilities
lftp
openssh-server 
libssl-dev
putty
rsh-client
#RDP Clients
rdesktop 
grdesktop 
xrdp
filezilla
telnet
rsh-client

#Printing
lpr

#Time
ntp 
ntp-doc

#utils
pmount
quota
traceroute
curl
blender
p7zip-full

#Multimedia
mplayer 
mplayer-gui
vlc

#(old crypt replacement)
mcrypt 

#matlab replacement
scilab 
scilab-doc 

#Simulator
spim

#nfs
nfs-common
autofs5

#Database tools
mysql-client
#mysql-query-browser
mysql-workbench

# For 32-bit compatibility
ia32-libs-multiarch

#Bootloader (will prompt)
lilo

#Do this last. (set domain to cecsyp)
nis 

\end{alltt} 

\chapter{Lab Maintenance} \label{ap:lab_maintenance}

\begin{verbatim}

#############################
#!/bin/bash
# resetX

# get the current time (use in naming backup files)
TIME=`date +%y%m%d%H%M` # echo $TIME
#

# save Old Working Directory
OWD="/net/d2/u5/f/conf"

# make sure we are in HOME directory
cd ~

# make a directory to save old conf
SAVE=.oldXconfig.$TIME #echo $SAVE
mkdir -p $SAVE

for FILE in `cat $OWD/Xrc.confiles`  ; do
  if [ -e $FILE ] ; then
     #echo $FILE
     mv -f $FILE $SAVE/
  fi
done
#ls -rlast $SAVE
tar -xf ~conf/data/skel.tar

#/bin/rm -rf /tmp/*$USER* >& /dev/null
#/bin/rm -rf /tmp/.*$USER* >& /dev/null 
/bin/rm -rf /tmp/.* >& /dev/null 
/bin/rm -rf /tmp/* >& /dev/null 

if [ -f ~/Desktop/xterm.desktop ] ; then
	/bin/rm -f ~/Desktop/xterm.desktop
fi
if [ -f ~/Desktop/TerminalServer.desktop ] ; then
	/bin/rm -f ~/Desktop/Terminal\ Server.desktop
fi

/bin/cp -fv ~conf/Desktop/* ~/Desktop/



#############################
#!/bin/sh
# This script takes in a file of student homes, then cds to each.
# mv .xinitrc .xinitrc.bak
v_homefile="$1"
v_filetomv="./.xinitrc"
v_logfile="/tmp/mvlog.log"

while read homedir
do
  echo "Moving to $homedir" >> "$v_logfile" 2>&1
  cd "$homedir"  >> "$v_logfile" 2>&1
  if [ -e "$v_filetomv" ];then
    echo "mving file $v_filetomv to ${v_filetomv}.bak..." >> "$v_logfile" 2>&1
    mv "${v_filetomv}" "${v_filetomv}".bak >> "$v_logfile" 2>&1
  else
    echo "$v_filetomv in $homedir not found." >> "$v_logfile" 2>&1
  fi
done < "$v_homefile"



###############################
# /etc/default/rcS
#
# Default settings for the scripts in /etc/rcS.d/
#
# For information about these variables see the rcS(5) manual page.
#
# This file belongs to the "initscripts" package.

TMPTIME=0
SULOGIN=no
DELAYLOGIN=yes
UTC=yes
VERBOSE=yes
FSCKFIX=yes



#############################
#!/bin/bash

# move to home
cd ~

# now, move down one
cd ..

echo "This will replace ALL the file in your home directory with the the ones from the "
echo "backup server.   The file servers are backed up nightly so the files will be from "
echo "the last backup.  This may not be what you want, so be VERY SURE before you "
echo "answer 'yes'!"



echo "Are you sure you want to do this? (answer with 'yes'!!)"
read ANSWER
if test  $ANSWER != "yes" 
then
	echo "You did not answer 'yes'."
	echo "good bye"
	exit 1
fi

echo "Are you REALY sure? (answer with 'yes'!!)"
read ANSWER
if test $ANSWER != "yes" 
then
	echo "You did not answer 'yes'."
	echo "good bye"
	exit 1
fi

echo "Okay here we go ...."
echo "This could take a while"

# call retrive for home directory
/usr/local/bin/retrieve -f $USER

# move back to home
cd ~

\end{verbaim}

\chapter{Limiting Access}

\begin{verbatim}
########################################
# hosts.allow	This file describes the names of the hosts which are
#		allowed to use the local INET services, as decided
#		by the '/usr/sbin/tcpd' server.
#
# loop back
ALL:127.

# CECS subnets
ALL:134.139.246.
ALL:134.139.247.
ALL:134.139.248.
ALL:134.139.249.
ALL:134.139.250.
ALL:134.139.251.
ALL:134.139.252.
ALL:134.139.253.
ALL:192.168.0.

# ecs faculty offices
#ALL:134.139.60.

########################################
#
# hosts.deny	This file describes the names of the hosts which are
#		*not* allowed to use the local INET services, as decided
#		by the '/usr/sbin/tcpd' server.
#
# The portmap line is redundant, but it is left to remind you that
# the new secure portmap uses hosts.deny and hosts.allow.  In particular
# you should know that NFS uses portmap!
# Deny access to everyone.
ALL: ALL@ALL, PARANOID # Matches any host whose name does not match its address.
########################################
\end{verbatim}

\chapter{Specialty Lab Notes} \label{ap:special}

\begin{alltt}
\normalfont
\singlespacing

#############################
Install cheetah

It's a completely new machine so u3 and u4 had to be copied over
Boot from net (instead of CD, this is minor).
Follow the steps in the install.djv script.
After the script completes make the following changes.

HOSTNAME (cheetah.cecs.csulb.edu)

rc.d/rc.inet1.conf
  eth0 netmask 255.255.255.240

Reboot to hard driver.
mke2fs /dev/sdb1
mke2fs /dev/sdc1
/etc/fstab
 Remove the cheetah entries NFS entries
 add:
  /dev/sdb1 /u3 ext2 defaults 1 1
  /dev/sdc1 /u4 ext2 defaults 1 1
mount -a

#Setup the home directory soft links, they
# are nfs mounts in the standard setup
mkdir /u3 /u4
cd /net/cheetah
rmdir u3 u4  (client nfs mount points)
ln -s /u3
ln -s /u4
# Setup the remote mounts links
mkdir /net/lab /net/aardvark /net/aardvark/u1

On old cheetah save out /etc to /u4:
Make up an empty directory /u4/cheetah_etc
tar -cp . | tar -xp -C /u4/cheetah_etc/
tar up u3 and u4 into /tmp (where djv can get to them)
On new cheetah, cd /root lftp in the tarballs (use djv on old cheetah)

shutdown, insert the network card and boot

On /u4 there is a directory cheetah_etc from which you
can copy stuff into  /etc

in /etc
group
  add
   faculty:x:30:
   student:x:40:
named
  copy in named.conf
   (Had to remove the port entry from the old config)
  copy in /etc/named directory
   cd /etc
   cp -p -r /etc/cheetah_etc/named .
  chmod a+x /etc/rc.d/rc.bind
  /etc/rc.d/rc.bind start   (to test)
netgroup
  copy in the example line
  tail -1 /u4/cheetah_etc/netgroup >> /etc/netgroup
passwd
  copy in from UID #100 on 
shadow
  copy in from UID #100 on
hosts.deny
  ALL:ALL
hosts.allow
  ALL:134.139.0.0/255.255.0.0
  ALL:127.0.0.1/255.255.255.255
ntp.conf
  server 134.139.248.10
  Make sure the restricts are removed since we provide time service
   to other machines.
  (rc.ntpd is a+x from the basic install)
exports
  copy in the exports file
  chmod a+x /etc/rc.d/rc.nfsd
  /etc/rc.d/rc.nfsd start   (to test)
cron.daily/
  backup.conf: copy in 
ypserv
  nisdomainname cecsnet  (set it up manually)
  /var/yp/Makefile (see /u4/cheetah_var_yp/Makefile)
  edit the make file, lower the MIN numbers, don't make the unnecessary
    publickey map
   MINUID=100
   MINGID=30
   shadow # publickey...commented out
  Fix the missing link
  cd /usr/lib
  ln -s /usr/lib64/yp
  /usr/lib64/yp/ypinit -m
  edit rc.yp, uncomment the ypserv lines
  chmod a+x /etc/rc.d/rc.yp   (undo the preboot disable of yp)
  /etc/rc.d/rc.yp start   (to test)
  load the databases from the flat files.
   cd /var/yp
   make
rc.d/rc.inet1.conf
  eth0 134.139.248.17 / 255.255.255.240
  eth1 134.139.248.2 / 255.255.255.248
  GATEWAY 134.139.248.1
rc.d/rc.inet1 add:
/sbin/route add -net 134.139.248.32 netmask 255.255.255.224 gw 134.139.248.18
/sbin/route add -net 134.139.248.64 netmask 255.255.255.224 gw 134.139.248.18


auto.conf
  copy in
  chmod a+x /etc/rc.d/rc.autofs
rc.d/
  check: ntp now has a start up file.

reboot, (replace cheetah) and test

/etc/fstab add
  134.139.248.3:/u1 /net/aardvark/u1 nfs soft 0 0

Set up register and getscore
 gcc ~volper/accounts/register.c -o /usr/local/bin/register
 gcc ~volper/testprograms/getscore.c -o /usr/local/bin/getscore
 chown volper:faculty (both the above)
 chown 4711 (both the above)

##############################
Install Jaguar
Before loading (or on the old machine if you are building a new box)
Backup /etc to /sdc/etc_jaguar (cp -p -r)
Backup /tftpboot (tar file) to /sdc

boot from network
mount /dev/sda1 /mnt
mount /dev/sdc1 /cdrom
cd /mnt
follow the tar and following steps from the install.djv script

HOSTNAME is jaguar.net.cecs.csulb.edu
edit the rc.inet1.conf
 Set up the ethernet cards;
   eth0 134.139.248.18  (HW 00:22:4d:4c:79:d9)
   eth1 134.139.248.33  (HW 00:01:02:76:98:a8)
   eth2 134.139.248.65  (HW 00:a0:cc:29:d0:70)
Modify fstab (and mkdir's)
  /dev/sdb1 /sdb
  /dev/sdc1 /sdc
  ln -s /sdc /load (short cut)

Brute force multiport network configuration.  For this section you may wish
to "chmod a-x /etc/rc.d/rc.yp" so the boots don't wait for yp to time out;
if you do don't forget to "a+x" after you get things setup.
 remove: /etc/udev/ruled.d/70-persistent-net.rules
  (so it gets rebuilt with the card numbers)
 reboot
 edit: /etc/udev/ruled.d/70-persistent-net.rules
   to have the correct eth0, eth1 and eth2 values.
 reboot
  (your mounts and ypbind should now work)

If this is a new computer and not just a reload you will need to
build tarball of /sdb and /sdc, transfer them to the new box and
unwind them.

/etc actions copy in from /sdc/etc_jaguar
exports
  cp in the exports 
  exports the sdb and load directories
  chmod a+x /etc/rc.d/rc.nfsd
Modify /etc/rc.d/rc.syslog so that it starts syslogd with -r
printcap
  copy in the printcap file
setup as a gateway
  chmod a+x /etc/rd.c/rc.ip_forward
dhcpd
  copy in /etc/dhcp.conf
  touch /var/state/dhcp/dhcpd.leases (missed by the install)
Remote boot (make tar of /tftpboot)
  set up the /tftpboot directory
Start tftp
  in.tftpd -l
rc.local
  /usr/sbin/dhcpd
  /usr/sbin/in.tftp -l
rc.lprng
  added before the line starting lpd:
    chown daemon /dev/lp0
  (reboot sets the ownership to root, which is wrong for the daemon)

Network cable note:
 yellow band: to lab34... IP 134.139.248.33
 red band:    to lab66... IP 134.139.248.65
 blue band:   to servers  IP 134.139.248.18

\end{alltt}


%\include{appendix_a} %-Use this to make a thesis office approved TOC, LOF, and LOT
%\include{appendix_b} %-Use this to make a thesis office approved TOC, LOF, and LOT
% \include{detailedDescriptionOfTheCircuit}
% \include{circuitConstruction}

%\bibliographystyle{utphys}  % With journal title
% \bibliographystyle{h-physrev} % No journal title appears
% \bibliography{papers}
%\begin{thebibliography}{9} %set to single digit for no columnation, two digits for columnation upto [99], etc.
% use style guide: IEEE Computer Society Style Guide


% Introduction
\bibitem{re:mri} %report
J. Rouet et al., ``Genetic algorithms for a robust 3-D MR-CT registration,'' in \textit{IEEE Trans. Inf. Technol. Biomed.}, 2000, vol. 4, pp. 126-136.
% @article{
% author = {Jean-michel Rouet and Jean-josé Jacq and Christian Roux},
% title = {Genetic algorithms for a robust 3-D MR-CT registration},
% journal = {IEEE Transactions on Information Technology in Biomedicine},
% volume = {4},
% year = {2000},
% pages = {126--136},
% doi = {10.1109/4233.845205},
% masid = {801806}
% }

\bibitem{re:lightbot} %electronic
% http://gfredericks.com/sandbox/light_bot
G. Fredericks, \textit{Light-Bot Genetic Algorithms,} 2011, June 16;   http://gfredericks.com/sandbox/light\_bot 
% @Electronic{ re:lightbot,
%   author    = "Gary Fredericks",
%   title     = "Light-Bot Genetic Algorithms",
%   url       = "http://gfredericks.com/sandbox/light_bot",
%   month     = jun,
%   year      = "2011",
% }

\bibitem{re:mastermind} %electronic
E. W. Weisstein, \textit{Mastermind,} 2011, June 16; http://mathworld.\\wolfram.com/Mastermind.html


\bibitem{re:knuth} %report/article
% Knuth, Donald (1976-77), "The Computer as a Master Mind," J. Recr. Math. (9): 1–6
D. Knuth, ``The Computer as a Master Mind'', \textit{J. Recr. Math.}, 1976, vol. 9, pp. 1–6.

\bibitem{re:stuckman} %report/article
% @article{DBLP:journals/corr/abs-cs-0512049,
%   author    = {Jeff Stuckman and
%                Guo-Qiang Zhang},
%   title     = {Mastermind is NP-Complete},
%   journal   = {INFOCOMP Journal of Computer Science}
%   volume    = {5},
%   year      = {2006},
		% pages = {25--28},
%   ee        = {http://arxiv.org/abs/cs/0512049},
%   bibsource = {DBLP, http://dblp.uni-trier.de}
% }
J. Stuckman and G. Zhang, ``Mastermind in NP-Complete,'' in \textit{INFOCOMP J. Comput. Sci.}, 2006, vol. 5, pp. 25-28.

\bibitem{re:simulatedannealing}  %report/article
% Kirkpatrick, S.; Gelatt, C. D.; Vecchi, M. P. (1983). "Optimization by Simulated Annealing". Science 220 (4598): 671–680.
S. Kirkpatrick et al., ``Optimization by Simulated Annealing,'' in \textit{Science}, 1983, vol. 220, Issue 4598, pp. 671-680.

\bibitem{re:antcolony} %book
 % Swarm intelligence book pp 105-109
J. Kennedy et al., \textit{Swarm Intelligence}, Morgan Kaufmann Publishers, 2001, pp. 105-109.
% @book{kennedy2001swarm,
%   title={Swarm intelligence},
%   author={Kennedy, J. and Eberhart, R.C. and Shi, Y.},
%   isbn={9781558605954},
%   lccn={00069641},
%   series={The Morgan Kaufmann series in evolutionary computation},
%   url={http://books.google.com/books?id=vOx-QV3sRQsC},
%   year={2001},
%   publisher={Morgan Kaufmann Publishers}
% }

\bibitem{re:originofspecies} %book
% sponge's book p 31-51 
D. B. Fogel, \textit{Evolutionary Computation: Toward a New Philosophy of Machine Intelligence}, 2nd ed., IEEE Press, 2000, pp. 31-51.

\bibitem{re:evolutionaryalgorithms} % swarm book pp 133-186
J. Kennedy et al., \textit{Swarm Intelligence}, Morgan Kaufmann Publishers, 2001, pp. 133-186.


\bibitem{re:selfishgene} %book
% dawkins pp 46-65
R. Dawkins, \textit{The Selfish Gene}, 30th Anniversary ed., Oxford University Press Inc., 1976, pp. 46-65.  

% Overview
\bibitem{re:mutationrate1} %report/article
% DeJong, K.A. and Spears, W.M. "An Analysis of the Interacting Roles of Population Size and Crossover in Genetic Algorithms," Proc. First Workshop Parallel Problem Solving from Nature, Springer-Verlag, Berlin, 1990. pp. 38-47.
K. A. DeJong and W. M. Spears, ``An Analysis of the Interacting Roles of Population Size and Crossover in Genetic Algorithms,'' in \textit{Proc. 1st Workshop Parallel Problem Solving from Nature}, Springer-Verlag, 1990, pp. 38-47.


\bibitem{re:mutationrate2} %report/article
% Grefenstette, J.J. "Optimization of Control Parameters for Genetic Algorithms," IEEE Trans. Systems, Man, and Cybernetics, vol. SMC-16, no. 1, Jan./Feb. 1986, pp. 122-128.
J. J. Grefenstette, ``Optimization of Control Parameters for Genetic Algorithms,'' in \textit{IIEEE Trans. Syst. Man Cybern.}, vol. SMC-16, no. 1, Jan./Feb. 1986, pp. 122-128.

% \bibitem{populationboiling} %

\bibitem{re:diversityincrossover} %report/article/chapter
% @inproceedings{
% author = {Larry Eshelman},
% title = {The CHC Adaptive Search Algo-rithm},
% booktitle = {Foundations of Genetic Algorithms},
% year = {1991},
% masid = {3866190}
% }
L. Eshelman, ``The CHC Adaptive Search Algorithm,'' in \textit{Foundations of Genetic Algorithms}, G. J. E. Rawlins, ed., Morgan Kaufmann Publishers Inc., 1991, pp. 265-283.


\bibitem{re:schema} %book
J. Holland, \textit{Adaptation in Natural and Artificial Systems}, University of Michigan Press, 1975.


% Complications 

\bibitem{re:parallelGAniche} %thesis
S. Mahfoud, ``Niching Methods for Genetic Algorithms,'' Ph.D. dissertation, Dept. General Eng., Univ. Illinois, Urbana-Champaign, 1995.

\bibitem{re:nfl} %report/article
D. H. Wolpert and W. G. Macready, ``No Free Lunch Theorems for Optimization,'' in \textit{IEEE Trans. Evol. Comput.}, vol. 1, no. 67, April 1997.

% Methodology
\bibitem{re:pearson} %report/article
R. L. Plackett, ``Karl Pearson and the Chi-Squared Test,'' in \textit{International Statistical Review / Revue Internationale de Statistique}, vol. 51, no. 1, Apr. 1983, pp. 59-72.
% @Article{ re:pearson,
%   author    = "R. L. Plackett",
%   title     = "Karl Pearson and the Chi-Squared Test",
%   journal   = "International Statistical Review / Revue Internationale de StatistiqueNumerische Mathematik",
%   volume    = "51", 
%   number		= "1",
%   pages     = "59-72",
%   month		= "April",
%   year      = "1978",
% }
\end{document}
%%                                                            %%
%%------------- End of Document ------------------------------%%
