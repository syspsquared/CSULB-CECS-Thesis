\chapter{Introduction} \label{ch:introduction}
\section{The Problem} \label{sec:the_problem}
%# Describe the problem that your case study solved. Not all this faculty garbage.  
University educational lab environments have a very different set of problems to manage than other Information Technology (IT) environments such as those in commercial, government, or corporate settings.  Most work IT environments care about such things as a stable and up to date email system, a defense system to thwart attacks on the outside, theft of trade secrets or misuse of customer information from employees, and creating a permission hierarchy to limit which resources employees can access.  Universities do not usually consider these as important, but they do have other concerns not present elsewhere.  Whereas most IT infrastructures handle groups of employees in various settings, university users are mostly students.  Faculty are also users but have different needs than other employees due to their need to teach courses using a computer lab environment.  

%-Limiting what students can do while still enabling them to do coursework. 
%-Limitations to faculty. 
Whereas employees may have varying levels of permissions, universities care primarily about two groups:  students and faculty.  Limitations on faculty are similar to other employees but faculty must be given enough permissions in order to teach a class and optionally administrate a group of students through a test or an interactive activity.  Full administrative rights on lab workstations are often requested but can be problematic to give to faculty who share a lab.  Thus a limited amount of rights should be given to a faculty group as needed.  For students, very heavy restrictions must be placed to ensure both the health of the workstations and that the students are limited to activities allowed for coursework.  It is important to realize that university environments do not necessarily have to be as restrictive as a lower educational institute.  For example, Elementary schools typically have heavy content filters for web access that a university does not usually need.

%-printing and how students break things more because they aren't really liable.
Printing is a common feature managed in IT environments.  Universities manage printing by contending with printer compatibility and heavy printer use by students.  Students will often use much more paper than employees in their courses.  The paper and ink cost can become exceedingly expensive for a university budget, so a system is often put into place to either limit the pages a student can print or allow them to buy paper as needed.  This paper accounting system may require special printers or separate print servers.  

%-special labs that don't fit with main labs.
A university typically prefers uniform labs as they are easier to maintain.  However, many labs are for special purposes.  For example, engineering and science students may have special test equipment connected to the workstations that needs separate skill sets from technicians.  Some digital design labs may have special licensed software that can only be installed into one dedicated lab for a subset of students.  Teaching IT courses will itself require special labs on isolated networks where students are granted elevated privileges.  Each of these labs must be handled separately and with the proper skill.  However, it can become very costly to manage too many of them. 

%-Handling of student files and media. 
Student data rarely contains anything requiring a high level of confidentiality, such as the case of bank customer data in bank IT environments.  Most of the time, students simply complete assignments in courses.  However, while outside entities may care little for this information, other students can potentially cheat by stealing it.  Care must be taken to ensure internal security of student documents.  The availability of the documents is also very important as downtime during courses can cause delays and potentially shorten a required curriculum.  Nightmare scenarios also appear where students lose access to assignments when they are due.  Care must be taken to ensure availability of any system that students rely on for coursework. 

%-automation of deployment. how to make all machines clones without going to every machine or relying on contract. 
Despite differences, university lab environments and other IT environments share the two largest overall problems, scalability and cross-platform interoperability..  IT staff is budget limited, therefore solutions to manage the lab environments must scale not only to an arbitrary environment size but also stay within the limits of a strictly set budget.  A single technician may be able to build and configure a handful of machines in a short amount of time, but university labs rarely consist of only a handful of machines.  Often, classrooms hold dozens of students that must be using each machine separately.  In large education environments, especially those in technical fields or requiring technical tools, the number of machines to manage can be in the hundreds.  To build and maintain all of them one by one quickly becomes impossible without either increasing the staff to a very expensive size or by managing them with a process that scales to large sizes.  In most cases, this is handled by a system of automated deployment of a lab master image to a large group of clones. 

%-cross-platform interoperability. 
The other main issue facing lab environments is cross-platform interoperability.  Different companies build their products very differently.  For example, account and login information is not uniform across Unix and Microsoft Windows platforms.  This means that users must either have two different sets of accounts and passwords, or some solution must be implemented to make these platforms share that information.   Solutions to interoperability vary.  In many cases, third-party tools can be purchased to handle interoperability at a cost.  Many platforms contain built-in tools that can be used although configuration can take a significant amount of time on the part of the technicians.  In some cases, it is impossible for certain features of a platform to correctly interact with other platforms, so workarounds must be found or the features cannot be used. 


%Doesn't help. Mention very briefly in case study but note that this is a management issue not a technical one.
%\subsection{Limited Faculty Testing}
%One of the main difficulties we face is a lack of faculty feedback during the architectural phase of lab builds.  Requests for changes pre-build come sparingly from a small handful of faculty.  Most faculty will wait until a week before the semester starts before making change or update requests.  Some of them will even wait until the semester starts.

%We understand the various reasons why this may occur.  Most faculty are not paid during the summer months and are, therefore, under no obligation to work.  However, those faculty wishing to avoid this conflict at the beginning of the semester, need only submit requests at the end of the previous semester.  Part-time faculty members face the dilemma of receiving their schedules a few days before or the week of start of term.  Tenured faculty, however, have no set penalties for postponing their lesson planning.  

%Furthermore, this is a problem for us technicians because it limits the amount of testing we can perform on our lab images.  Oftentimes, we lack the knowledge base to test software needed for classes.  We research as much as we can and hope that our constricted knowledge is enough.  While we have a relatively fast lab cloning procedure to recreate the labs in the event of an emergency when the semester starts, in many cases, we must ignore requests from faculty members until the next iteration of the lab images are created for the following semester. 

%doesn't help
%\subsection{Staff and Faculty Turnover}
%In addition to limited faculty testing, we also face a high turnover of faculty.  Part-time lecturers do not stay if their student generated reviews are poor.  Student reviews not only reflect student approval, but also faculty endorsement.  The fast  turnaround, therefore often leaves us technicians with an incomplete faculty list.  Incomplete faculty lists often poses a problem.  We often run into problems where part time staff will not have user accounts to log into our workstations.  Without official verification details, we cannot grant them access to our accounts.  This will cause delays in courses that require lab components. 

%Move to intro to case study
%\subsection{Mixed Environment}
%The fourth floor of the Engineering and Computer Science (ECS) building is dedicated to our labs.  We have 3 Linux labs of 30 workstations each, 4 Microsoft Windows labs of 30 workstations each, an Apple iMac lab of 25 workstations, and several specialty labs.  Operating systems on these workstations are very different from each another.  This requires experience in many areas and maintenance is virtually impossible for a single technician to accomplish alone.  The servers that support the labs are also running different operating systems and require separate levels of experience as well. 

%handled above
%\subsection{Frequent Software Updates}
%There are numerous pieces of software to install and maintain on our workstations.  We create master lab images for each platform and deploy them through a cloning procedure.  Maintenance of these master images is time consuming as there are frequent software updates as well as operating system updates.  Updates while courses are being taught are risky because they have the potential to break some software feature that instructors need to teach. 

%reword as a common issue in IT industry but is often unaddressed.
%\subsection{Limited Staff}
%We have a limited amount of technicians available to us.  This is not uncommon in academia or in industry.  Information Technology (IT) departments are often kept lean because they are seen as an expense.  This results in limits to how much work can be done and how many technicians are available to support a number of workstations.  We improve this situation through automation, but there is only so much we can automate.  Often, we have to postpone needed updates and feature requests until we have enough resources.

%introduce case study

This thesis describes a case study in which I the author helped to manage several university lab environments as a technician.  This environment is specifically for purposes of teaching Computer Engineering and Computer Science.  It has most of the IT issues described in addition to the problem of more periodic updates for an ever changing course curriculum.  The case study describes how I, as part of the technical staff responsible for lab building and maintenance, handled the various issues.
