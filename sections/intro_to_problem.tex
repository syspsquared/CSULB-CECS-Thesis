\chapter{Introduction} \label{ch:introduction}
\section{Overview of the CECS Labs}
The Computer Engineering and Computer Science (CECS) computer labs are student oriented.  Some research can be done with them, but we focus primarily on coursework.  We restrict privileges on them but have a fairly liberal security policy so that students can do development work.  The labs are maintained by a small handful of technicians with a wide breadth of experience.  We report to the department chairman and may or may not be assisted by other members of the faculty. 

We are tasked with building and maintaining everything within our labs including all setup, installation, and maintenance.  We must also handle user support which may include angry students or angry faculty members.  Workloads can vary from slow during the middle of the semester to tight during the beginning or the end.  Most of the setup and installs happens during breaks in the year between semesters.  Large builds are handled during the summer months while small upgrades are made during the month of January.  

Each of the techs specializes in a certain topic, but all technicians know enough about each other's topics to give minor support.  Our technicians consist of a core Linux server administrator, a Linux workstation administrator, two Windows administrators, and a graduate assistant who specializes in Linux.  All of us are capable of building the systems and each of us helps manage a part of the network.  We have no real Macintosh expert, but the combined knowledge we have allows us to get by and learn as we go.  Even with all of our combined experience though, we are still plagued with problems.  Some of these are beyond our control while others we find efficient solutions for. 

\section{The Problem} \label{sec:the_problem}
\subsection{Limited faculty Testing}
One of the main problems we have with faculty is a lack of feedback during the architect phase for lab images.  Requests for changes during the time we build the labs come from only a small handful of faculty.  Most faculty will wait until a week before the semester starts before making change or update requests.  Some of them will even wait until the semester starts.\footnote{These requests are often ignored from faculty who are repeat offenders.}  

We understand why this occurs.  Most faculty are not paid during the summer months and are therefore under no obligation to work.  However, this does not excuse them from submitting requests at the end of the previous semester.  They are also never given any sort of penalty for doing this especially if they are tenured.  Part time faculty also have the problem where they are not told what classes they will be teaching until a few days before semester starts or even after it starts.  

In any case, this is a problem for us because we can do very limited amount of testing on our lab images.  We often do not have the knowledge base to test software needed for classes because we do not understand it.  We research as much as we can and hope that that is enough.  We also have a relatively fast lab cloning procedure to recreate the labs in the event of emergency when the semester starts.  However, in many cases we must ignore requests from faculty members until the next iteration of the lab images the following semester.

\subsection{Staff and Faculty Turnover}
In addition we also have the problem of short faculty turn over.  Our part timer lecturers do not stay if their reviews are poor.  This is good for the students and usually if they are not well liked by the students, they are not well liked by the staff either.  The problem with this turn around is that we do not have a complete list of faculty.  The list is changed often and no attempt is made to keep it updated.  This means that we cannot reliably send out email to all faculty to explain issues or give instructions.\footnote{In many cases, the faculty simply ignore the emails until it starts to negatively affect them. This often results in a blame game.}  In addition, we often run into problems where part time staff will not have user accounts to log into our workstations.  Because we do not know who they are, we cannot give them accounts without verifying details.  This will cause delays in courses that require lab components. 

\subsection{Mixed Environment}
We have the fourth floor of the Engineering and Computer Science building dedicated to our labs.  We have 3 Linux labs of 30 workstations each, 4 Microsoft Windows labs of 30 workstations each, and an Apple iMac lab of 25 workstations.  Operating systems on these workstations are very different from one another.  This requires experience in many areas and is virtually impossible for a single staff member to accomplish alone.  Our servers are also running different operating systems and require separate experience as well.  

\subsection{Frequent Software Updates}
There are numerous pieces of software to install and maintain on our workstations.  We do this by creating master lab images for each platform and deploying them through a cloning procedure.  Maintenance of these master images is time consuming as there are frequent software updates to software on them as well as to the operating systems.  Updates while courses are being taught are difficult because they have the potential to break something that instructors need to be able to teach. 

\subsection{Limited Staff}
We have a limited amount of staff available to us.  This is not uncommon in industry.  IT departments are often kept lean because they are seen as an expense.  This results in limits to how much work can be done and how many technicians are available to support a number of workstations.  We improve this situation through automation, but there is only so much we can automate.  Often, we have to postpone needed updates and feature requests until we have time.   

\subsection{Partial Solution}
Despite the problems we have, the situation is not hopeless.  We do as much as we can to keep things running efficiently.  This paper will document the various steps we take to do so.  However, it will focus on the Linux workstation environment and only touch briefly on other platforms.  

This paper is divided into 5 chapters.  In Documentation Solution, I will describe what we do to keep things documented and how we address some support issues.  In Lab Infrastructure, I will explain what needs to be in place to support our workstations.  This includes a description of our Print Servers, our policy on filesystem management, and how our lab printing system works along with associated costs.  In Lab Installs, I will describe how to build the lab workstation master image and how to clone it across many machines.  In Lab Maintenance, I will describe some of the processes we use to maintain our labs during the semester.  Lastly, in Special Considerations, I will give an overview on our specialty labs and the considerations we take across multiple platforms. 
