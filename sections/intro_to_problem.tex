\chapter{Introduction} \label{ch:introduction}
\section{Overview of the CECS Labs}

The Computer Engineering and Computer Science (CECS) computer labs are student oriented.  Research can be done on them, but the primary focus is on coursework.  We restrict privileges on them, but we have a fairly liberal security policy so that students can do development work.  The labs are maintained by a small handful of technicians with a wide breadth of experience.  We report to the department chairman and are sometimes assisted by other members of the faculty. 

We are tasked with building and maintaining everything within our labs including all setup, installation, and maintenance.  We must also handle user support which may include displeased students and faculty members.  Workloads vary from slow during the middle of the semester, to brisk during the beginning and end of the semester.  Most of the setup and software installs happen during semester breaks.  Large builds are handled during the summer months, while small upgrades are made during the month of January. 

Each of the techs specializes in a certain topic, but all technicians know enough about each other’s specialties to provide minor support.  Our technicians consist of a core Linux server administrator, a Linux workstation administrator, two Windows administrators, and a graduate assistant who specializes in Linux.  All of us are capable of building the systems and each of us help manage a part of the network.  We have no Macintosh expert, but the combined knowledge we have allows us to get by and learn as we go.  Even with all of our combined experiences, we are still plagued with problems.  Some of the complications are beyond our control, but we strive to find efficient solutions for all.

\section{The Problem} \label{sec:the_problem}
\subsection{Limited faculty Testing}
One of the main difficulties we face is with a lack of faculty feedback during the architectural phase of lab images.  Requests for changes pre-build come sparingly from a small handful of faculty.  Most faculty will wait until a week before the semester starts before making change or update requests.  Some of them will even wait until the semester starts\footnote{These requests are often ignored from faculty who are repeat offenders.}. 

We understand why this occurs.  Most faculty are not paid during the summer months and are, therefore, under no obligation to work.  However, those faculty wishing to avoid this conflict at the beginning of the semester, need only submit requests at the end of the previous semester.  Part-time faculty members face the additional dilemma of receiving their schedules a few days before or the week of start of term.  Tenured faculty, however, have no set penalties for postponing their lesson planning.  

Furthermore, this is a problem for us technicians because we can only do limited amounts of testing on our lab images.  Oftentimes, we lack the knowledge base to test software needed for classes.  We research as much as we can and hope that our constricted knowledge is enough.  While we have a relatively fast lab cloning procedure to recreate the labs in the event of an emergency when the semester starts, in many cases, we must ignore requests from faculty members until the next iteration of the lab images are created for the following semester. 

\subsection{Staff and Faculty Turnover}
In addition to limited faculty testing, we also face a high turnover of faculty.  Part-time lecturers do not stay if their student generated reviews are poor.  Student reviews not only reflect student approval, but also faculty endorsement.  The fast  turnaround, therefore provides us technicians with an incomplete list of faculty.  The official faculty list is often neglected and little attempt is made to keep it updated. This administrative problem limits our ability to reliably send out email to all faculty to explain issues or give instructions\footnote{In many cases, the faculty simply ignore the emails until it starts to negatively affect them. This often results in a blame game.}.

Incomplete faculty lists often poses a problem.  We often run into problems where part time staff will not have user accounts to log into our workstations.  Without official verification details, we cannot grant them access to our accounts.  This will cause delays in courses that require lab components. 

\subsection{Mixed Environment}

The fourth floor of the Engineering and Computer Science (ECS) building is dedicated to our labs.  We have 3 Linux labs of 30 workstations each, 4 Microsoft Windows labs of 30 workstations each, and an Apple iMac lab of 25 workstations.  Operating systems on these workstations are very different from each another.  This requires experience in many areas and maintenance is virtually impossible for a single staff member to accomplish alone.  Our servers are also running different operating systems and require separate experiences as well. 

\subsection{Frequent Software Updates}

There are numerous pieces of software to install and maintain on our workstations.  We create master lab images for each platform and deploy them through a cloning procedure.  Maintenance of these master images is time consuming as there are frequent software updates as well as operating system updates.  Updates while courses are being taught are difficult because they have the potential to break some software feature that instructors need to teach. 

\subsection{Limited Staff}

We have a limited amount of staff available to us.  This is not uncommon in industry.  IT departments are often kept lean because they are seen as an expense.  This results in limits to how much work can be done and how many technicians are available to support a number of workstations.  We improve this situation through automation, but there is only so much we can automate.  Often, we have to postpone needed updates and feature requests until we have enough resources.

\subsection{Partial Solution}
Despite the problems we have, the situation is not hopeless.  We do as much as we can to keep things running efficiently.  This paper will document the various steps we take to do so.  However, it will focus on the Linux workstation environment and only briefly touch on other platforms.  

This paper is divided into 5 chapters.  In Documentation Solution, I will describe what we do to keep things documented and how we address some support issues.  In Lab Infrastructure, I will explain what needs to be in place to support our workstations.  This includes a description of our Print Servers, our policy on filesystem management, and how our lab printing system works along with associated costs.  In Lab Installs, I will describe how to build the lab workstation master image and how to clone it across machines.  In Lab Maintenance, I will describe some of the processes we use to maintain our labs during the semester.  Lastly, in Special Considerations, I will give an overview on our specialty labs and the considerations we take across multiple platforms.  

