\section{Unclogging}\label{sec:unclogging}
\subsection{Background}
Sometimes restricting privileges on lab machines is not enough to prevent students from clogging the machines.  If not locked down sufficiently, students can potentially cause headaches to system administrators.  Often, the students are not even malicious.  They could simply be doing assignments that have the potential, if done incorrectly, to overload the lab machines.

For example, in an operating systems course, students must learn how to properly create, fork, kill, and end processes.  As with any programming, practice is needed to properly learn a new programming technique and many mistakes will be made along the way with each program run.  With operating systems concepts, many leftovers will remain after each program run.  Forking of processes will leave behind orphaned children if the program does not exit cleanly.  In addition, if a student's loop runs away, the potential for overloading the system with forked processes exists.  
\subsection{Limiting}
To prevent this, steps need to be taken to limit what a student can do.  For example, a student can run through a finite set of process IDs (PIDs) very quickly with a loop.  If the system does not have a limit for a single user to have, the user can make the system run out of PIDs.  By default, some Linux distributions set the kernel limit for a single user to 1024.  However, on other distributions and other types of Unix, this limit is not necessarily in place or is too high.  Thus, it is important to check this and set it accordingly.  It may be a good idea to drop the process count to below a hundred if the student does not require a window manager (which could also be useful for forcing students to learn good command line habits).  

In addition to PIDs, students can still clog the machine by using too much CPU, memory, or disk space.  CPU and memory usage can both be limited by editing kernel parameters by user.\footnote{On most Unix systems, CPU and memory usage are unlimited by default.}  Disk space can be limited by setting a quota on home directories.  However, users may still be able to write to temporary directories on the local machine and fill those up, so it might be useful to mount those directories from separate partitions of limited size.  Temporary directories may still be important for operating system performance.  If it becomes necessary, a cronjob could be created to delete excess temporary files.  However, this problem is rare, so we have not implemented such capability.  Besides that, each machine can be cloned again within 10-15 minutes.  We have found that it is simpler from an administration perspective to just let the students use the full resources of lab machines (minus administrative privileges) for programming.  

\subsection{Simplest Solution}
For many of these issues, a very simple solution is to schedule a reboot every night.  This clears all orphaned processes, interrupts any processes that could be using resources, and resets the environment.  We have done this by making an entry in root's crontab.  The reboots take place in the middle of the night while the labs are closed. 

This may create a few issues on some Linux distributions though.  On Ubuntu, the file system can get somewhat dirty after a few weeks and on reboot will mount the file system read-only if it detects errors.  To prevent this, we set it to do a File System Check (FSCK) if errors exist.  Delayed login mode is required for this though, but since students are prevented from manually rebooting the machines, this will not affect much.  It also helps that the reboots take place at night.  Lastly, the setup should not require and administrator to interact if problems are found.  We have set FSCK to fix errors automatically without asking for confirmation.  Instructions on how to do this can be found in Appendix~\ref{ap:lab_installation}.

We have also found that booting into verbose mode is very helpful for troubleshooting machines quickly especially while classes are in session.  A splash screen is nice when trying to hide confusing output from non-technical users, but in a room full of computer science students this really is not necessary.  
