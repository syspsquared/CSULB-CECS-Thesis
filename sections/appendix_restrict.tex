\chapter{Limiting Access}

\begin{verbatim}
########################################
# hosts.allow	This file describes the names of the hosts which are
#		allowed to use the local INET services, as decided
#		by the '/usr/sbin/tcpd' server.
#
# loop back
ALL:127.

# CECS subnets
ALL:134.139.246.
ALL:134.139.247.
ALL:134.139.248.
ALL:134.139.249.
ALL:134.139.250.
ALL:134.139.251.
ALL:134.139.252.
ALL:134.139.253.
ALL:192.168.0.

# ecs faculty offices
#ALL:134.139.60.

########################################
#
# hosts.deny	This file describes the names of the hosts which are
#		*not* allowed to use the local INET services, as decided
#		by the '/usr/sbin/tcpd' server.
#
# The portmap line is redundant, but it is left to remind you that
# the new secure portmap uses hosts.deny and hosts.allow.  In particular
# you should know that NFS uses portmap!
# Deny access to everyone.
ALL: ALL@ALL, PARANOID # Matches any host whose name does not match its address.
########################################
\end{verbatim}
