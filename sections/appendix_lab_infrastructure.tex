\chapter{Lab Infrastructure Instructions} \label{ap:lab_infrastructure}

\begin{alltt}
\normalfont
\singlespacing
#############################
Printserver setup:

Machine:
  setup as test160 because that's in the ypserver list of brain.
Remote File systems:
 mkdir /net /net/d1 /net/d2 /net/d3
 edit /etc/rc.local
   add automount for d1, d2, d3
 /etc
   add auto.d1 auto.d2 auto.d3

Time
  edit /etc/rc.d/rc.local
    add netdate call
  edit /etc/ntp.conf
    server time.cecs.csulb.edu
  make sure rc.ntpd is active

NIS:
  /etc/rc.d/rc.yp
    uncomment nisdomain, ypserv and ypbind sections
    eliminate the -broadcast from the ypserv
    chmod a+x rc.yp
  /etc/defaultdomain
    edit to cecsyp
  /etc/yp.conf
    ensure ypbind uses ourself as the ypserver
    ypserver 127.0.0.1
  /etc/nsswitch.conf:
    set to use nis (yp)
     passwd,group,shadow = files nis
  byhand for init:
    niscdomainname cecsyp
  /usr/lib/yp/ypinit -s brain.cecs.csulb.edu
    need authorization for this particular server from brain
  /usr/lib/yp:
    brought in ypxfr_2perhour from another printer
     (this transfers in the passwd map using ypxfer)
  root crontab:
    added ypxfer_2perhour calls at 7
     7,17,27,37,47,57 7-21 * * * /usr/lib/yp/ypxfr_2perhour

Samba:
  smb.conf: from old psvr into /etc/samba/smb.conf
   The faculty subnet needs to be added to smb.conf
   The domain controller is now NTSERVER not MSAD
  chmod a+x /etc/rc.d/rc.samba

Printing:
  ljfilter:
    copied in old (~djv/printing/tcp/* is same as ~prmaster/src/*)
      ljfilter.c connectTCP.h paper.h
    edit ljfilter.c:
      DEVICE is /dev/usb/lp0
      added include of stdlib.h to avoid warnings
      added more robust parsing of input because usb does not line buffer
       and the parallel did (the old parse depended on the line buffering).
    compiled and copied into /usr/local/sbin
       protection: rwsr_x__T root lp
  clean_restart_lpd:
    modified the kill -9 to kill -TERM
    copied into /usr/local/sbin
      note: on standard printer system this was in /usr/local/bin
    edited inittab:
      ca::ctrlaltdel:/usr/local/sbin/clean_restart_lpd
  printcap: (Windows calls the printer LP0)
   lp|LP0:sd=/var/spool/lpd:lp=/dev/null:if=/usr/local/sbin/ljfilter
  print queue display: ***
    bring in /usr/local/showqueue
      (this is a binary, where is source?)
    /etc/rc.local:
      add /usr/local/bin/showqueue &

cron:
  need cron entry for ypxfr (see above)

dhcp:
  bring dhcp conf files into /etc:
    dhcpd.conf dhcpd.conf.NO-PXE dhcpd.conf.PXE
     multiple files are for script turning on/off of netboot
   add to dhcp.conf file "ddns-update-style none;" (new required options)
  rc.local (start the dhcp daemon)
   /usr/sbin/dhcpd
pxe boot:
  rc.local
   /usr/sbin/in.tftpd -l -c -vvvv -s /tftpboot
  /tftpboot (Windows)
    pxelinux.0
    pxelinux.cfg/default
      This is a menu requires the following to exist:
       Mr.Grey.img
       pxekinfe/memory_test/memtest.tgz
       memdisk
    Linux labs have a different tftpboot
  pxe configuration account (snagglepus)
   add account to password file (group users, local home)
   create home directory and .ssh directory for account
   add public key from lab top to account.
   add account to sudoers file with permission to run /root/pxeboot.sh
   add pxeboot.sh and restart.dhcp file to /root


Cloning:
  1) add USB hard drive with tar image
  2) PXE boot in ecs 405 
    boot once to get MAC
    add MAC to dhcp on jaguar and restart
    boot
  3) partition new drive, linux swap space 10%, linux ext2 90%
  4) mkswap, mke2fs on partitions
  5) mount USB image with tar
  6) mount ext2 on new drive and cd into it.
  7) Untar image from USB drive
  8) install boot sector
     mount -o bind /proc /mnt/proc
     mount -o bind /sys /mnt/proc
     chroot /mnt lilo
  9) Edit untar'ed image
     a) change HOSTNAME to test160
     b) change rc.inet1 to test160 ip numbers
     c) chown lp.lp /var/spool/lp; chmod 770 /var/spool/lp ;
        rm /var/spool/lp/* (protections issue)
     d) rm /etc/udev/rules.d/*  (entries for old ethernet MAC and CD id)
     e) rm /etc/ssh/ssh_host*   (keys for the machine we are cloning from)
  10) shutdown, move to ECS 408 and boot
  11) get dhcp config from psvr to vjd directory
  12) get dhcp config from vjd directory to new drive
  13) Edits
     a) add "ddns_update_style none;" line to dhcp.conf's
     b) change HOSTNAME to psvr name
     c) change rc.inet1 to psvr ip numbers (IP_ADDR, GATEWAY)
     d) edit hosts, add psvr (for safety only)
  14) shutdown, move to lab
  15) reconfigure switch to accept new computer
  16) replace printer and boot.
  17) test
    local printing, printing from a lab machine
    z drive access, PXE configuration

#############################
#auto.master - ubuntu

/net/aardvark /etc/auto.aardvark
/net/d1 /etc/auto.d1
/net/d2 /etc/auto.d2
/net/d3 /etc/auto.d3
/net/a1 /etc/auto.a1
/net/charlotte /etc/auto.charlotte

#############################
#auto.d1 - ubuntu

u1 -soft,intr,suid,noacl,nolock,quota,timeo=9600 d1.cecs.csulb.edu:/u1
u2 -soft,intr,suid,noacl,nolock,quota,timeo=9600 d1.cecs.csulb.edu:/u2
u3 -soft,intr,suid,noacl,nolock,quota,timeo=9600 d1.cecs.csulb.edu:/u3
u4 -soft,intr,suid,noacl,nolock,quota,timeo=9600 d1.cecs.csulb.edu:/u4
u5 -soft,intr,suid,noacl,nolock,quota,timeo=9600 d1.cecs.csulb.edu:/u5
u6 -soft,intr,suid,noacl,nolock,quota,timeo=9600 d1.cecs.csulb.edu:/u6
u7 -soft,intr,suid,noacl,nolock,quota,timeo=9600 d1.cecs.csulb.edu:/u7
u1BU -soft,intr,noacl,nolock d1.cecs.csulb.edu:/u1BU
u2BU -soft,intr,noacl,nolock d1.cecs.csulb.edu:/u2BU
u3BU -soft,intr,noacl,nolock d1.cecs.csulb.edu:/u3BU
u4BU -soft,intr,noacl,nolock d1.cecs.csulb.edu:/u4BU
u5BU -soft,intr,noacl,nolock d1.cecs.csulb.edu:/u5BU
u6BU -soft,intr,noacl,nolock d1.cecs.csulb.edu:/u6BU
u7BU -soft,intr,noacl,nolock d1.cecs.csulb.edu:/u7BU

#############################
# DHCP server configuration file (/etc/dhcpd.conf)
# writen 20030526 by malcolm
# updated 20080814 by malcolm
# updated 20120823 by npickrel
# 	
allow booting;
allow bootp;
authoritative;
option domain-name "cecs.csulb.edu";
option domain-name-servers 134.139.249.20;
option routers 134.139.247.193;
option subnet-mask 255.255.255.192;
default-lease-time 21600; # 6 hour lease
max-lease-time 43200;     # max 12 hours
option ntp-servers 134.139.248.10;
ddns-update-style none;

  subnet 134.139.247.192 netmask 255.255.255.192 {
    #range 134.139.247.195 134.139.247.250;
    range 134.139.247.226 134.139.247.252;
    #deny unknown-clients;
    # UNCOMMENT below for network (PXE)boot
    filename "pxelinux.0";
}

\end{alltt}
