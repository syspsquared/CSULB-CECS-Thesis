\chapter{Case Study:  Computer Engineering and Computer Science Labs} \label{ch:introduction_case_study}
\section{Overview of the CECS Labs}

The Computer Engineering and Computer Science (CECS) computer labs are student oriented.  Research can be done with them, but the primary focus is on coursework.  We restrict privileges on them, but we have a fairly liberal security policy so that students can do development work.  The labs are maintained by a small group of technicians with a wide breadth of experience.  We report to the department chairman and are sometimes assisted by other members of the faculty. 

We are tasked with building and maintaining everything within our labs including all hardware and software setup, installation, and maintenance.  We must also handle user support which may include exacting students and faculty members.  Workloads vary from slow during the middle of the semester, to brisk during the beginning and end of the semester.  Most of the setup and software installations occur during semester breaks.  Large builds are handled during the summer months, while small upgrades are made during the month of January.  A common problem with lab builds is a lack of sufficient faculty testing before deployment.  Though it is not a technical issue, as lab maintainers we can only anticipate and test the lab builds as best we can. 

The fourth floor of the Engineering and Computer Science (ECS) building is dedicated to our labs.  We have three Linux labs of thirty workstations each, four Microsoft Windows labs of thirty workstations each, an Apple iMac lab of twenty-five workstations, and several specialty labs.  Operating systems on these workstations are very different from each other.  This requires experience in many areas and maintenance is virtually impossible for a single technician to accomplish alone.  The servers that support the labs are also running different operating systems and require separate levels of experience as well.

Each of the technicians specializes in certain topics, but all technicians know enough about each other’s specialties to provide minor support.  Our technicians consist of a core Linux server administrator, a Linux workstation administrator, two Windows administrators, and myself, a graduate assistant who specializes in Linux.  We have no Macintosh expert, but the combined knowledge we have allows us to get by and learn as we go.  Even with all of our combined experiences, we are still plagued with problems.  Some of the complications are beyond our control, but we strive to find efficient solutions. 

This case study is divided into 5 chapters.  In the chapter on ``Documentation Solution," I will describe what we do to keep things documented and how we address some support issues.  In the chapter on ``Lab Infrastructure," I will explain what needs to be in place to support our workstations.  This includes a description of our Print Servers, our policy on filesystem management, and how our lab printing system works along with associated costs.  In the chapter on ``Lab Installs," I will describe how to build the lab workstation master image and how to clone it across machines.  In the chapter on ``Lab Maintenance," I will describe some of the processes we use to maintain our labs during the semester.  Lastly, in the chapter on ``Special Considerations," I will give an overview on our specialty labs and the considerations we take across multiple platforms.  It is important to note that this paper will focus on the Linux workstation environment and only briefly touch on other platforms.  For most of the techniques used on these workstations, an equivalent exists on the Microsoft Windows and Apple Macintosh platforms.
