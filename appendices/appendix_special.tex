\chapter{Specialty Lab Notes} \label{ap:special}

\begin{alltt}
\normalfont
\singlespacing

#############################
Install cheetah

It's a completely new machine so u3 and u4 had to be copied over
Boot from net (instead of CD, this is minor).
Follow the steps in the install.djv script.
After the script completes make the following changes.

HOSTNAME (cheetah.cecs.csulb.edu)

rc.d/rc.inet1.conf
  eth0 netmask 255.255.255.240

Reboot to hard driver.
mke2fs /dev/sdb1
mke2fs /dev/sdc1
/etc/fstab
 Remove the cheetah entries NFS entries
 add:
  /dev/sdb1 /u3 ext2 defaults 1 1
  /dev/sdc1 /u4 ext2 defaults 1 1
mount -a

#Setup the home directory soft links, they
# are nfs mounts in the standard setup
mkdir /u3 /u4
cd /net/cheetah
rmdir u3 u4  (client nfs mount points)
ln -s /u3
ln -s /u4
# Setup the remote mounts links
mkdir /net/lab /net/aardvark /net/aardvark/u1

On old cheetah save out /etc to /u4:
Make up an empty directory /u4/cheetah_etc
tar -cp . | tar -xp -C /u4/cheetah_etc/
tar up u3 and u4 into /tmp (where djv can get to them)
On new cheetah, cd /root lftp in the tarballs (use djv on old cheetah)

shutdown, insert the network card and boot

On /u4 there is a directory cheetah_etc from which you
can copy stuff into  /etc

in /etc
group
  add
   faculty:x:30:
   student:x:40:
named
  copy in named.conf
   (Had to remove the port entry from the old config)
  copy in /etc/named directory
   cd /etc
   cp -p -r /etc/cheetah_etc/named .
  chmod a+x /etc/rc.d/rc.bind
  /etc/rc.d/rc.bind start   (to test)
netgroup
  copy in the example line
  tail -1 /u4/cheetah_etc/netgroup >> /etc/netgroup
passwd
  copy in from UID #100 on 
shadow
  copy in from UID #100 on
hosts.deny
  ALL:ALL
hosts.allow
  ALL:134.139.0.0/255.255.0.0
  ALL:127.0.0.1/255.255.255.255
ntp.conf
  server 134.139.248.10
  Make sure the restricts are removed since we provide time service
   to other machines.
  (rc.ntpd is a+x from the basic install)
exports
  copy in the exports file
  chmod a+x /etc/rc.d/rc.nfsd
  /etc/rc.d/rc.nfsd start   (to test)
cron.daily/
  backup.conf: copy in 
ypserv
  nisdomainname cecsnet  (set it up manually)
  /var/yp/Makefile (see /u4/cheetah_var_yp/Makefile)
  edit the make file, lower the MIN numbers, don't make the unnecessary
    publickey map
   MINUID=100
   MINGID=30
   shadow # publickey...commented out
  Fix the missing link
  cd /usr/lib
  ln -s /usr/lib64/yp
  /usr/lib64/yp/ypinit -m
  edit rc.yp, uncomment the ypserv lines
  chmod a+x /etc/rc.d/rc.yp   (undo the preboot disable of yp)
  /etc/rc.d/rc.yp start   (to test)
  load the databases from the flat files.
   cd /var/yp
   make
rc.d/rc.inet1.conf
  eth0 134.139.248.17 / 255.255.255.240
  eth1 134.139.248.2 / 255.255.255.248
  GATEWAY 134.139.248.1
rc.d/rc.inet1 add:
/sbin/route add -net 134.139.248.32 netmask 255.255.255.224 gw 134.139.248.18
/sbin/route add -net 134.139.248.64 netmask 255.255.255.224 gw 134.139.248.18


auto.conf
  copy in
  chmod a+x /etc/rc.d/rc.autofs
rc.d/
  check: ntp now has a start up file.

reboot, (replace cheetah) and test

/etc/fstab add
  134.139.248.3:/u1 /net/aardvark/u1 nfs soft 0 0

Set up register and getscore
 gcc ~volper/accounts/register.c -o /usr/local/bin/register
 gcc ~volper/testprograms/getscore.c -o /usr/local/bin/getscore
 chown volper:faculty (both the above)
 chown 4711 (both the above)

##############################
Install Jaguar
Before loading (or on the old machine if you are building a new box)
Backup /etc to /sdc/etc_jaguar (cp -p -r)
Backup /tftpboot (tar file) to /sdc

boot from network
mount /dev/sda1 /mnt
mount /dev/sdc1 /cdrom
cd /mnt
follow the tar and following steps from the install.djv script

HOSTNAME is jaguar.net.cecs.csulb.edu
edit the rc.inet1.conf
 Set up the ethernet cards;
   eth0 134.139.248.18  (HW 00:22:4d:4c:79:d9)
   eth1 134.139.248.33  (HW 00:01:02:76:98:a8)
   eth2 134.139.248.65  (HW 00:a0:cc:29:d0:70)
Modify fstab (and mkdir's)
  /dev/sdb1 /sdb
  /dev/sdc1 /sdc
  ln -s /sdc /load (short cut)

Brute force multiport network configuration.  For this section you may wish
to "chmod a-x /etc/rc.d/rc.yp" so the boots don't wait for yp to time out;
if you do don't forget to "a+x" after you get things setup.
 remove: /etc/udev/ruled.d/70-persistent-net.rules
  (so it gets rebuilt with the card numbers)
 reboot
 edit: /etc/udev/ruled.d/70-persistent-net.rules
   to have the correct eth0, eth1 and eth2 values.
 reboot
  (your mounts and ypbind should now work)

If this is a new computer and not just a reload you will need to
build tarball of /sdb and /sdc, transfer them to the new box and
unwind them.

/etc actions copy in from /sdc/etc_jaguar
exports
  cp in the exports 
  exports the sdb and load directories
  chmod a+x /etc/rc.d/rc.nfsd
Modify /etc/rc.d/rc.syslog so that it starts syslogd with -r
printcap
  copy in the printcap file
setup as a gateway
  chmod a+x /etc/rd.c/rc.ip_forward
dhcpd
  copy in /etc/dhcp.conf
  touch /var/state/dhcp/dhcpd.leases (missed by the install)
Remote boot (make tar of /tftpboot)
  set up the /tftpboot directory
Start tftp
  in.tftpd -l
rc.local
  /usr/sbin/dhcpd
  /usr/sbin/in.tftp -l
rc.lprng
  added before the line starting lpd:
    chown daemon /dev/lp0
  (reboot sets the ownership to root, which is wrong for the daemon)

Network cable note:
 yellow band: to lab34... IP 134.139.248.33
 red band:    to lab66... IP 134.139.248.65
 blue band:   to servers  IP 134.139.248.18

\end{alltt}
